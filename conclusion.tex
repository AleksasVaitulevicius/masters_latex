
Šiame magistro baigiamame darbe atlikta:

\begin{enumerate}
	\item Aprašytos ir realizuotos kapsulinių neuroninių tinklų dvi modifikacijos - daugiavaizdžiai kapsuliniai neuroniniai tinklai, iš kurių su vienas yra su vaizdų sujungimo sluoksniu ir kitas su vaizdų kapsuliniu sluoksniu.
	\item Atlikti tyrimai skirti nustatyti geriausią kapsulinių neuroninių tinklų modifikacijų konfigūracijas ir palyginti jas su nemodifikuotu kapsuliniu neuroniniu tinklu.
	\item Atlikti tyrimai, skirti palyginti kapsulinių neuroninių tinklų modifikacijas su dabartiniu geriausiu daugiavaizdžiu konvoliuciniu neuroniniu tinklu.
\end{enumerate}

Atlikus tyrimus šiame magistro baigiamame darbe gautos tokios išvados:

\begin{enumerate}
	\item Daugiavaizdžiai kapsuliniai neuroniniai tinklai pasiekia geresnius rezultatus per panašų apmokymo laiką nei kapsuliniai neuroniniai tinklai.
	\item Skirtumas tarp daugiavaizdžių kapsulinių neuroninių tinklų, iš kurių vienas yra su vaizdų kapsuliniu sluoksniu ir kitas su vaizdų sujungimo sluoksniu, yra statistiškai nereikšmingas. Tačiau daugiavaizdis kapsulinis neuroninis tinklas su vaizdų kapsuliniu sluoksniu apmokomas vienu etapu pasiekia limitą greičiau daugiavaizdis kapsulinis neuroninis tinklas su vaizdų sujungimo sluoksniu. Tad daugiavaizdis kapsulinis neuroninis tinklas su vaizdų kapsuliniu sluoksniu yra pranašesnis laiko atžvilgiu.
	\item Nei viena kapsulinio neuroninio tinklo modifikacija nėra pranašesnė už dabartinį geriausią daugiavaizdį konvoliucinį neuroninį tinklą.
	\item Kuo mažesnė duomenų imtis tuo skirtumas tarp daugiavaizdžio kapsulinio neuroninio tinklo su vaizdų kapsuliniu sluoksniu, kuris yra apmokomas vienu etapu, ir dabartinio geriausio daugiavaizdžio konvoliucinio neuroninio tinklo yra nereikšmingesnis.
\end{enumerate}

Daugiavaizdžiui konvoliuciniui neuroniniui tinklui yra sudėtinga pritaikyti vaizdų kapsulinį sluoksnį nepaverčiant jo daugiavaizdžiu kapsuliniu neuroniniu tinklu. Tačiau pirmieji kapsulinių neuroninių tinklų sluoksniai sudaro konvoliucinį neuroninį tinklą. Todėl egzistuoja galimybė apjungti VGG-11 architektūrą, kuri naudojama dabartinio geriausio daugiavaizdžio konvoliucinio neuroninio tinklo pirmojo etapo apmokyme, su daugiavaizdžiu kapsuliniu neuroniniu tinklu su vaizdų kapsuliniu sluoksniu. Tad ateityje reikia tyrimais palyginti šį apjungtą dirbtinį neuroninį tinklą su dabartiniu geriausiu daugiavaizdžiu konvoliuciniu neuroniniu tinklu.
