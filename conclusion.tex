
Šiame magistro baigiamame darbe atlikta:

\begin{enumerate}
	\item Aprašytos ir realizuotos kapsulinių neuroninių tinklų dvi modifikacijos - daugiavaizdžiai kapsuliniai neuroniniai tinklai, iš kurių su vienas yra su vaizdų sujungimo sluoksniu ir kitas su vaizdų kapsuliniu sluoksniu.
	\item Atlikti tyrimai skirti nustatyti geriausią kapsulinių neuroninių tinklų modifikacijų konfigūracijas ir palyginti jas su nemodifikuotu kapsuliniu neuroniniu tinklu.
	\item Atlikti tyrimai, skirti palyginti kapsulinių neuroninių tinklų modifikacijas su dabartiniu geriausiu daugiavaizdžiu konvoliuciniu neuroniniu tinklu.
\end{enumerate}

Atlikus tyrimus šiame magistro baigiamame darbe gautos tokios išvados:

\begin{enumerate}
	\item Atlikti eksperimentiniai tyrimai su didele duomenų imtimi indikuoja, kad daugiavaizdžiai kapsuliniai neuroniniai tinklai pasiekia geresnius rezultatus per panašų apmokymo laiką nei kapsuliniai neuroniniai tinklai.
	\item Taip pat šie tyrimai indikuoja, kad daugiavaizdžiai kapsuliniai neuroniniai tinklai su vaizdų kapsuliniu sluoksniu yra pranašesni tikslumo atžvilgiu už daugiavaizdžį kapsulinį neuroninį tinklą su vaizdų sujungimo sluoksniu.
	\item Tyrimai su didele duomenų imtimi indikuoja, kad daugiavaizdžio kapsulinio neuroninio tinklo su vaizdų kapsuliniu sluoksniu, apmokant jį su vienu etapu, tikslumas pradeda konverguoti greičiau nei apmokant jį dvejais etapais. Tad šios daugiavaizdžio kapsulinio neuroninio tinklo modifikacijos apmokymas su vienu etapu yra pranašesnis laiko atžvilgiu už apmokymą su dvejais etapais.
	\item Išvada padaryta iš tyrimų su didele duomenų imtimi yra, kad šie tyrimai indikuoja dabartinio geriausio daugiavaizdžio konvoliucinio neuroninio tinklo pranašumą prieš visas šiame magistro darbe tirtas kapsulinio neuroninio tinklo modifikacijas atsižvelgiant į tikslumo kriterijų.
	\item Tuo metu atlikti eksperimentai su mažesnėmis duomenų imtimis indikuoja, kad kuo mažesnė duomenų imtis, tuo skirtumas tarp daugiavaizdžio kapsulinio neuroninio tinklo su vaizdų kapsuliniu sluoksniu, kuris yra apmokomas vienu etapu, ir dabartinio geriausio daugiavaizdžio konvoliucinio neuroninio tinklo yra nereikšmingesnis.
\end{enumerate}

Daugiavaizdžiui konvoliuciniui neuroniniui tinklui yra sudėtinga pritaikyti vaizdų kapsulinį sluoksnį nepaverčiant jo daugiavaizdžiu kapsuliniu neuroniniu tinklu. Tačiau pirmieji kapsulinių neuroninių tinklų sluoksniai sudaro konvoliucinį neuroninį tinklą. Todėl egzistuoja galimybė apjungti VGG-11 architektūrą, kuri naudojama dabartinio geriausio daugiavaizdžio konvoliucinio neuroninio tinklo pirmojo etapo apmokyme, su daugiavaizdžiu kapsuliniu neuroniniu tinklu su vaizdų kapsuliniu sluoksniu. Tad ateityje reikia tyrimais palyginti šį apjungtą dirbtinį neuroninį tinklą su dabartiniu geriausiu daugiavaizdžiu konvoliuciniu neuroniniu tinklu.
