Kapsuliniai neuroniniai tinklai yra aprašyti darbe \cite{capsNet}. Kapsuliniai neuroniniai tinklai yra giliųjų neuroninių tinklų tipas, kurio sluoksnio perceptronai yra grupuojami į kapsules. Kiekviena kapsulė apskaičiuoja tikimybę, kad paveikslėlyje pavaizduotas objektas priklauso kažkuriai klasei, ir išgauna informaciją apie tokius objekto bruožus kaip pozicija, orientacija, mastelis, deformacija, spalva ir kitus panašius objekto bruožus. Pirminių kapsulių sluoksniuose nagrinėjami objektai yra paprastos geometrinės figūros. Tolimesniuose sluoksniuose objektai darosi sudėtingesni, jie ima atitikti realaus pasaulio objektus. Kapsulės tarp sluoksnių yra sujungiamos į hierarchiją. Taip kapsulinis neuroninis tinklas sukuria hierarchinę vaizdo reprezentaciją.

Pirmieji du sluoksniai kapsuliniame neuroniniame tinkle yra konvoliucijos sluoksnis ir apjungtas netiesiškumo ir ištaisymo sluoksnis su ištaisymo tiesine aktyvacijos funkcija. Šių sluoksnių tikslas yra išgauti pagrindinius požymius, kurie tolimesniame sluoksnyje yra naudojami objektų konstrukcijai.

Tolimesnio sluoksnio tipas yra pirminės kapsulės (angl. primary capsules). Šiame sluoksnyje aktyvacijos žemėlapiai yra konvertuojami į vektorius. Toliau kiekvienas vektorius atskirai yra pateikiamas squash funkcijai kaip argumentai. Squash funkcija yra formulė \ref{label}, kur $||s||$ yra visų matricos $s$ narių suma.

\begin{equation}
\label{eqn:squash}
	squash(s) = \dfrac{||s||^2}{1 + ||s||^2}\dfrac{s}{||s||}
\end{equation}

Tolimesnių sluoksnių tipai yra kapsulės. Šiuose sluoksniuose yra vykdomas dinaminis maršrutizavimas tarp kapsulių (angl. dynamic routing between capsules). Dinaminis maršrutizavimas tarp kapsulių - tai iteratyvus procesas, kurio paskirtis yra apjungti kapsules tarp dviejų sluoksnių. Prieš pradedant iteratyvią proceso dalį, kiekvienai sluoksnio $l$ kapsulei $i$ ir sluoksnio $(l + 1)$ kapsulei $j$ yra inicializuojami kintamieji $b_{ij}$ su reikšme 0. Taip pat kiekvienai kapsulių $i$ ir $j$ porai yra apskaičiuojami vektoriai $\hat{u}_{j|i}$ pagal formulę \ref{eqn:pred_vectors}, kur $W_{ij}$ yra svorio matrica tarp kapsulių $i$ ir $j$ bei $u_{i}$ - tai kapsulės $i$ išvestis.

\begin{equation}
\label{eqn:pred_vectors}
	\hat{u}_{j|i} = W_{ij} u_{ij}
\end{equation}

Tada pirmasis iteratyvaus proceso žingsnis yra apskaičiuoti apjungimo koeficientus $c_{ij}$ kiekvienai kapsulių $i$ ir $j$ porai pagal softmax funkciją atvaizduota formulėje \ref{eqn:coupling_coef}, kur $n$ yra sluoksnio $(l + 1)$ kapsulių skaičius.

\begin{equation}
\label{eqn:coupling_coef}
	c_{ij} = \dfrac{\exp^{b_{ij}}}{\sum_{k = 1}^{n} \exp^{b_{ik}}}
\end{equation}

Tolimesnis žingsnis yra apskaičiuoti svertines sumas $s_j$ kiekvienai kapsulei $j$ naudojantis formulę \ref{eqn:weighted_sum}, kur $m$ yra kapsulių skaičius sluoksnyje $l$.

\begin{equation}
\label{eqn:weighted_sum}
	s_{j} = \sum_{i = 1}^{m} c_{ij} \hat{u}_{j|i}
\end{equation}

Toliau yra apskaičiuojami vektoriai $v_j$ kiekvienai kapsulei $j$ naudojantis softmax funkcija su argumentu $s_j$. Kitaip tariant yra apskaičiuojama formulė $v_j = softmax(s_j)$. Paskutinis iteratyvios dalies žingsnis yra pakeisti kintamųjų $b_{ij}$ reikšmes naudojantis formulę $b_{ij} = b_{ij} + \hat{u}_{j|i} v_j$.

Iteratyvi dinaminio maršrutizavimo tarp kapsulių proceso dalis yra kartojama nurodyta skaičių iteracijų ir šio proceso rezultatas yra vektorius $v_j$. Šiame vektoriuje yra tikimybės, kad objektas, nagrinėjamas kapsulės $i$, yra dalis objekto, nagrinėjamo kapsulės $j$.
