
Vienas iš fundamentalių kompiuterinės regos uždavinių yra informacijos apie 3 dimensijų (3D) pasaulį išgavimas naudojant 2 dimensijų (2D) nuotraukas. Šio uždavinio tikslas yra atpažinti konkrečius 3D objektus, naudojant jų, skirtingų apžvalgos taškų 2D nuotraukas. Šiam tikslui pasiekti, yra konstruojami objektų atpažinimo algoritmai, kurie klasifikuoja 2D nuotraukas į klases, kurios atstovauja vieną iš 3D objektų modelių.

3D objektų atpažinimas iš 2D nuotraukų yra naudojamas srityse, kuriose turimi 3D objektai turi būti atpažinti iš visų galimų 2D nuotraukų, turint tik poaibį šių nuotraukų. Keli šių sričių pavyzdžiai yra vogtų objektų aptikimas - turint algoritmą, atpažįstantį konkretų automobilį, galima iš viešų erdvių nuotraukų atrasti automobilio poziciją, vietos nustatymas - turint algoritmą, atpažįstantį konkrečius objektus esančius skirtingose vietovėse, ir tų vietovių koordinates galima nustatyti kurioje vietovėje buvo padaryta nuotrauka. Deja, laiko ir duomenų kaštai yra per dideli, kad pasiekti absoliutų tikslumą sprendžiant šį uždavinį. Todėl taikomi metodai yra euristiniai. Dėl to renkantis metodą, spręsti 3D objektų atpažinimo iš 2D nuotraukų uždaviniui, reikia atsižvelgti į laiko kaštus ir kaip tiksliai tuo metodu pagrįstas algoritmas klasifikuoja 2D nuotraukas, spręsdamas šį uždavinį. Šiame darbe, bus atliekami tyrimai, skirti nustatyti metodą, kuris būtų pranašesnis sprendžiant 3D objektų atpažinimo iš 2D nuotraukų uždavinį. Pasirinkti metodai yra konvoliucinio ir CapsNet neuroninio tinklo architektūros.

Gana dažnai naudojamas metodas, šiam uždaviniui spręsti, yra dirbtiniai gilieji neuroniniai tinklai. 3D objektų atpažinimo iš 2D nuotraukų uždavinyje, naudojami duomenys - 2D nuotraukos, yra nestruktūrizuoti, jiems sudėtinga vykdyti požymių išgavimą. Todėl daugelis kitų sprendimų nėra tokie patrauklūs kaip dirbtiniai gilieji neuroniniai tinklai, dėl savo sugebėjimo efektyviai vykdyti automatinį požymių išgavimą iš nestruktūrizuotų duomenų. Tačiau, norint pasiekti aukštą klasifikavimo tikslumą, naudojant šį metodą, yra reikalingas didelis kiekis duomenų. Konkrečiai šiam uždaviniui reikia didelio kiekio 3D modelių. Laimei, šiuo metu egzistuoja viešai prieinamos didelės 3D repozitorijos. Tokios kaip 3D Warehouse, TurboSquid, ir Shapeways. Dėl to, šiuo metu daugelis architektūrų jau yra išbandytos sprendžiant 3D objektų atpažinimo iš 2D nuotraukų uždavinį. Pavyzdžiui viena iš architektūrų, kuri buvo išbandyta, yra gilaus pasitikėjimo architektūra. Šiai architektūrai atlikti tyrimai yra aprašyti darbe \cite{dbnExp}.

Šiuo metu, šiam uždaviniui spręsti, optimaliausius  rezultatus, laiko ir tikslumo atžvilgiu, pasiekusi dirbtinio neuroninio tinklo architektūra yra konvoliuciniai neuroniniai tinklai. Tyrimai, kuriuose ši architektūra buvo išbandyta, yra aprašyta darbuose \cite{cnnExp1, cnnExp2}. Darbe \cite{dbnExp} atlikto tyrimo rezultatai parodo, kad pateiktas sprendimas, kuriame 3D objektų atpažinimas yra konstruojamas naudojantis tik 2D nuotraukomis, yra tikslesnis 8\% (77\% → 85\%). Todėl šiame darbe bus daromi eksperimentai su dirbtinio neuroninio tinklo architektūrų įgyvendinimais, kurie yra pagrįsti būtent šiuo metodu. Tad šiam darbui, vienas iš pasirinktų metodų yra konvoliuciniai neuroniniai tinklai, kurio įgyvendinimas ir tyrimai yra aprašyti darbe \cite{cnnExp1}. Mat šio darbo įgyvendinimas naudoja tik 2D nuotraukas, konstruojant 3D objektų atpažinimo algoritmą.

Kita tiriama dirbtinio neuroninio tinklo architektūra yra CapsNet. Lyginant su konvoliuciniu neuroniniu tinklu, tai gana nauja architektūra. Aprašyta 2017 metais darbe \cite{capsNet} CapsNet architektūros veikimo principas tiksliau imituoja žmogaus rega, remiantis faktu, kad žmogaus rega ignoruoja nereikšmingas vaizdo detales, naudodama tik seką fokusuotų taškų, taip apdorodama tik dalį vaizdinės informacijos su labai aukšta rezoliucija. \cite{capsNet} darbe atliktas tyrimas parodo, kad ši architektūra atlieka skaičių klasifikavimo užduotį tiksliau nei konvoliucinis neuroninis tinklas. Tad 