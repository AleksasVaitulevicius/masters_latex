
Vienas iš fundamentalių kompiuterinės regos uždavinių yra informacijos apie trijų dimensijų (3D) pasaulį išgavimas naudojant dviejų dimensijų (2D) nuotraukas. Šio uždavinio tikslas yra atpažinti konkrečius 3D objektus, naudojant jų, 2D nuotraukas, padarytas iš skirtingų kampų. Šiam tikslui pasiekti, yra konstruojami objektų atpažinimo algoritmai, kurie klasifikuoja 2D nuotraukas į klases, kurios atstovauja vieną iš 3D objektų modelių.

3D objektų atpažinimas iš 2D nuotraukų yra naudojamas srityse, kuriose turimi 3D objektai turi būti atpažinti iš visų galimų 2D nuotraukų, turint tik šių nuotraukų poaibį. Pavyzdžiui, vienas iš šių sričių yra automatinė objektų apžiūra - turint algoritmą, atpažįstantį konkretų objektą, kuris turi tik jam būdingą 3D formą, galima nustatyti nuotraukas, kuriose yra tas objektas. Kitas pavyzdys yra navigacija - turint algoritmą, atpažįstantį konkrečius objektus, esančius skirtingose vietovėse, ir tų vietovių koordinates, galima nustatyti kurioje vietovėje buvo padaryta nuotrauka. Deja, laiko ir duomenų kaštai yra per dideli, kad pasiekti absoliutų tikslumą sprendžiant šį uždavinį. Todėl taikomi metodai yra euristiniai. Dėl to renkantis metodą, spręsti 3D objektų atpažinimo iš 2D nuotraukų uždaviniui, reikia atsižvelgti į laiko kaštus ir kaip tiksliai tuo metodu pagrįstas algoritmas klasifikuoja 2D nuotraukas, spręsdamas šį uždavinį. Šiame magistro baigiamajame darbe bus atliekami tyrimai, skirti nustatyti metodą, sprendžiantį 3D objektų atpažinimo iš 2D nuotraukų uždavinį, kuris pasiekia didžiausią tikslumą ir reikalauja mažiausiai laiko mokymui.

Gana dažnai metodas, naudojamas šiam uždaviniui spręsti, yra dirbtiniai gilieji neuroniniai tinklai. 3D objektų atpažinimo iš 2D nuotraukų uždavinyje naudojami duomenys yra 2D nuotraukos, yra nestruktūrizuoti, jiems sudėtinga vykdyti požymių išgavimą. Todėl daugelis kitų sprendimų nėra tokie patrauklūs kaip dirbtiniai gilieji neuroniniai tinklai, dėl savo sugebėjimo efektyviai vykdyti automatinį požymių išgavimą iš nestruktūrizuotų duomenų. Tačiau, norint pasiekti aukštą klasifikavimo tikslumą, naudojant šį metodą, yra reikalingas didelis kiekis duomenų. Konkrečiai šiam uždaviniui reikia didelio kiekio 3D modelių. Laimei, šiuo metu egzistuoja viešai prieinamos didelės 3D repozitorijos. Tokios kaip 3D Warehouse, TurboSquid, ir Shapeways. Dėl to šiuo metu daugelis senesnių architektūrų jau yra išbandytos sprendžiant 3D objektų atpažinimo iš 2D nuotraukų uždavinį. Pavyzdžiui, viena iš architektūrų, kuri buvo išbandyta, yra konvoliucinio gilaus pasitikėjimo neuroninio tinklo (angl. convolutional deep belief neural network) architektūra. Šiai architektūrai atlikti tyrimai yra aprašyti darbe \cite{dbnExp}. Tačiau 2017 metais buvo aprašyta nauja architektūra, tai kapsuliniai neuroniniai tinklai. Tyrimai parodė, kad ji yra pranašesnė tikslumo atžvilgiu už ankstesnes architektūras, sprendžiant uždavinius panašius į 3D objektų atpažinimo iš 2D nuotraukų uždavinį.

Šiuo metu šiam uždaviniui spręsti, optimaliausius  rezultatus, laiko ir tikslumo atžvilgiu, pasiekusi dirbtinio neuroninio tinklo architektūra yra konvoliucinių neuroninių tinklų tipas - daugiavaizdžiai konvoliuciniai neuroniniai tinklai (angl. multi-view convolutional neural networks). Tyrimai, kuriuose ši architektūra buvo išbandyta, yra aprašyta darbuose \cite{cnnExp1, cnnExp2}. Darbe \cite{dbnExp} atlikto tyrimo rezultatai parodo, kad pateiktas sprendimas, kuriame 3D objektų atpažinimas yra konstruojamas naudojantis tik 2D nuotraukomis, yra tikslesnis 8~\%. Algoritmas, naudojantis 3D modelius, pasiekė 77~\% tikslumą, o algoritmas, naudojantis tik 2D nuotraukas, pasiekė 85~\% tikslumą. Todėl šiame magistro baigiamajame darbe bus daromi eksperimentai su dirbtinio neuroninio tinklo architektūrų įgyvendinimais, kurie yra pagrįsti būtent šiuo metodu. Tad šiam darbui vienas iš pasirinktų metodų yra daugiavaizdis konvoliucinis neuroninis tinklas, kurio įgyvendinimas yra aprašyti darbe \cite{cnnExp1}. Mat šio darbo įgyvendinimas naudoja tik 2D nuotraukas, konstruojant 3D objektų atpažinimo algoritmą.

Kita tiriama dirbtinio neuroninio tinklo architektūra yra kapsuliniai neuroniniai tinklai. Lyginant su konvoliuciniais neuroniniais tinklais, tai gana nauja architektūra. Aprašytos 2017 metais darbe \cite{capsNet} kapsulinių neuroninių tinklų architektūros veikimo principas tiksliau imituoja žmogaus rega, remiantis faktu, kad žmogaus rega ignoruoja nereikšmingas vaizdo detales, naudodama tik seką fokusuotų taškų, taip apdorodama tik dalį vaizdinės informacijos su labai aukšta rezoliucija. Darbe \cite{capsNet} atliktas tyrimas parodo, kad ši architektūra atlieka ranka rašytų skaičių klasifikavimo užduotį tiksliau nei konvoliuciniai neuroniniai tinklai. Kitas tyrimas, kuris yra atliktas darbe \cite{capsCNN} su 4 duomenų rinkiniais, kuriuose yra veidai, kelio ženklai ir kasdieniai objektai, parodo, kad dabartiniai kapsuliniai neuroniniai tinklai ne visada yra pranašesni už konvoliucinius neuroninius tinklus. Parinkus geresnius parametrus ir modifikacijas (sluoksnių skaičių, neuronų skaičių kiekviename sluoksnyje, aktyvacijos funkcijos), konvoliucinis neuroninis tinklas dar vis būna pranašesnis už kapsulinį neuroninį tinklą. Tačiau taip pat darbe \cite{capsCNN} yra teigiama, kad kapsuliniai neuroniniai tinklai dar nėra pasiekę pilno savo potencialo ir tolimesni išsamesni tyrimai yra būtini.

Tad šio darbo \textbf{tikslas} yra įrodyti arba paneigti keliamą \textbf{hipotezę}:

\textit{Kapsuliniai neuroniniai tinklai sprendžia 3D objektų atpažinimo iš 2D nuotraukų uždavinį efektyviau nei konvoliuciniai neuroniniai tinklai remiantis apmokymo laiko ir tikslumo kriterijais}.

Tikimasi, kad, sprendžiant 3D objektų atpažinimo iš 2D nuotraukų uždavinį, kapsulinio neuroninio tinklo mokymas truks trumpiau nei konvoliucinio neuroninio tinklo. Taip pat, kad apmokytas kapsulinis neuroninis tinklas vykdys klasifikavimą tiksliau nei daugiavaizdis konvoliucinis neuroninis tinklas.

Siekiant patikrinti iškeltą hipotezę reikia atlikti šiuos uždavinius:

\begin{enumerate}
	\item Išanalizuoti ir nustatyti dabartinį efektyviausią 3D objektų atpažinimo iš 2D nuotraukų uždavinio sprendinį.
	\item Išanalizuoti kapsulinių neuroninių tinklų veikimą.
	\item Surasti duomenis, skirtus spręsti 3D objektų atpažinimo iš 2D nuotraukų uždaviniui.
	\item Eksperimentiškai nustatyti efektyviausią modifikaciją su tinkamiausiais parametrais, skirtus spręsti 3D objektų atpažinimo iš 2D nuotraukų uždaviniui, kapsulinio neuroninio tinklo realizacijai, remiantis apmokymo laiko ir tikslumo kriterijais.
	\item Atlikti eksperimentus, skirtus palyginti kapsulinio neuronino tinklo ir daugiavaizdžio konvoliucinio neuronino tinklo tikslumą ir apmokymo laiką, sprendžiant 3D objektų atpažinimo iš 2D nuotraukų uždavinį.
\end{enumerate}

Šiame magistro baigiamajame darbe laukiami rezultatai:

\begin{enumerate}
	\item Nustatyta, kad šiuo metu efektyviausias 3D objektų atpažinimo iš 2D nuotraukų uždavinio sprendinys yra daugiavaizdžiai konvoliuciniai neuroniniai tinklai, lyginant eksperimentų, aprašytų skirtinguose literatūros šaltiniuose, rezultatus. Šiuose šaltiniuose buvo surasta daugiavaizdžio konvoliucinio neuroninio tinklo realizacija ir duomenys skirti apmokymui ir testavimui.
	\item Išanalizuotas kapsulinių neuroninių tinklų veikimas, surasta jo realizacija.
	\item Planuojama eksperimentiškai nustatyti efektyviausią kapsulinio neuroninio tinklo konfigūraciją, skirtą spręsti 3D objektų atpažinimo iš 2D nuotraukų uždavinį, naudojantis apmokymo laiko ir tikslumo kriterijais.
	\item Planuojama eksperimentiškai palyginti kapsulinio neuroninio tinklo tikslumą ir apmokymo laiką su konvoliuciniu neuroniniu tinklu, naudojantis apmokymo laiko ir tikslumo kriterijais.
\end{enumerate}

Darbas remiasi šiomis prielaidomis:

\begin{enumerate}
	\item Kiekvienam 2D paveikslėliui yra priskirtas jį atitinkantis 3D objektas.
	\item Kiekvienas 3D objektas turi bent po vieną jį atitinkantį 2D paveikslėlį.
\end{enumerate}

Šio darbo turinys yra sudarytas iš 4 skyrių. Pirmame skyriuje yra pateikiama literatūros analizė. Jame yra pateiktas 3D objektų atpažinimo iš 2D nuotraukų uždavinio aprašymas, egzistuojančių sprendimų apžvalga, bendrieji neuroninių tinklų principai, daugiavaizdžio konvoliucinio neuroninio tinklo veikimo aprašymas ir kapsulinio neuroninio tinklo aprašymas.
Tada antrame skyriuje yra pateikiami šiame magistro baigiamajame darbe bandomų kapsulinių neuroninių tinklų modifikacijos ir parinkti parametrai.
Trečiame skyriuje yra aprašomi tyrimams naudoti duomenys.
Taip pat šiame skyriuje yra aprašomi tyrimai, skirti nustatyti kapsulinių neuroninių tinklų modifikaciją ir parametrus, kurie pasiekia didžiausią tikslumą ir reikalauja mažiausiai laiko apmokymui, sprendžiant 3D objektų atpažinimo iš 2D nuotraukų uždavinį.
Galiausiai šiame skyriuje yra aprašomi tyrimai, skirti palyginti kapsulinių neuroninių tinklų ir daugiavaizdžių konvoliucinių neuroninių tinklų tikslumą ir apmokymo laiką, sprendžiant 3D objektų atpažinimo iš 2D nuotraukų uždavinį.
Paskutiniame skyriuje yra pateikiami rezultatai ir išvados.
