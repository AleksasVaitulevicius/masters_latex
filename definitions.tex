
\begin{enumerate}
	\item AdaDelta - Prisitaikančios deltos optimizavimo algoritmas (angl. adaptive delta)
	\item AdaGrad - Prisitaikančio gradiento optimizavimo algoritmas (angl. adaptive gradient)
	\item Adam - Inercijos apskaičiavimo optimizavimo algoritmas (angl. adaptive moment estimation)
	\item Apibrėžiantis stačiakampis (angl. bounding box)
	\item Apmokymas be mokytoju (angl. unsupervised learning)
	\item Apmokymas su mokytoju (angl. supervised learning)
	\item Daugiasluoksnis perceptronas (angl. multilayer perceptron)
	\item Daugiavaizdžiai konvoliuciniai neuroniniai tinklai (angl. multi-view convolutional neural networks)
	\item Dinaminis maršrutizavimas tarp kapsulių (angl. dynamic routing between capsules)
	\item Dirbtiniai neuroniniai tinklai (angl. artificial neural networks)
	\item Duomenų rinkinys (angl. batch)
	\item F-dalių kryžminis validavimas (angl. f-folds cross validation)
	\item Gabalais tiesinė funkcija (angl. piecewise linear function)
	\item Gauso funkcija (angl. Gaussian function)
	\item Gilieji neuroniniai tinklai (angl. deep neural networks)
	\item Gylio nuotrauka (angl. depth image)
	\item Grįžtamojo ryšio neuroniniai tinklai (angl. feedback neural networks)
	\item Inercija (angl. momentum)
	\item Inercijos konstanta (angl. momentum constant)
	\item Išretinta matrica (angl. sparse matrix)
	\item Išretinta sąveika (angl. sparsity)
	\item Ištaisymo sluoksnis (angl. rectification layer)
	\item Klaidos sklidimo atgal algoritmas (angl. back-propagation learning algorithm)
	\item Konvoliucinis gilaus pasitikėjimo neuroninis tinklas (angl. convolutional deep belief neural network)
	\item Kreivių palyginimas (angl. curve matching)
	\item Kryžminė entropija (angl. cross-entropy)
	\item leaky ReLU - nesandari ištaisymo tiesinė funkcija (angl. leaky rectified linear function)
	\item Maksimalus sujungimas (angl. max pooling)
	\item Mokymo greitis (angl. learning rate function)
	\item Mokymosi žingsnio dydžiu (angl. learning step size)
	\item Nepersidengiančiu sujungimo sluoksnis (angl. non-overlapping pooling layer)
	\item Netiesiškumo sluoksnis (angl. non-linearity layer)
	\item noisy ReLU - Ištaisymo tiesinė funkcija su triukšmu (angl. noisy rectified linear function)
	\item Nulinis svoris/slenkstis (angl. bias)
	\item Nuostolių funkcija (angl. loss function/cost function/objective function)
	\item Orientuotas beciklis grafas (angl. directed acyclic graph)
	\item Panašumo metrikos (angl. similarity metrics)
	\item Persidengiantis sujungimo sluoksnis (angl. overlapping pooling layer)
	\item Persimokymas (angl. overfitting)
	\item Pilnai sujungtas sluoksnis (angl. fully connected layer)
	\item Pirminių kapsulių sluoksnis (angl. primary capsules)
	\item Postūmio matrica (angl. bias matrix)
	\item Požymių žemėlapiai (angl. feature map)
	\item RMSProp - Šaknies vidurkio kvadrato išskleidimo optimizavimo algoritmas (angl. root mean square propagation)
	\item ReLu - Ištaisymo tiesinė funkcija (angl. rectified linear function)
	\item Sigmoidinė funkcija (angl. sigmoid function)
	\item Skiriamasis paviršius (angl. decision boundary)
	\item Slenkstinė funkcija (angl. unit step function)
	\item Sujungimo sluoksnis (angl. pooling layer)
	\item Šviesos lauko deskriptorius (angl. light field descriptor)
	\item Tiesinė funkcija (angl. linear function)
	\item Tiesioginio sklidimo neuroniniai tinklai (angl. feedfoward neural networks)
	\item Tikrinis vektorius (angl. eigenspace)
	\item Vaizdų sujungimo sluoksnis (angl. view pooling layer)
	\item Vidutinis sujungimas (angl. average pooling)
	\item Vidutinė kvadratinė paklaida (angl. mean square error)
\end{enumerate}