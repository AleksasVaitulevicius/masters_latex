
3D objektų klasifikavimo naudojantis 2D nuotraukomis uždavinys -- tai klasifikavimo uždavinys, kuriame pateiktos 2D nuotraukos, kuriose atvaizduotas 3D objektas iš atsitiktinio apžvalgos taško, turi būti priskirtos 3D modelio, kuris yra atvaizduotas toje 2D nuotraukoje, klasę.

Klasifikavimo uždavinys -- tai uždavinys, kurio tikslas yra automatiškai nustatyti pavyzdžio iš tiriamos srities populiacijos klasę. 3D objektų atpažinimo iš 2D nuotraukų uždavinio atveju, tiriama sritis yra 2D nuotraukos, kuriose yra atvaizduotas 3D objektas iš bet kurio apžvalgos taško ir klasė - 3D objektas. Taip pat, šio magistro baigiamojo darbo atveju, metodas yra dirbtinio neuroninio tinklo (kapsulinio arba konvoliucinio) apmokytas modelis.

Kaip jau minėta įvade, šiam uždaviniui spręsti efektyviausia yra naudoti mašininio mokymo metodą, kurio mokymo duomenys yra tik 2D nuotraukos, o klasės reprezentuoja 3D objektus. Darbe \cite{dbnExp} atlikto tyrimo rezultatai parodo, kad pateiktas sprendimas, kuriame 3D objektų klasifikavimas yra konstruojamas naudojantis tik 2D nuotraukomis, yra tikslesnis 8 \%. Algoritmas, naudojantis 3D modelius, pasiekė 77 \% tikslumą, o algoritmas, naudojantis tik 2D nuotraukas, pasiekė 85 \% tikslumą. Šaltinyje \cite{cnnExp1} yra teigiama, kad to priežastis yra reliatyviai efektyvesnis 2D nuotraukų informacijos saugojimas negu 3D modelių. Todėl, kad, nors 3D modelis turi visą informaciją apie atvaizduotą 3D objektą, tačiau tam, kad panaudoti vokselinę 3D objekto reprezentaciją mašininiame mokyme, kurio mokymas su pakankamai didele duomenų imtimi užtruktų racionalų laiko tarpą, tenka ženkliai sumažinti 3D modelio rezoliuciją. Pavyzdžiui, 3D modelio, kurio rezoliucija yra $30\times30\times30$ vokseliai, įvesties dydis yra apytiksliai lygus 2D paveikslėlio, kurio rezoliucija yra $164\times164$ pikseliai, kur vokselis yra vienas taškas trimatėje erdvėje. Tad šiuo atveju, 3D modelis yra apdorojamas per tiek pat laiko kaip ir 2D paveikslėlis, bet modelio rezoliucija yra apytiksliai 5,5 karto mažesnė. Todėl mašininio mokymo metodas, kurio mokymo duomenys yra 3D modelis, gauna mažesnės raiškos įvestį, negu metodas, kurio mokymo duomenys yra 2D paveikslėliai.

\subsection{3D objektų klasifikavimo naudojantis 2D nuotraukomis uždavinio sprendinių pavyzdžiai}

Vienas seniausių šio uždavinio sprendinių, taikantis tokią metodologiją, yra aprašytas darbe \cite{prevWparEig}. Šis sprendinys atpažįsta 3D objektus lygindamas jų vaizdus, kurie buvo suformuoti iš didelės imties 2D nuotraukų, parametrizuotoje tikriniu vektoriumi (angl. eigenspace). Šios nuotraukos buvo sugeneruotos iš 3D modelių naudojant skirtingus apžvalgos taškus ir apšvietimus. Kitas pavyzdys, kuris yra gana populiarus kompiuterinėje grafikoje, yra šviesos lauko deskriptorius (angl. light field descriptor), kuris yra aprašytas darbe \cite{prevWLightFld}. Šis sprendinys išgauna geometrinius ir Furje deskriptorius iš 3D objektų siluetų, kurie buvo sugeneruoti iš 3D modelių, naudojant skirtingus apžvalgos taškus. Darbe \cite{prevWShockGraph} aprašytas šio uždavinio sprendimas, kuris 3D objekto siluetus išskaido į dalis ir išsaugo juos į orientuotą beciklį grafą (angl. directed acyclic graph), kuris yra vadinamas šoko grafu. Kitas pavyzdys aprašytas darbe \cite{prevWSimMet}, naudoja panašumo metrikas (angl. similarity metrics), kurios yra pagrįstos kreivių palyginimu (angl. curve matching) ir sugrupuotomis panašiomis 2D nuotraukomis.

Šiuo metu, 3D objektų klasifikavimo naudojantis 2D nuotraukomis uždaviniui spręsti, optimalius rezultatus, laiko ir tikslumo atžvilgiu, pasiekęs mašininio mokymo metodu pagrįstas sprendimas yra daugiavaizdžiai konvoliuciniai dirbtiniai neuroniniai tinklai. Tai eksperimentu yra įrodyta darbe \cite{cnnExp1}, palyginant įvairių tipų konvoliucinius neuroninius tinklus su kitais sprendimo metodais. Geriausią rezultatą pasiekęs tipas yra daugiavaizdis konvoliucinis neuroninis tinklas (angl. multi-view convolutional network), kurio tikslumas tame eksperimente yra 90,1\%. Tolimesniame tyrime, kuris yra atliktas darbe \cite{cnnExp2}, daugiavaizdžiui konvoliuciniui neuroniniui tinklui yra pritaikyta kita konfigūracija ir tame darbe atlikti tyrimai pasiekė aukštesnį tikslumą - 91.3\%. Tad šio magistro baigiamojo darbo tyrimams yra naudojama konfigūracija, kuri yra taikyta darbe \cite{cnnExp2}.