
3D objektų atpažinimo iš 2D nuotraukų uždavinys - tai klasifikavimo uždavinys, kuriame pateiktos 2D nuotraukos, kuriose yra atvaizduotas 3D objektas iš atsitiktinio apžvalgos taško, turi būti priskirtas 3D modeliui, kuris yra atvaizduotas toje 2D nuotraukoje.

Klasifikavimo uždavinys - tai uždavinys, kuriame kuriamas metodas, kaip nustatyti pavyzdžio, iš tiriamos srities populiacijos, klasę. 3D objektų atpažinimo iš 2D nuotraukų uždavinio atveju, tiriama sritis yra 2D nuotraukos, kuriose yra atvaizduotas 3D objektas iš bet kurio apžvalgos taško ir klasė - 3D objektas. Taip pat, šio darbo atveju, metodas yra dirbtinio neuroninio tinklo (CapsNet arba konvoliucinio) apmokytas modelis.

Kaip jau minėta įvade, šiam uždaviniui spręsti efektyviausia yra naudoti mašininio mokymo metodą, kurio mokymo duomenys yra tik 2D nuotraukos, o 3D objektai bus tik duomenų klasės. Darbe \cite{dbnExp} atlikto tyrimo rezultatai parodo, kad pateiktas sprendimas, kuriame 3D objektų atpažinimas yra konstruojamas naudojantis tik 2D nuotraukomis, yra tikslesnis 8\% (77\% → 85\%). Šaltinyje \cite{cnnExp1} yra teigiama, kad to priežastis yra relativiai efektyvesnis 2D nuotraukų informacijos saugojimas negu 3D modelių. Todėl, kad, nors 3D modelis turi visą informaciją apie atvaizduotą 3D objektą, tačiau tam, kad panaudoti vokselinę 3D objekto reprezentaciją mašininiame mokyme, kurio mokymas su pakankamai didele duomenų imtimi užtruktų racionalų laiko tarpą, tenka ženkliai sumažinti 3D modelio rezoliuciją. Pavyzdžiui, 3D modelio, kurio rezoliucija yra $30\times30\times30$ vokseliai, įvesties dydis yra apytiksliai lygus 2D paveikslėlio, kurio rezoliucija yra $164\times164$ pikseliai. Tad šiuo atveju, 3D modelis yra apdorojamas per tiek pat laiko kaip ir 2D paveikslėlis, bet modelio rezoliucija yra apytiksliai 5.5 karto mažesnė. Todėl mašininio mokymo metodas, kurio mokymo duomenys yra 3D modelis, gauna mažesnės raiškos įvestį, negu metodas, kurio mokymo duomenys yra 2D paveikslėliai.
