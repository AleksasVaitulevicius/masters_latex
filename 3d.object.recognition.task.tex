
3D objektų atpažinimo iš 2D nuotraukų uždavinys - tai klasifikavimo uždavinys, kuriame pateiktos 2D nuotraukos, kuriose yra atvaizduotas 3D objektas iš atsitiktinio apžvalgos taško, turi būti priskirtas 3D modeliui, kuris yra atvaizduotas toje 2D nuotraukoje.

Klasifikavimo uždavinys - tai uždavinys, kuriame kuriamas metodas, kaip nustatyti pavyzdžio, iš tiriamos srities populiacijos, klasę. 3D objektų atpažinimo iš 2D nuotraukų uždavinio atveju, tiriama sritis yra 2D nuotraukos, kuriose yra atvaizduotas 3D objektas iš bet kurio apžvalgos taško ir klasė - 3D objektas. Taip pat, šio darbo atveju, metodas yra dirbtinio neuroninio tinklo (kapsulinio arba konvoliucinio) apmokytas modelis.

Kaip jau minėta įvade, šiam uždaviniui spręsti efektyviausia yra naudoti mašininio mokymo metodą, kurio mokymo duomenys yra tik 2D nuotraukos, o 3D objektai bus tik duomenų klasės. Darbe \cite{dbnExp} atlikto tyrimo rezultatai parodo, kad pateiktas sprendimas, kuriame 3D objektų atpažinimas yra konstruojamas naudojantis tik 2D nuotraukomis, yra tikslesnis 8\% (77\% → 85\%). Šaltinyje \cite{cnnExp1} yra teigiama, kad to priežastis yra reliatyviai efektyvesnis 2D nuotraukų informacijos saugojimas negu 3D modelių. Todėl, kad, nors 3D modelis turi visą informaciją apie atvaizduotą 3D objektą, tačiau tam, kad panaudoti vokselinę 3D objekto reprezentaciją mašininiame mokyme, kurio mokymas su pakankamai didele duomenų imtimi užtruktų racionalų laiko tarpą, tenka ženkliai sumažinti 3D modelio rezoliuciją. Pavyzdžiui, 3D modelio, kurio rezoliucija yra $30\times30\times30$ vokseliai, įvesties dydis yra apytiksliai lygus 2D paveikslėlio, kurio rezoliucija yra $164\times164$ pikseliai. Tad šiuo atveju, 3D modelis yra apdorojamas per tiek pat laiko kaip ir 2D paveikslėlis, bet modelio rezoliucija yra apytiksliai 5.5 karto mažesnė. Todėl mašininio mokymo metodas, kurio mokymo duomenys yra 3D modelis, gauna mažesnės raiškos įvestį, negu metodas, kurio mokymo duomenys yra 2D paveikslėliai.

Vienas seniausių šio uždavinio sprendinių, taikantis tokią metodologiją, yra aprašytas darbe \cite{prevWparEig}. Šis sprendinys atpažįsta 3D objektus lygindamas jų vaizdus, kurie buvo suformuoti iš didelės imties 2D nuotraukų, parametrizuotoje 
eigenerdvėje (angl. eigenspace) % Kaip lietuviškai?
. Šios nuotraukos buvo sugeneruotos iš 3D modelių naudojant skirtingus apžvalgos taškus ir apšvietimus. 
Kitas pavyzdys, kuris yra gana populiarus kompiuterinėje grafikoje, yra 
šviesos lauko deskriptorius (angl. light field descriptor)% light field descriptor
, kuris yra aprašytas darbe \cite{prevWLightFld}. Šis sprendinys išgauna geometrinius ir 
Fourier'io % kaip daryti su angliškais vardais?
deskriptorius iš 3D objektų siluetų, kurie buvo sugeneruoti iš 3D modelių, naudojant skirtingus apžvalgos taškus. 
Darbe \cite{prevWShockGraph} aprašytas šio uždavinio sprendimas, kuris 3D objekto siluetus išskaido į dalis ir išsaugo juos į 
orientuotą beciklį grafą (angl. directed acyclic graph) % directed acyclic graph 
, šoko grafą. % shock graph
Kitas pavyzdys aprašytas darbe \cite{prevWSimMet}, naudoja panašumo metrikas (angl. similarity metrics) % similarity metrics
, kurios yra pagrįstos kreivių palyginimu (angl. curve matching)% curve matching
ir sugrupuotomis panašiomis 2D nuotraukomis.

Šiuo metu, 3D objektų atpažinimo iš 2D nuotraukų uždaviniui spręsti, optimaliausius  rezultatus, laiko ir tikslumo atžvilgiu, pasiekęs mašininio mokymo metodu pagrįstas sprendimas yra konvoliuciniai dirbtiniai neuroniniai tinklai. Tai eksperimentu buvo įrodyta darbe \cite{cnnExp1}. Šiame eksperimente buvo palyginti įvairios konvoliucinių neuroninių tinklų tipai sprendžiant šį uždavinį ir geriausią rezultatą pasiekęs tipas buvo
daugiavaizdis (angl. multi-view convolutional network)% multi-view
konvoliucinis neuroninis tinklas, kurio tikslumas buvo 90.1\%.