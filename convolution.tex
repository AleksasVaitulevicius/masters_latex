\subsubsection{Konvoliucija}

Konvoliuciniai neuroniniai tinklai yra vieni populiariausių giliųjų neuroninių tinklų tipas. Pirma karta sėkmingai įgyvendintas konvoliucinis neuroninis tinklas yra aprašytas darbe \cite{cnn}. Šis tinklas yra skirtas ranka rašytiems pašto kodams atpažinti. Konvoliucinis neuroninis tinklas - tai gilusis neuroninis tinklas, kurio bent viename sluoksnyje yra naudojama konvoliucijos operacija, dar vadinama sąsuka. Konvoliucija - tai matematinė operacija, kurios operandai yra dvi funkcijos $f$ ir $g$, ir kurios rezultatas yra funkcija, kuri apibūdina kaip viena funkcija keičia kitą. Ši operacija yra žymima $f * g$ ir ji yra apibrėžiama kaip integralinės transformacijos rūšis pavaizduota formulėje \ref{eqn:convolution}, kur a ir b nurodo funkcijų  $f$ ir $g$ apibrėžimo sritį.

\begin{equation}
\label{eqn:convolution}
	(f * g)(t) = \int_{a}^{b} f(\tau)g(t - \tau) d\tau
\end{equation}

Konvoliucijos algebros savybės yra komutatyvumas ($f * g = g * f$), asociatyvumas ($f * (g * h) = (f * g) * h$), distributyvumas ($f * (g + h) = (f * g) + (f * h)$), vienetinis elementas $f * \delta = \delta * f = f$ ir daugybos su skaliaru asociatyvumas ($a(f * g) = (af) * g = f * (ag)$, kur $a \in \R$).

Dažniausiai konvoliuciniuose neuroniniuose tinkluose yra vykdoma konvoliucija diskrečioms funkcijoms. Konvoliucija, kurios operandai yra diskrečios funkcijos yra vadinama diskreti konvoliucija ir ji yra apibrėžiama kaip formulė \ref{eqn:discrete_convolution}, kur a ir b nurodo funkcijų  $f$ ir $g$ apibrėžimo sritį.

\begin{equation}
\label{eqn:discrete_convolution}
	(f * g)(t) = \sum_{\tau = a}^{b} f(\tau)g(t - \tau)
\end{equation}

Šio darbo tyrimuose yra naudojamos 2D nuotraukos, kurios yra saugomos kaip dviejų dimensijų vaizdai, vadinamos matricomis. Diskreti konvoliucija matricoms yra atliekama naudojantis formulę \ref{eqn:matrix_convolution}.

\begin{equation}
\label{eqn:matrix_convolution}
	(I * K)(i, j) = \sum_{m} \sum_{n} I(m, n) K(i - m, j - n)
\end{equation}

Konvoliuciniuose neuroniniuose tinkluose matrica $I$ yra vadinama įvestimi, o matrica $K$ - branduoliu arba filtru. Konvoliucija yra komutatyvi, todėl formulė \ref{eqn:matrix_convolution} gali būti išreikšta kaip \ref{eqn:cnn_convolution}.

\begin{equation}
\label{eqn:cnn_convolution}
	(K * I)(i, j) = \sum_{m} \sum_{n} I(i - m, j - n) K(m, n)
\end{equation}

Dažniausiai ši išraiška yra naudojama konvoliuciniuose neuroniniuose tinkluose. Branduolys $K$ dažniausiai yra žymiai mažesnio dydžio nei įvesties matrica $I$ išretinta matrica (angl. sparse matrix). Išretinta matrica yra matrica, kurios didžioji dalis elementų yra lygūs 0.

% cont 19