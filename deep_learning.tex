\subsubsection{Gilieji neuroniniai tinklai}

Kaip jau minėta praeitame poskyryje tiek konvoliuciniai, tiek kapsuliniai neuroniniai tinklai yra daugiasluoksnio perceptrono plėtiniai. Abu šie plėtiniai priklauso daugiasluoksnio perceptrono plėtinių klasei, giliesiems neuroniniams tinklams (angl. deep neural networks). Pirmasis giliojo dirbtinio neuroninio tinklo aprašymas yra pateiktas darbe \cite{deepNN}. Gilusis neuroninis tinklas -- tai daugiasluoksnis perceptronas, turintis daugiau nei vieną paslėptąjį sluoksnį. Šie neuroniniai tinklai dažniausiai būna žymiai sudėtingesni nei paprasti daugiasluoksniai perceptronai. Tad jų apmokymas trunka ilgiau, bet jų tikslumas yra ženkliai didesnis nei paprastų daugiasluoksnių perceptronų.

Šių neuroninių tinklų apmokymuose gradientinio nusileidimo strategija grįstas algoritmas, klaidos sklidimo atgal, gali būti pakeistas algoritmu, grįstu stochastinio gradientinio nusileidimo strategija. Pagrindinis skirtumas tarp algoritmų grįstų gradientinio nusileidimo strategija ir stochastinio gradientinio nusileidimo strategija yra funkcija naudojama svorių keitimui. Gradientinio nusileidimo strategija naudoja funkciją (\ref{eqn:w_change}), o stochastinio gradientinio nusileidimo strategija grįsti algoritmai naudoja funkciją (\ref{eqn:sgd}), kur gradientas $\Delta w_{ki}$, yra \textit{i}-tojo įėjimo vektoriaus paklaidos išvestinė svoriams $\Delta w_{ki} = \dfrac{\partial e_i(w)}{\partial w}$.

\begin{equation}
\label{eqn:sgd}
	w_{ki}(t + 1) = w_{ki}(t) - \eta \Delta w_{ki}
\end{equation}

Labai dažnai giliųjų neuronų apmokymuose stochastinio gradientinio nusileidimo strategija yra praplečiama įtraukiant inercijos (angl. momentum) konstantą (angl. momentum constant). Tokiu atveju svoriai keičiami pagal funkciją (\ref{eqn:sgdm}), kur $\Delta w$ yra svorių pokytis praeitoje iteracijoje, o $\alpha$ - inercijos konstanta, kurios reikšmė yra teigiama intervale (0, 1).

\begin{equation}
\label{eqn:sgdm}
	w_{ki}(t + 1) = w_{ki}(t) - \eta \Delta w_{ki} + \alpha \Delta w
\end{equation}

Taip pat labai dažnai šių neuroninių tinklų vienoje iteracijoje yra naudojamas ne vienas įėjimo vektorius, bet apmokymo duomenų poaibis, vadinamas duomenų rinkiniu (angl. batch). Duomenų rinkiniai sudaromi parenkant iš apmokymo duomenų nurodytą skaičių vektorių, kurie nebuvo naudojami praeitose iteracijose $n$-tąjį kartą. Jei tokių vektorių nėra arba yra nepakankamai, kad sudaryti naują duomenų rinkinį su nurodytu dydžiu, tai trūkstami vektoriai yra parenkami tokie, kurie nebuvo naudoti praeitose iteracijose $(n + 1)$-tąjį kartą. $n$ pradedamas skaičiuoti nuo vieneto. Šiame magistro baigiamajame darbe apmokymai yra vykdomi naudojant duomenų rinkinius.

Jei dirbtinio neuroninio tinklo apmokyme yra naudojami duomenų rinkiniai, kurių dydis yra didesnis negu 1, tai tada yra naudojama modifikuota stochastinio gradientinio nusileidimo strategija, kurioje gradientas yra apskaičiuojamas naudojantis ne visus apmokymo duomenis, bet tik iteracijos duomenų rinkinio. Stochastinio gradientinio nusileidimo strategija ir jos modifikacijos yra vadinamos optimizavimo algoritmais, nes šių algoritmų paskirtis yra optimizuoti nuostolių funkciją (angl. loss function, cost function arba objective function). Nuo pasirinkto algoritmo priklauso apmokymo trukmė. Šių algoritmų pavyzdžiai yra AdaGrad, prisitaikančio gradiento (angl. adaptive gradient); RMSProp, šaknies vidurkio kvadrato išskleidimo (angl. root mean square propagation); AdaDelta ir Adam, prisitaikančios inercijos apskaičiavimo (angl. adaptive moment estimation), algoritmai. Šiame magistro baigiamajame darbe yra naudojamas vienas populiariausių modifikacijų - Adam algoritmas.

Adam algoritmas yra aprašytas darbe \cite{adam}. Šis algoritmas apjungia algoritmų RMSProp ir AdaGrad privalumus bei darbe \cite{adam} yra pateikiamas empirinis įrodymas, kad Adam algoritmo veikimas yra sparčiausias iš tame darbe nagrinėtų algoritmų. Pirmas Adam algoritmo žingsnis yra apskaičiuoti gradientą \textit{t}-ajai iteracijai $g_t$, naudojantis formulę (\ref{eqn:grad}), kur $\Delta l(g_{t-1})$ yra funkcijos $l$ išvestinė svoriams ir $l$ yra nuostolių funkcija.

\begin{equation}
\label{eqn:grad}
	g_t = \Delta l(g_{t-1})
\end{equation}

Antras šio algoritmo žingsnis yra apskaičiuoti $t$-ajai iteracijai eksponentiškai mažėjantį praeitų kvadratu pakeltų gradientų vidurkį $v_t$, naudojantis formule (\ref{eqn:v_t}), ir eksponentiškai mažėjantį praeitų gradientų vidurkį, naudojantis formule (\ref{eqn:m_t}), kur $\beta_1$ ir $\beta_2$ yra nurodomos konstantos.

\begin{equation}
\label{eqn:m_t}
	m_t = \beta_1 m_{t-1} + (1 - \beta_1)g_t
\end{equation}

\begin{equation}
\label{eqn:v_t}
	v_t = \beta_2 v_{t-1} + (1 - \beta_2)g_t^2
\end{equation}

Pirmojoje iteracijoje $t = 1$ naudojami $m_0 = 0$ ir $v_0 = 0$. Darbe \cite{adam} pateikiama, kad parametrai $v_t$ ir $m_t$ yra linkę konverguoti į 0. Tad tolimesnis Adam algoritmo žingsnis apskaičiuoja koreguotas šių parametrų reikšmes $\hat{v_t}$ ir $\hat{m_t}$ naudodamas formules (\ref{eqn:corr_v_t}) ir (\ref{eqn:corr_m_t}).

\begin{equation}
\label{eqn:corr_m_t}
	\hat{m_t} = \dfrac{m_t}{1 - \beta_1^t}
\end{equation}

\begin{equation}
\label{eqn:corr_v_t}
	\hat{v_t} = \dfrac{v_t}{1 - \beta_2^t}
\end{equation}

Paskutinis šio algoritmo žingsnis yra svorių korekcija naudojantis formule (\ref{eqn:adam}), kur $w_t$ yra $t$-osios iteracijos svorių aibė, $\eta$ nurodyta konstanta, vadinama mokymosi žingsnio dydžiu (angl. learning step size), ir $\epsilon$ yra labai mažas skaičius, naudojamas tam kad išvengti dalybos iš 0.

\begin{equation}
\label{eqn:adam}
	w_t = w_{t-1} - \eta \dfrac{\hat{m_t}}{\sqrt{\hat{v_t}} + \epsilon}
\end{equation}

Darbe \cite{adam} atlikti tyrimai parodė, kad Adam algoritmas veikia sparčiausiai su parametrų reikšmėmis $\eta = 0,001$, $\beta_1 = 0,9$, $\beta_2 = 0,999$ ir $\epsilon = 10^{-8}$. Tad šiame magistro baigiamajame darbe yra naudojamos būtent šitos parametrų reikšmės.

Naudojamos nuostolių funkcijos optimizavimo algoritmuose priklauso nuo giliojo neuroninio tinklo tipo ir išėjimo sluoksnio aktyvacijos funkcijos. Pavyzdžiui, konvoliucinis neuroninis tinklas su išėjimo sluoksnio aktyvacijos funkcija SoftMax naudoja kryžminės entropijos (angl. cross-entropy) nuostolių funkciją. Giliojo neuroninio tinklo apmokymas yra svorių radimas, su kuriais nuostolių funkcija įgyja kiek galima mažiausią reikšmę.
