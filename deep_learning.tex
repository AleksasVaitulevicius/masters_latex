\subsubsection{Gilieji neuroniniai tinklai}

Kaip jau minėta praeitame poskyryje tiek konvoliuciniai tiek kapsuliniai neuroniniai tinklai yra daugiasluoksnio perceptrono plėtiniai. Abu šie plėtiniai priklauso daugiasluoksnio perceptrono plėtinių klasei, giliesiems neuroniniams tinklams (angl. deep neural networks). Pirmasis giliojo dirbtinio neuroninio tinklo aprašymas yra pateiktas darbe \cite{deepNN}. Gilusis neuroninis tinklas tai daugiasluoksnis perceptronas, turintis daugiau nei vieną paslėptąjį sluoksnį.

Šie neuroniniai tinklai dažniausiai būna žymiai sudėtingesni nei paprasti daugiasluoksniai perceptronai. Todėl jų skaičiavimai reikalauja didesnių resursų. Tad 

% Apmokymas vyksta epochomis, batcho dydis, duomenų atskyrimas į training, testing ir validation, adam optimizer