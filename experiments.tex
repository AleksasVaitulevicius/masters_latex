
Šiame darbe yra palyginami 3 neuroninių tinklų architektūros: daugiavaizdis neuroninis tinklas, aprašytas poskyryje Daugiavaizdis konvoliucinis neuroninis tinklas, kapsulinis neuroninis tinklas, aprašytas poskyryje Tiriamo kapsulinio neuroninio tinklo architektūra, ir daugiavaizdis kapsulinis neuroninis tinklas, aprašytas poskyryje Tiriamo daugiavaizdžio kapsulinio neuroninio tinklo architektūra. Kiekvienas tiriamas dirbtinis neuroninis tinklas yra apmokomas naudojantis visais duomenimis, aprašytais poskyryje Tyrimams naudoti duomenys. Šie duomenys apmokymo metu yra padalinami į duomenų rinkinius, kurių dydžiai yra 96. Daugiavaizdžio konvoliucinio ir kapsulinio neuroninių tinklų apmokymų antram etapui duomenų rinkiniai buvo sudaryti iš nuotraukų grupių, kuriose yra visos konkretaus 3D objekto modelio nuotraukos. Kiekvienas dirbtinis neuroninis tinklas yra apmokomas per 10 epochų. Daugiavaizdžio konvoliucinio ir kapsulinio neuroninių tinklų abu apmokymo etapai yra apmokomi po 5 epochas.

Po kiekvienos epochos yra renkamos tikslumo metrikos: tikslumas klasifikuojant apmokymo duomenis, tikslumas klasifikuojant testavimo duomenis, nuostolių funkcijos rezultatas klasifikuojant apmokymo duomenis ir nuostolių funkcijos rezultatas klasifikuojant testavimo duomenis. Tikslumas yra teisingai suklasifikuotų įrašų dalis klasifikuotų duomenų aibėje. Dauigavaizdžiame kapsuliniame neuroniniame tinkle ir kapsuliniame neuroniniame tinkle naudojama nuostolių funkcija yra margin nuostolių funkcija, o daugiavaizdžiame konvoliuciniame neuroniniame tinkle - kryžminės entropijos nuostolių funkcija.
