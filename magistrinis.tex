\documentclass{VUMIFInfMagistrinis}
\usepackage{algorithmicx}
\usepackage{algorithm}
\usepackage{algpseudocode}
\usepackage{amsfonts}
\usepackage{amsmath}
\usepackage{bm}
\usepackage{color}
\usepackage{listings}
\usepackage{graphicx}
\newcommand{\R}{\mathbb{R}}

% Titulinio aprašas
\university{Vilniaus universitetas}
\faculty{Matematikos ir informatikos fakultetas}
\department{Informatikos katedra}
\papertype{Magistro baigiamasis darbas}
\title{3D objektų klasifikavimas naudojantis 2D nuotraukomis}
\titleineng{3D object classification using 2D images}
\status{2 kurso 1 grupės studentas}
\author{Aleksas Vaitulevičius}
\supervisor{prof. dr. Olga Kurasova}
\reviewer{dr. Linas Petkevičius}
\date{Vilnius – \the\year}

% Nustatymai
% \setmainfont{Palemonas}   % Pakeisti teksto šriftą į Palemonas (turi būti įdiegtas sistemoje)
\bibliography{bibliografija}

\begin{document}
\maketitle

\sectionnonumnocontent{Santrauka}
3D objektų klasifikavimas naudojantis 2D nuotraukomis yra naudojamas srityse, kuriose turimi 3D objektų klasės turi būti atpažintos iš visų galimų 2D nuotraukų, turint tik šių nuotraukų poaibį. Šiame magistro baigiamame darbe yra palyginamas kapsulinio neuroninio tinklo modifikacijos su dabartiniu geriausiu šio uždavinio sprendiniu, daugiavaizdžiu konvoliuciniu neuroniniu tinklu. Tiriamos kapsulinio neuroninio tinklo modifikacijos yra daugiavaizdžiai kapsuliniai neuroniniai tinklai, iš kurių vienas naudoja vaizdų sujungimo sluoksnį ir kitas - vaizdų kapsulinį sluoksnį. Atlikti eksperimentai leidžia teigti, kad dabartinis geriausias sprendimas yra pranašesnis už išbandytas kapsulinio neuroninio tinklo modifikacijas.
\raktiniaizodziai{Klasifikavimo uždavinys, 3D objektai, dirbtiniai neuroniniai tinklai, kapsuliniai neuroniniai tinklai, tiesioginio sklidimo neuroniniai tinklai}

% Glaustai aprašomas darbo turinys, pristatoma nagrinėta problema ir padarytos
% išvados. Santraukos apimtis ne didesnė nei 0,5 puslapio. Santraukų gale
% nurodomi darbo raktiniai žodžiai. 
% Nurodomi iki 5 svarbiausių temos raktinių žodžių (terminų).
% Vienas terminas gali susidėti iš kelių žodžių.

\sectionnonumnocontent{Summary}
3D object classification by using 2D images is used in fields, where given 3D object classes have to be recognized in any possible 2D image, by using only a subset of those images. In this master's thesis neural capsule network modifications are compared with current state of the art solution of this problem, multi-view convoliutional network. Tested neural capsule network modifications are multi-view neural capsule networks, of which one is using view pooling layer and the other one - view capsule layer. Conducted experiments allowed to draw a conclusion that current state of the art solution is better than tested capsule network modifications.
\keywords{classification, 3D objects, artificial neural networks, CapsNet neural networks, convolutional neural networks}

% Santrauka anglų kalba.

\tableofcontents

\sectionnonum{Santrumpos ir Terminai}

\begin{enumerate}
	\item AdaDelta - Prisitaikančios deltos optimizavimo algoritmas (angl. adaptive delta)
	\item AdaGrad - Prisitaikančio gradiento optimizavimo algoritmas (angl. adaptive gradient)
	\item Adam - Inercijos apskaičiavimo optimizavimo algoritmas (angl. adaptive moment estimation)
	\item Apibrėžiantis stačiakampis (angl. bounding box)
	\item Apmokymas be mokytoju (angl. unsupervised learning)
	\item Apmokymas su mokytoju (angl. supervised learning)
	\item Blokinė diagrama (angl. box plot)
	\item Daugiasluoksnis perceptronas (angl. multilayer perceptron)
	\item Daugiavaizdžiai konvoliuciniai neuroniniai tinklai (angl. multi-view convolutional neural networks)
	\item Dinaminis maršrutizavimas tarp kapsulių (angl. dynamic routing between capsules)
	\item Dirbtiniai neuroniniai tinklai (angl. artificial neural networks)
	\item Duomenų rinkinys (angl. batch)
	\item F-dalių kryžminis validavimas (angl. f-folds cross validation)
	\item F1 įvertis (angl. F1 score)
	\item Svertinis vidurkis (angl. weighted average)
	\item Gabalais tiesinė funkcija (angl. piecewise linear function)
	\item Gauso funkcija (angl. Gaussian function)
	\item Gilieji neuroniniai tinklai (angl. deep neural networks)
	\item Gylio nuotrauka (angl. depth image)
	\item Grįžtamojo ryšio neuroniniai tinklai (angl. feedback neural networks)
	\item Inercija (angl. momentum)
	\item Inercijos konstanta (angl. momentum constant)
	\item Išretinta matrica (angl. sparse matrix)
	\item Išretinta sąveika (angl. sparsity)
	\item Ištaisymo sluoksnis (angl. rectification layer)
	\item Klaidingų teigiamų (angl. false positives)
	\item Klaidingų neigiamų (angl. false negatives)
	\item Klaidos sklidimo atgal algoritmas (angl. back-propagation learning algorithm)
	\item Konvoliucinis gilaus pasitikėjimo neuroninis tinklas (angl. convolutional deep belief neural network)
	\item Kraštutinės vertės (angl. outliers)
	\item Kreivių palyginimas (angl. curve matching)
	\item Kryžminė entropija (angl. cross-entropy)
	\item Kvartilis (angl. quartile)
	\item leaky ReLU - nesandari ištaisymo tiesinė funkcija (angl. leaky rectified linear function)
	\item Maksimalus sujungimas (angl. max pooling)
	\item Mokymo greitis (angl. learning rate function)
	\item Mokymosi žingsnio dydžiu (angl. learning step size)
	\item Nepersidengiančiu sujungimo sluoksnis (angl. non-overlapping pooling layer)
	\item Netiesiškumo sluoksnis (angl. non-linearity layer)
	\item noisy ReLU - Ištaisymo tiesinė funkcija su triukšmu (angl. noisy rectified linear function)
	\item Nulinis svoris/slenkstis (angl. bias)
	\item Nuostolių funkcija (angl. loss function/cost function/objective function)
	\item Orientuotas beciklis grafas (angl. directed acyclic graph)
	\item Panašumo metrikos (angl. similarity metrics)
	\item Persidengiantis sujungimo sluoksnis (angl. overlapping pooling layer)
	\item Persimokymas (angl. overfitting)
	\item Pilnai sujungtas sluoksnis (angl. fully connected layer)
	\item Pirminių kapsulių sluoksnis (angl. primary capsules)
	\item Postūmio matrica (angl. bias matrix)
	\item Požymių žemėlapiai (angl. feature map)
	\item RMSProp - Šaknies vidurkio kvadrato išskleidimo optimizavimo algoritmas (angl. root mean square propagation)
	\item ReLu - Ištaisymo tiesinė funkcija (angl. rectified linear function)
	\item Sigmoidinė funkcija (angl. sigmoid function)
	\item Skiriamasis paviršius (angl. decision boundary)
	\item Slenkstinė funkcija (angl. unit step function)
	\item Sujungimo sluoksnis (angl. pooling layer)
	\item Šviesos lauko deskriptorius (angl. light field descriptor)
	\item Teisingi teigiami (angl. true positives)
	\item Tiesinė funkcija (angl. linear function)
	\item Tiesioginio sklidimo neuroniniai tinklai (angl. feedfoward neural networks)
	\item Tikrinis vektorius (angl. eigenspace)
	\item Vaizdų sujungimo sluoksnis (angl. view pooling layer)
	\item Vidutinis sujungimas (angl. average pooling)
	\item Vidutinė kvadratinė paklaida (angl. mean square error)
\end{enumerate}

\sectionnonum{Įvadas}

Vienas iš fundamentalių kompiuterinės regos uždavinių yra informacijos apie 3 dimensijų (3D) pasaulį išgavimas naudojant 2 dimensijų (2D) nuotraukas. Šio uždavinio tikslas yra atpažinti konkrečius 3D objektus, naudojant jų, 2D nuotraukas, padarytas iš skirtingų kampų. Šiam tikslui pasiekti, yra konstruojami objektų atpažinimo algoritmai, kurie klasifikuoja 2D nuotraukas į klases, kurios atstovauja vieną iš 3D objektų modelių.

3D objektų atpažinimas iš 2D nuotraukų yra naudojamas srityse, kuriose turimi 3D objektai turi būti atpažinti iš visų galimų 2D nuotraukų, turint tik poaibį šių nuotraukų. Keli šių sričių pavyzdžiai yra automatinė objektų inspekcija - turint algoritmą, atpažįstantį konkretų objektą, kuris turi tik jam būdinga 3D formą, galima nustatyti nuotraukas, kuriose yra tas objektas, navigacijoje - turint algoritmą, atpažįstantį konkrečius objektus, esančius skirtingose vietovėse, ir tų vietovių koordinates, galima nustatyti kurioje vietovėje buvo padaryta nuotrauka. Deja, laiko ir duomenų kaštai yra per dideli, kad pasiekti absoliutų tikslumą sprendžiant šį uždavinį. Todėl taikomi metodai yra euristiniai. Dėl to renkantis metodą, spręsti 3D objektų atpažinimo iš 2D nuotraukų uždaviniui, reikia atsižvelgti į laiko kaštus ir kaip tiksliai tuo metodu pagrįstas algoritmas klasifikuoja 2D nuotraukas, spręsdamas šį uždavinį. Šiame darbe bus atliekami tyrimai, skirti nustatyti metodą, sprendžiantį 3D objektų atpažinimo iš 2D nuotraukų uždavinį ir kuris pasiekia didžiausią tikslumą ir reikalauja mažiausiai laiko mokymui.

Gana dažnai naudojamas metodas šiam uždaviniui spręsti, yra dirbtiniai gilieji neuroniniai tinklai. 3D objektų atpažinimo iš 2D nuotraukų uždavinyje naudojami duomenys yra 2D nuotraukos, yra nestruktūrizuoti, jiems sudėtinga vykdyti požymių išgavimą. Todėl daugelis kitų sprendimų nėra tokie patrauklūs kaip dirbtiniai gilieji neuroniniai tinklai, dėl savo sugebėjimo efektyviai vykdyti automatinį požymių išgavimą iš nestruktūrizuotų duomenų. Tačiau, norint pasiekti aukštą klasifikavimo tikslumą, naudojant šį metodą, yra reikalingas didelis kiekis duomenų. Konkrečiai šiam uždaviniui reikia didelio kiekio 3D modelių. Laimei, šiuo metu egzistuoja viešai prieinamos didelės 3D repozitorijos. Tokios kaip 3D Warehouse, TurboSquid, ir Shapeways. Dėl to šiuo metu daugelis senesnių architektūrų jau yra išbandytos sprendžiant 3D objektų atpažinimo iš 2D nuotraukų uždavinį. Pavyzdžiui viena iš architektūrų, kuri buvo išbandyta, yra konvoliucinio gilaus pasitikėjimo neuroninio tinklo (angl. convolutional deep belief neural network) architektūra. Šiai architektūrai atlikti tyrimai yra aprašyti darbe \cite{dbnExp}. Tačiau 2017 metais buvo aprašyta nauja architektūra, kapsuliniai neuroniniai tinklai. Tyrimai, parodė, kad ji yra pranašesnė tikslumo atžvilgiu už senesnes architektūras, sprendžiant uždavinius panašius į 3D objektų atpažinimo iš 2D nuotraukų uždavinį.

Šiuo metu šiam uždaviniui spręsti, optimaliausius  rezultatus, laiko ir tikslumo atžvilgiu, pasiekusi dirbtinio neuroninio tinklo architektūra yra konvoliuciniai neuroniniai tinklai (angl. convolutional neural networks). Tyrimai, kuriuose ši architektūra buvo išbandyta, yra aprašyta darbuose \cite{cnnExp1, cnnExp2}. Darbe \cite{dbnExp} atlikto tyrimo rezultatai parodo, kad pateiktas sprendimas, kuriame 3D objektų atpažinimas yra konstruojamas naudojantis tik 2D nuotraukomis, yra tikslesnis 8 \%. Algoritmas, naudojantis 3D modelius, pasiekė 77 \% tikslumą, o algoritmas, naudojantis tik 2D nuotraukas, pasiekė 85 \% tikslumą. Todėl šiame darbe bus daromi eksperimentai su dirbtinio neuroninio tinklo architektūrų įgyvendinimais, kurie yra pagrįsti būtent šiuo metodu. Tad šiam darbui, vienas iš pasirinktų metodų yra konvoliucinis neuroninis tinklas, kurio įgyvendinimas ir tyrimai yra aprašyti darbe \cite{cnnExp1}. Mat šio darbo įgyvendinimas naudoja tik 2D nuotraukas, konstruojant 3D objektų atpažinimo algoritmą.

Kita tiriama dirbtinio neuroninio tinklo architektūra yra kapsuliniai neuroniniai tinklai. Lyginant su konvoliuciniais neuroniniais tinklais, tai gana nauja architektūra. Aprašytos 2017 metais \cite{capsNet} darbe Kapsulinių neuroninių tinklų architektūros veikimo principas tiksliau imituoja žmogaus rega, remiantis faktu, kad žmogaus rega ignoruoja nereikšmingas vaizdo detales, naudodama tik seką fokusuotų taškų, taip apdorodama tik dalį vaizdinės informacijos su labai aukšta rezoliucija. \cite{capsNet} darbe atliktas tyrimas parodo, kad ši architektūra atlieka ranka rašytų skaičių klasifikavimo užduotį tiksliau nei konvoliuciniai neuroniniai tinklai. Kitas tyrimas, kuris yra atliktas darbe \cite{capsCNN} su 4 duomenų rinkiniais, kuriuose yra veidai, kelio ženklai ir kasdieniai objektai, parodo, kad dabartiniai kapsuliniai neuroniniai tinklai ne visada yra pranašesni už konvoliucinius neuroninius tinklus. Parinkus geresnius parametrus ir modifikacijas (sluoksnių skaičių, neuronų skaičių kiekviename sluoksnyje, aktyvacijos funkcijos), konvoliucinis neuroninis tinklas dar vis būna pranašesnis už kapsulinį neuroninį tinklą. Tačiau taip pat darbe \cite{capsCNN} yra teigiama, kad kapsuliniai neuroniniai tinklai dar nėra pasiekę pilno savo potencialo ir tolimesni tyrimai turi būti atlikti.

Tad šio darbo \textbf{tikslas} yra įrodyti arba paneigti keliamą \textbf{hipotezę}:

\textit{Kapsuliniai neuroniniai tinklai sprendžia 3D objektų atpažinimo iš 2D nuotraukų uždavinį efektyviau nei konvoliuciniai neuroniniai tinklai remiantis apmokymo laiko ir tikslumo kriterijais}.

Tikimasi, kad, sprendžiant 3D objektų atpažinimo iš 2D nuotraukų uždavinį, kapsulinio neuroninio tinklo mokymas truks trumpiau nei konvoliucinio neuroninio tinklo. Taip pat, kad apmokytas kapsulinis neuroninis tinklas vykdys klasifikavimą tiksliau nei konvoliucinis neuroninis tinklas.

Siekiant patikrinti iškeltą hipotezę reikia atlikti šiuos uždavinius:

\begin{enumerate}
	\item Išanalizuoti ir nustatyti dabartinį efektyviausią 3D objektų atpažinimo iš 2D nuotraukų uždavinio sprendinį.
	\item Išanalizuoti kapsulinių neuroninių tinklų veikimą.
	\item Surasti duomenis, skirtus spręsti 3D objektų atpažinimo iš 2D nuotraukų uždaviniui.
	\item Eksperimentiškai nustatyti efektyviausius parametrus ir modifikacijas, skirtus spręsti 3D objektų atpažinimo iš 2D nuotraukų uždaviniui, kapsulinio neuroninio tinklo implementacijai, remiantis apmokymo laiko ir tikslumo kriterijais.
	\item Atlikti eksperimentus, skirtus palyginti kapsulino neuronino tinklo ir konvoliucinio neuronino tinklo tikslumą ir apmokymo laiką, sprendžiant 3D objektų atpažinimo iš 2D nuotraukų uždavinį.
\end{enumerate}

Šiame darbe planuojami rezultatai:

\begin{enumerate}
	\item Nustatyta, kad šiuo metu efektyviausias 3D objektų atpažinimo iš 2D nuotraukų uždavinio sprendinys yra konvoliuciniai neuroniniai tinklai, lyginant eksperimentų, aprašytų skirtinguose literatūros šaltiniuose, rezultatus. Šiuose šaltiniuose buvo surasta konvoliucinio neuroninio tinklo implementaciją ir duomenys skirti apmokymui ir testavimui.
	\item Išanalizuotas kapsulinių neuroninių tinklų veikimas, surasta jo implementacija.
	\item Eksperimentiškai nustatyta efektyviausia konfigūracija kapsuliniui neuroniniui tinklui sprendžiant 3D objektų atpažinimo iš 2D nuotraukų uždaviniui, naudojantis apmokymo laiko ir tikslumo kriterijais.
	\item Eksperimentiškai palygintas kapsulinio neuroninio tinklo tikslumas ir apmokymo laikas su konvoliuciniu neuroniniu tinklu, naudojantis apmokymo laiko ir tikslumo kriterijais.
\end{enumerate}

Darbas remiasi šiomis prielaidomis:

\begin{enumerate}
	\item Kiekvienam 2D paveikslėliui yra priskirta jam jį atitinkantis 3D objektas.
	\item Kiekvienas 3D objektas turi bent po vieną jį atitinkantį 2D paveikslėlį.
\end{enumerate}

Šio darbo turinys yra sudarytas iš 4 skyrių. Pirmame skyriuje yra pateikiama literatūros analizė. Jame yra pateiktas 3D objektų atpažinimo iš 2D nuotraukų uždavinio aprašymas, egzistuojančių sprendimų apžvalga, bendrieji neuroninių tinklų principai, konvoliucinio neuroninio tinklo veikimo aprašymas ir kapsulinio neuroninio tinklo aprašymas.
Tada antrame skyriuje yra pateikiami šiame darbe bandomų kapsulinių neuroninių tinklų modifikacijos ir parinkti parametrai.
Trečiame skyriuje yra aprašomi tyrimams naudoti duomenys.
Taip pat šiame skyriuje yra aprašomi tyrimai, skirti nustatyti kapsulinių neuroninių tinklų modifikaciją ir parametrus, kurie pasiekia didžiausią tikslumą ir reikalauja mažiausiai laiko apmokymui, sprendžiant 3D objektų atpažinimo iš 2D nuotraukų uždavinį.
Galiausiai šiame skyriuje yra aprašomi tyrimai, skirti palyginti kapsulinių neuroninių tinklų ir konvoliucinių neuroninių tinklų tikslumą ir apmokymo laiką, sprendžiant 3D objektų atpažinimo iš 2D nuotraukų uždavinį.
Paskutiniame skyriuje yra pateikiami rezultatai ir išvados.


\section{Literatūros analizė}

\subsection{3D objektų klasifikavimo naudojantis 2D nuotraukomis uždavinys}

3D objektų atpažinimo iš 2D nuotraukų uždavinys - tai klasifikavimo uždavinys, kuriame pateiktos 2D nuotraukos, kuriose yra atvaizduotas 3D objektas iš atsitiktinio apžvalgos taško, turi būti priskirtas 3D modeliui, kuris yra atvaizduotas toje 2D nuotraukoje.

Klasifikavimo uždavinys - tai uždavinys, kuriame kuriamas metodas, kaip nustatyti pavyzdžio, iš tiriamos srities populiacijos, klasę. 3D objektų atpažinimo iš 2D nuotraukų uždavinio atveju, tiriama sritis yra 2D nuotraukos, kuriose yra atvaizduotas 3D objektas iš bet kurio apžvalgos taško ir klasė - 3D objektas. Taip pat, šio darbo atveju, metodas yra dirbtinio neuroninio tinklo (kapsulinio arba konvoliucinio) apmokytas modelis.

Kaip jau minėta įvade, šiam uždaviniui spręsti efektyviausia yra naudoti mašininio mokymo metodą, kurio mokymo duomenys yra tik 2D nuotraukos, o 3D objektai bus tik duomenų klasės. Darbe \cite{dbnExp} atlikto tyrimo rezultatai parodo, kad pateiktas sprendimas, kuriame 3D objektų atpažinimas yra konstruojamas naudojantis tik 2D nuotraukomis, yra tikslesnis 8 \%. Algoritmas, naudojantis 3D modelius, pasiekė 77 \% tikslumą, o algoritmas, naudojantis tik 2D nuotraukas, pasiekė 85 \% tikslumą. Šaltinyje \cite{cnnExp1} yra teigiama, kad to priežastis yra reliatyviai efektyvesnis 2D nuotraukų informacijos saugojimas negu 3D modelių. Todėl, kad, nors 3D modelis turi visą informaciją apie atvaizduotą 3D objektą, tačiau tam, kad panaudoti vokselinę 3D objekto reprezentaciją mašininiame mokyme, kurio mokymas su pakankamai didele duomenų imtimi užtruktų racionalų laiko tarpą, tenka ženkliai sumažinti 3D modelio rezoliuciją. Pavyzdžiui, 3D modelio, kurio rezoliucija yra $30\times30\times30$ vokseliai, įvesties dydis yra apytiksliai lygus 2D paveikslėlio, kurio rezoliucija yra $164\times164$ pikseliai. Tad šiuo atveju, 3D modelis yra apdorojamas per tiek pat laiko kaip ir 2D paveikslėlis, bet modelio rezoliucija yra apytiksliai 5.5 karto mažesnė. Todėl mašininio mokymo metodas, kurio mokymo duomenys yra 3D modelis, gauna mažesnės raiškos įvestį, negu metodas, kurio mokymo duomenys yra 2D paveikslėliai.

\subsection{3D objektų atpažinimo iš 2D nuotraukų uždavinio sprendinių pavyzdžiai}

Vienas seniausių šio uždavinio sprendinių, taikantis tokią metodologiją, yra aprašytas darbe \cite{prevWparEig}. Šis sprendinys atpažįsta 3D objektus lygindamas jų vaizdus, kurie buvo suformuoti iš didelės imties 2D nuotraukų, parametrizuotoje 
eigenerdvėje (angl. eigenspace) % Kaip lietuviškai?
. Šios nuotraukos buvo sugeneruotos iš 3D modelių naudojant skirtingus apžvalgos taškus ir apšvietimus. 
Kitas pavyzdys, kuris yra gana populiarus kompiuterinėje grafikoje, yra 
šviesos lauko deskriptorius (angl. light field descriptor)% light field descriptor
, kuris yra aprašytas darbe \cite{prevWLightFld}. Šis sprendinys išgauna geometrinius ir 
Fourier'io % kaip daryti su angliškais vardais?
deskriptorius iš 3D objektų siluetų, kurie buvo sugeneruoti iš 3D modelių, naudojant skirtingus apžvalgos taškus. 
Darbe \cite{prevWShockGraph} aprašytas šio uždavinio sprendimas, kuris 3D objekto siluetus išskaido į dalis ir išsaugo juos į 
orientuotą beciklį grafą (angl. directed acyclic graph) % directed acyclic graph 
, šoko grafą. % shock graph
Kitas pavyzdys aprašytas darbe \cite{prevWSimMet}, naudoja panašumo metrikas (angl. similarity metrics) % similarity metrics
, kurios yra pagrįstos kreivių palyginimu (angl. curve matching)% curve matching
ir sugrupuotomis panašiomis 2D nuotraukomis.

Šiuo metu, 3D objektų atpažinimo iš 2D nuotraukų uždaviniui spręsti, optimaliausius  rezultatus, laiko ir tikslumo atžvilgiu, pasiekęs mašininio mokymo metodu pagrįstas sprendimas yra konvoliuciniai dirbtiniai neuroniniai tinklai. Tai eksperimentu buvo įrodyta darbe \cite{cnnExp1}. Šiame eksperimente buvo palyginti įvairūs konvoliucinių neuroninių tinklų tipai sprendžiant šį uždavinį ir geriausią rezultatą pasiekęs tipas buvo
daugiavaizdis (angl. multi-view convolutional network)% multi-view
konvoliucinis neuroninis tinklas, kurio tikslumas buvo 90.1\%.

\subsection{Dirbtinių neuroninių tinklų bendrieji principai}
\subsubsection{Dirbtinis neuronas, perceptronas}

Šiame magistro baigiamajame darbe nagrinėjami dirbtiniai neuroniniai tinklai yra sudaryti iš Rosenblato darbe \cite{rosenPerc} aprašytų dirbtinių neuronų, perceptronų. Perceptronas --  tai iteratyviai apmokomas tiesinis klasifikatorius, kuris susideda iš $\boldsymbol{x} = \{x_{0}, x_{1}, x_{2}, ..., x_{p}\}$ mokymo aibės vektorių, vadinamais įėjimais, $\{w_{0}, w_{1}, w_{2}, ..., w_{p}\} \in \R$ perdavimo koeficientų, vadinamų svoriais, aktyvacijos (perdavimo) funkcijos $f(a)$ ir $\{y_{0}, y_{1}, y_{2}, ..., y_{n}\}$ reikšmių, vadinamų išėjimais. Įėjimas $x_{0}$ yra vadinamas nuliniu įėjimu ir jo reikšmė yra pastovi $x_{0} = 1$, o $w_{0}$ - nuliniu svoriu arba slenksčiu (angl. bias). Perceptronas yra atvaizduotas \ref{img:perceptron} paveikslėlyje.

\begin{figure}[H]
	\centering
	\includegraphics[scale=0.5]{img/perceptron.png}
	\caption{Perceptronas}
	\label{img:perceptron}
\end{figure}

Formulė (\ref{eqn:activ_arg}) yra aktyvacijos funkcijos argumentas.

\begin{equation}
	\label{eqn:activ_arg}
	a = \sum_{k = 0}^{p} w_{k}x_k
\end{equation}

Dažniausiai perceptronui yra naudojamos šios aktyvacijos funkcijos: slenkstinė (angl. unit step) (\ref{eqn:unitStep}), sigmoidinė (angl. sigmoid) (\ref{eqn:sigmoid}), gabalais tiesinė (angl. piecewise linear) (\ref{eqn:pieceLinear}), Gauso (angl. Gaussian) (\ref{eqn:gaussian}) ir tiesinė (angl. linear) (\ref{eqn:linear}), kur $\beta$, $\mu$, $\sigma$, $m$, $a_{min}$, $a_{max}$ yra konstantos priklausančios realiųjų skaičių aibei bei $a_{min} < a_{max}$.

\begin{equation}
	\label{eqn:unitStep}
	f(a) =
	\begin{cases}
		0, & \mbox{jei } 0 > a \\
		1, & \mbox{jei } 0 \leq a
	\end{cases}
\end{equation}

\begin{equation}
	\label{eqn:sigmoid}
	f(a) = \dfrac{1}{1 + \exp(-a)}
\end{equation}

\begin{equation}
	\label{eqn:pieceLinear}
	f(a) =
	\begin{cases}
		0, & \mbox{jei } a_{min} \geq a \\
		ma + b, & \mbox{jei } a_{min} < a < a_{max} \\
		1, & \mbox{jei } a_{max} \leq a
	\end{cases}
\end{equation}

\begin{equation}
	\label{eqn:gaussian}
	f(a) = \dfrac{1}{\sqrt{2\pi\sigma}} \exp(\dfrac{-(a - \mu)^2}{2\sigma^2})
\end{equation}

\begin{equation}
	\label{eqn:linear}
	f(a) = ma + b
\end{equation}

Perceptronas yra skirtas spręsti klasifikavimo uždavinius. Tam kad perceptronas spręstų konkretų klasifikavimo uždavinį, jis turi būti apmokytas. Perceptrono apmokymas yra iteratyvus procesas, kuriame randami svoriai $W = \{w_{0}, w_{1}, w_{2}, ..., w_{p}\}$, su kuriais funkcijos (\ref{eqn:mse}) rezultatas įgyja kiek galima mažiausią reikšmę. Funkcijoje (\ref{eqn:mse}) $y_i$ yra perceptrono \textit{i}-tasis išėjimas, $t_i$ - \textit{i}-tojo įėjimo norima klasė ir $n$ - apmokymo duomenų vektorių skaičius.

\begin{equation}
	\label{eqn:mse}
	e(w) = \dfrac{1}{n}\sum_{i=1}^{n}(y_i - t_i)^2
\end{equation}

Apmokymo pradžioje pradiniai svoriai yra parenkami atsitiktinai. Toliau gradientinio nusileidimo algoritmu judant antigradiento kryptimi, svorių reikšmės perskaičiuojamos naudojantis formule (\ref{eqn:w_recalc}), kur \textit{k}-tojo svorio gradientas $\Delta w_k(t)$ apskaičiuojamas pagal formulę (\ref{eqn:w_change}), $t$ - iteracijos numeris, $\eta \in [0, +\infty]$ - parinktas mokymo greitis (angl. learning rate). Vienoje iteracijoje yra naudojamas tik vienas įėjimo vektorius iš duomenų aibės. Svoriai yra perskaičiuojami norima skaičių kartų.

\begin{equation}
	\label{eqn:w_recalc}
	w_k(t + 1) = w_k(t) + \Delta w_k(t)
\end{equation}

\begin{equation}
	\label{eqn:w_change}
	\Delta w_k(t) = - \eta \dfrac{\partial e(w)}{\partial w_k}
\end{equation}
% išsivedamas bendras atvejis ------------------------------------------------------------------------------------------------------------------------
Pritaikius formulę (\ref{eqn:activ_arg}) \textit{i}-tojo įėjimo vektoriaus aktyvacijos funkcijos argumento apskaičiavimui gaunama formulė (\ref{eqn:activ_arg_per_i}), kur $a_i$ yra i-tojo įėjimo vektoriaus aktyvacijos funkcijos argumentas, $x_{ik}$ yra \textit{i}-tojo įėjimo vektoriaus \textit{k}-atoji komponentė.

\begin{equation}
	\label{eqn:activ_arg_per_i}
	a_i = \sum_{k = 0}^{p} w_{k}x_{ik}
\end{equation}

Tad \textit{i}-tasis perceptrono išėjimas $y_i$ yra $y_i = f(a_i)$. Tada funkcijos (\ref{eqn:mse}) išvestinė yra paskaičiuojama pagal formulę (\ref{eqn:expanded}).

\begin{equation}
	\label{eqn:expanded}
	\dfrac{\partial e(w)}{\partial w_k} = (\dfrac{1}{n}\sum_{i=1}^{n} (y_i - t_i)^2)'
		= \dfrac{2}{n}\sum_{i=1}^{n}
			((y_i - t_i)(f'(a_i))(\sum_{k = 1}^{p} x_{ik}))
\end{equation}

Tada bendru atveju perceptrono mokymo taisyklė (\ref{eqn:w_recalc}) yra funkcija (\ref{eqn:general}).

\begin{equation}
	\label{eqn:general}
	w_k(t + 1) = w_k(t) - \eta \dfrac{2}{n}\sum_{i=1}^{n} ((y_i - t_i)(f'(a_i))(\sum_{k = 0}^{p} x_{ik}))
\end{equation}

% išsivedamas bendras atvejis ------------------------------------------------------------------------------------------------------------------------
Naudojantis apmokytu perceptronu galima nustatyti ar duotas duomenų vektorius $\boldsymbol{x}'$ priklauso klasei $c$. Pirmiausiai randamas skiriamasis paviršius (angl. decision boundary). Skiriamasis paviršius - tai kreivė, gaunama iš formulės (\ref{eqn:activ_arg}) su apmokyto perceptrono svoriais ir kai $a = d$, kur $d$ yra konstanta, su kuria vektoriai, kurie patenkina sąlygą $d \ge f(a)$, yra interpretuojami kaip nepriklausantys klasei $c$. Pavyzdžiui, aktyvacijos funkcijos (\ref{eqn:sigmoid}) konstanta $d = 0,5$. Skiriamasis paviršius padalina duomenų vektorių erdvę į du regionus. Jei $\boldsymbol{x}'$ priklauso regionui, kuriame vektoriai patenkina sąlygą $a > d$, kur $a$ randamas naudojant formulę (\ref{eqn:activ_arg}) su apmokyto perceptrono svoriais ir $\{x_{0}, x_{1}, x_{2}, ..., x_{p}\} = \boldsymbol{x}'$, tai $\boldsymbol{x}'$ priklauso klasei $c$, kitu atveju - ne.
\subsubsection{Dirbtinis neuronas, perceptronas}

Perceptronas -  tai iteratyviai apmokomas tiesinis klasifikatorius, kuris susideda iš $\{x_{0}, x_{1}, x_{2}, ..., x_{n}\}$ mokymo aibės vektorių, vadinamais įėjimais, $\{w_{0}, w_{1}, w_{2}, ..., w_{n}\} \in \R$ perdavimo koeficientų, vadinamų svoriais, aktyvacijos (perdavimo) funkcijos $f(a)$ ir $\{y_{0}, y_{1}, y_{2}, ..., y_{n}\}$ reikšmių, vadinamų išėjimais. Įėjimas $x_{0}$ yra vadinamas nuliniu įėjimu ir jo reikšmė yra pastovi $x_{0} = 1$, o $w_{0}$ - nuliniu svoriu arba slenksčiu (angl. bias). Aktyvacijos funkcijos argumentas yra įėjimo reikšmių ir svorių sandaugų suma:

\begin{equation}
	a = \sum_{k = 1}^{n} w_{k}x_{k}
\end{equation}

Dažniausiai yra naudojamos šios aktyvacijos funkcijos: slenkstinė (angl. unit step) \ref{eqn:unitStep}, sigmoidinė (angl. sigmoid) \ref{eqn:sigmoid}, gabalais tiesinė (angl. piecewise linear) \ref{eqn:pieceLinear}, Gauso (angl. Gaussian) \ref{eqn:gaussian} ir tiesinė (angl. linear) \ref{eqn:linear}

\begin{equation}
\label{eqn:unitStep}
	f(a) =
	\begin{cases}
		0, & \mbox{if } \beta > a \\
		1, & \mbox{if } \beta \leq a
	\end{cases}
\end{equation}

\begin{equation}
	\label{eqn:sigmoid}
	f(a) = \dfrac{1}{1 + \exp^{-\beta a}}
\end{equation}

\begin{equation}
	\label{eqn:pieceLinear}
	f(a) =
	\begin{cases}
		0, & \mbox{if } a_{min} \geq a \\
		ma + b, & \mbox{if } a_{min} < a < a_{max} \\
		1, & \mbox{if } a_{max} \leq a
	\end{cases}
\end{equation}

\begin{equation}
	\label{eqn:gaussian}
	f(a) = \dfrac{1}{\sqrt{2\pi\sigma}} \exp^{\dfrac{-(x - \mu)^2}{2\sigma^2}}
\end{equation}

\begin{equation}
	\label{eqn:linear}
	f(a) = ma + b
\end{equation}

Perceptrono mokymas yra iteratyvus procesas, kuriame randami svoriai $W = \{w_{0}, w_{1}, w_{2}, ..., w_{n}\}$, su kuriais funkcijos \ref{eqn:mse} rezultatas įgyja mažiausią reikšmę. Funkcijoje \ref{eqn:mse} $y_i$ yra perceptrono i-tasis išėjimas ir $t_i$ - i-tojo įėjimo norima klasė.

% coming up next: rasyk kad generuoja random pirminius svorius, tada minimizavimo funkcija

\begin{equation}
\label{eqn:mse}
e(w) = \dfrac{1}{n}\sum_{i=1}^{n}(y_i - t_i)^2
\end{equation}

\subsubsection{Dirbtiniai neuroniniai tinklai}
\subsubsection{Gilieji neuroniniai tinklai}

% Apmokymas vyksta epochomis, batcho dydis

\subsection{Daugiavaizdžių konvoliucinių dirbtinių neuroninių tinklų apžvalga}
\subsubsection{Konvoliucija}

Konvoliuciniai neuroniniai tinklai yra vieni populiariausių giliųjų neuroninių tinklų tipas. Pirma karta sėkmingai įgyvendintas konvoliucinis neuroninis tinklas yra aprašytas darbe \cite{cnn}. Šis tinklas yra skirtas ranka rašytiems pašto kodams atpažinti. Konvoliucinis neuroninis tinklas - tai gilusis neuroninis tinklas, kurio bent viename sluoksnyje yra naudojama konvoliucijos operacija, dar vadinama sąsuka. Konvoliucija - tai matematinė operacija, kurios operandai yra dvi funkcijos $f$ ir $g$, ir kurios rezultatas yra funkcija, kuri apibūdina kaip viena funkcija keičia kitą. Ši operacija yra žymima $f * g$ ir ji yra apibrėžiama kaip integralinės transformacijos rūšis pavaizduota formulėje \ref{eqn:convolution}, kur a ir b nurodo funkcijų  $f$ ir $g$ apibrėžimo sritį.

\begin{equation}
\label{eqn:convolution}
	(f * g)(t) = \int_{a}^{b} f(\tau)g(t - \tau) d\tau
\end{equation}

Konvoliucijos algebros savybės yra komutatyvumas ($f * g = g * f$), asociatyvumas ($f * (g * h) = (f * g) * h$), distributyvumas ($f * (g + h) = (f * g) + (f * h)$), vienetinis elementas $f * \delta = \delta * f = f$ ir daugybos su skaliaru asociatyvumas ($a(f * g) = (af) * g = f * (ag)$, kur $a \in \R$).

Dažniausiai konvoliuciniuose neuroniniuose tinkluose yra vykdoma konvoliucija diskrečioms funkcijoms. Konvoliucija, kurios operandai yra diskrečios funkcijos yra vadinama diskreti konvoliucija ir ji yra apibrėžiama kaip formulė \ref{eqn:discrete_convolution}, kur a ir b nurodo funkcijų  $f$ ir $g$ apibrėžimo sritį.

\begin{equation}
\label{eqn:discrete_convolution}
	(f * g)(t) = \sum_{\tau = a}^{b} f(\tau)g(t - \tau)
\end{equation}

Šio darbo tyrimuose yra naudojamos 2D nuotraukos, kurios yra saugomos kaip dviejų dimensijų vaizdai, vadinamos matricomis. Diskreti konvoliucija matricoms yra atliekama naudojantis formulę \ref{eqn:matrix_convolution}.

\begin{equation}
\label{eqn:matrix_convolution}
	(I * K)(i, j) = \sum_{m} \sum_{n} I(m, n) K(i - m, j - n)
\end{equation}

Konvoliuciniuose neuroniniuose tinkluose matrica $I$ yra vadinama įvestimi, o matrica $K$ - branduoliu arba filtru. Konvoliucija yra komutatyvi, todėl formulė \ref{eqn:matrix_convolution} gali būti išreikšta kaip \ref{eqn:cnn_convolution}.

\begin{equation}
\label{eqn:cnn_convolution}
	(K * I)(i, j) = \sum_{m} \sum_{n} I(i - m, j - n) K(m, n)
\end{equation}

Dažniausiai ši išraiška yra naudojama konvoliuciniuose neuroniniuose tinkluose. Branduolys $K$ dažniausiai yra žymiai mažesnio dydžio nei įvesties matrica $I$ ir $K$ dažniausiai yra išretinta matrica (angl. sparse matrix). Išretinta matrica yra matrica, kurios didžioji dalis elementų yra lygūs 0. Konvoliucijos naudojimas giliuosiuose neuroniniuose tinkluose pagreitina mašininį mokymąsi dėl konvoliucijos principų - išretintos sąveikos (angl. sparsity), parametrų pasidalinimo ir ekvivalentiško atvaizdavimo.

Paprastame daugiasluoksniame perceptrone kiekvieno sluoksnio visi perceptronai turi po vieną jungtį su tolimesnio sluoksnio kiekvienu perceptronu. Tuo metu konvoliuciniuose tinkluose yra taikoma išretinta sąveika. Išretinta sąveika - tai konvoliucijos padarinys dirbtiniam neuroniniam tinklui, dėl kurios sluoksniuose, kuriuose taikoma konvoliucija, sumažėja jungčių skaičius su tolimesnio sluoksnio perceptronais. Išretintos sąveikos tarp dviejų sluoksnių pavyzdys yra pateiktas paveikslėlyje \ref{img:sparsity}

\begin{figure}[H]
	\centering
	\includegraphics[scale=0.5]{img/sparsity.png}
	\caption{Išretinta sąveika}
	\label{img:sparsity}
\end{figure}

Dėl išretintos sąveikos konvoliuciniai neuroniniai tinklai atsižvelgia tik į reikšmingus požymius. Todėl konvoliucinių neuroninių tinklų mokymas trunka trumpiau ir triukšmas turi mažesnę įtaką rezultatui.

Kitas principas dėl kurio konvoliuciniai neuroniniai tinklai yra spartesni nei daugiasluoksniai perceptronai yra parametrų pasidalinimas. Parametrų pasidalinimas yra 

\subsubsection{Konvoliucinio neuroninio tinklo sluoksnių tipai}

Konvoliucinio neuroninio tinklo sluoksniai yra skirstomi į tipus. Pagrindinis konvoliucinio neuroninio tinklo sluoksnio tipas yra konvoliucijos sluoksnis. Šio sluoksnio paskirtis yra padalinti kiekvieną įėjimo vaizdą į lokalius fragmentus ir nustatyti kiekvieno fragmento atitikimą kiekvienam požymiui naudojantis konvoliucija. Konvoliucijos sluoksnio rezultatas yra požymių žemėlapiai (angl. feature map). Šiuose žemėlapiuose yra saugoma informacija - kiekvieno fragmento atitikimas konkrečiam požymiui. Kiekvienas požymių žemėlapis yra sudaromas naudojantis unikalų, tik jam priskirtą filtrą. I-tasis požymių žemėlapis $Y_i^{(l)}$, priklausantis l-tajam sluoksniui, yra apskaičiuojamas pagal formulę \ref{eqn:feature_map}, kur $Y_j^{l-1}$ yra j-tasis praeitas sluoksnis, $m_1^{(l-1)}$ - praeito sluoksnio perceptronų išėjimų skaičius, $K_{i,j}^{(l)}$ - naudojamas filtras l-tajame sluoksnyje apskaičiuoti i-tąjį požymių žemėlapį j-tajam įėjimo vaizdui, ir $B_i^{(l)}$ - tai i-toji l-sluoksnio postūmio matrica (angl. bias matrix). Konvoliucijos sluoksnio perceptronų išėjimai yra šie žemėlapiai.


\begin{equation}
\label{eqn:feature_map}
	Y_i^{(l)} = B_i^{(l)} + \sum_{j = 1}^{m_1^{(l-1)}} K_{i,j}^{(l)} * Y_j^{l-1}
\end{equation}

Po konvoliucijos sluoksnio tolimesnis sluoksnio tipas konvoliuciniame neuroniniame tinkle yra netiesiškumo sluoksnis (angl. non-linearity layer). Šis sluoksnis yra sudarytas iš aktyvacijos funkcijos ir šio sluoksnio rezultatas yra šios funkcijos rezultatas, vadinamas aktyvacijos žemėlapiu. Netiesiškumo sluoksnio aktyvacijos funkcija gali būti bet kuri funkcija, kuri taip pat yra naudojama ir vienasluoksniame perceptrone, ir kurios argumentas yra matrica ir rezultatas yra matrica, kurios matmenys yra lygūs argumento matricos matmenims. Dažniausiai aktyvacijos funkcijos yra sigmoidinė, hiperbolinio tangento ir ištaisymo tiesinė (angl. rectified linear function (ReLu)) funkcijos.

% cont 36 slide


\subsubsection{Daugiavaizdis konvoliucinis neuroninis tinklas}

Daugiavaizdis konvoliucinis neuroninis tinklas yra aprašytas darbe \cite{cnnExp1}. Daugiavaizdis konvoliucinis neuroninis tinklas - tai konvoliucinis neuroninis tinklas, turintis vieną vaizdų sujungimo sluoksnį (angl. view pooling layer). Vaizdų sujungimo sluoksnis - tai sluoksnis, kuriame kiekvieno duomenų rinkinio požymių žemėlapių rinkiniai, vadinami vaizdais, yra apjungiami. Vaizdų apjungimas yra atliekamas padalinant visus vaizdus į grupes su nurodytu tuo pačiu dydžiu, ir išsirenkant iš kiekvienos grupės po tiksliausią požymių žemėlapių rinkinį.

Daugiavaizdžio konvoliucinio neuroninio tinklo apmokymas vyksta dvejais etapais. Pirmasis etapas yra pasirinkto konvoliucinio neuroninio tinklo apmokymas. Tada antrasis etapas yra vaizdų apjungimo sluoksnio įterpimas ir apmokymo pratęsimas. Šis sluoksnis padalina apmokytą konvoliucinį neuroninį tinklą į du tinklus $C_1$ ir $C_2$. Tęsiant apmokymą, kiekviena 2D nuotrauka atskirai pereis $C_1$ tinklą. Tada vaizdų apjungimo sluoksnyje šios nuotraukos bus apjungiamos. Pabaigoje vaizdų apjungimo sluoksnio rezultatas pereis tinklą $C_2$. Daugiavaizdis konvoliucinis neuroninis tinklas yra atvaizduotas \ref{img:mvcnn} paveikslėlyje.

\begin{figure}[H]
	\centering
	\includegraphics[scale=0.5]{img/mvcnn.png}
	\caption{Daugiavaizdis konvoliucinis neuroninis tinklas}
	\label{img:mvcnn}
\end{figure}

Darbo \cite{cnnExp1} autoriai teigia, kad teoriškai vaizdų apjungimo sluoksnį galima įterpti į bet kurią apmokyto konvoliucinio tinklo vietą. Tačiau tame darbe atlikti tyrimai parodė, kad didžiausias tikslumas yra pasiekiamas įterpus šį sluoksnį šalia paskutinio konvoliucijos sluoksnio.

Daugiavaizdžiai konvoliuciniai neuroniniai tinklai dažniausiai yra naudojami spręsti uždavinius su 2D nuotraukomis, kuriuose yra atvaizduoti 3D objektai. Tam 2D nuotraukos iš apmoymo duomenų yra sugrupuojamos pagal tai kokį 3D objektą jos atvaizduoja ir šios grupės yra pateikiamos daugiavaizdžiui konvoliuciniui neuroniniui tinklui kaip duomenų rinkiniai.

Darbe \cite{cnnExp1} aprašytas daugiavaizdis konvoliucinis neuroninis tinklas pirmame apmokymo etape naudoja VGG-M architektūra, kuri yra aprašyta darbe \cite{vggM}. Tačiau darbe \cite{cnnExp2} yra pasirinkta VGG-11 architektūra ir darbe \cite{cnnExp2} atliktame tyrime  daugiavaizdis konvoliucinis neuroninis tinklas pasiekė šiek tiek geresnius rezultatus. VGG-11 architektūra yra aprašyta darbe \cite{vgg11}. Šios architektūros konfigūracija yra atvaizduota lentelėje \ref{tbl:vgg11}. VGG-11 architektūroje po kiekvieno konvoliucijos ir pilnai sujungto sluoksnio, išskyrus paskutinio, tolimesnis sluoksnis yra apjungtas netiesiškumo ir ištaisymo sluoksnis su ištaisymo tiesine aktyvacijos funkcija. Tad, tam kad padaryti architektūros atvaizdavimą paprastesnį, šie sluoksniai nėra atvaizduoti lentelėje \ref{tbl:vgg11}. Po paskutinio pilnai sujungto sluoksnio tolimesnis sluoksnis yra netiesiškumo sluoksnis su softmax aktyvacijos funkcija. Taip pat šioje architektūroje visų konvoliucinių sluoksnių branduolių matmenys yra 3x3 ir lango žingsnis yra 1. Tuo metu visų sujungimo sluoksnių langų matmenys yra 2x2, lango žingsniai yra 2 ir visi sujungimo sluoksniai naudoja maksimalaus sujungimo metodą. Tad lentelėje \ref{tbl:vgg11} šie parametrai nėra atvaizduojami.

\begin{table}[h]
	\begin{tabular}{|l|c|l|}
		\hline
		Sluoksnio žymėjimas & Sluoksnio tipas            & \multicolumn{1}{c|}{Parametrai}       \\ \hline
		& Įėjimo sluoksnis           & Įėjimo matmenys = 224x224 RGB matrica \\ \hline
		$c_1$               & Konvoliucijos sluoksnis    & Branduolių skaičius = 64              \\ \hline
		$p_1$               & Sujungimo sluoksnis        &                                       \\ \hline
		$c_1$               & Konvoliucijos sluoksnis    & Branduolių skaičius = 128             \\ \hline
		$p_1$               & Sujungimo sluoksnis        &                                       \\ \hline
		$c_1$               & Konvoliucijos sluoksnis    & Branduolių skaičius = 256             \\ \hline
		$c_1$               & Konvoliucijos sluoksnis    & Branduolių skaičius = 256             \\ \hline
		$p_1$               & Sujungimo sluoksnis        &                                       \\ \hline
		$c_1$               & Konvoliucijos sluoksnis    & Branduolių skaičius = 512             \\ \hline
		$c_1$               & Konvoliucijos sluoksnis    & Branduolių skaičius = 512             \\ \hline
		$p_1$               & Sujungimo sluoksnis        &                                       \\ \hline
		$c_1$               & Konvoliucijos sluoksnis    & Branduolių skaičius = 512             \\ \hline
		$c_1$               & Konvoliucijos sluoksnis    & Branduolių skaičius = 512             \\ \hline
		$p_1$               & Sujungimo sluoksnis        &                                       \\ \hline
		$fc_1$              & Pilnai sujungtas sluoksnis & Neuronų skaičius = 4096               \\ \hline
		$fc_1$              & Pilnai sujungtas sluoksnis & Neuronų skaičius = 4096               \\ \hline
		$fc_1$              & Pilnai sujungtas sluoksnis & Neuronų skaičius = 1000               \\ \hline
		$fc_1$              & Pilnai sujungtas sluoksnis & Aktyvacijos funkcija - softmax        \\ \hline
	\end{tabular}
	\caption{VGG-11 architektūra}
	\label{tbl:vgg11}
\end{table}

Pirmame ir antrame apmokymo etape yra optimizuojama kryžminės entropijos nuostolių funkcija naudojantis Adam optimizavimo algoritmu. Antro etapo pradžioje vaizdų apjungimo sluoksnis yra įterpiamas tarp sluoksnių $p_5$ ir $fc_1$. Šio sluoksnio vaizdų grupių dydžiai yra lygūs 12.

% more info on research result or maybe write about it in research section?

\subsection{Kapsulinių neuroninių tinklų apžvalga}
Kapsuliniai neuroniniai tinklai yra aprašyti darbe \cite{capsNet}. Kapsuliniai neuroniniai tinklai yra giliųjų neuroninių tinklų tipas, kurio sluoksnio perceptronai yra grupuojami į kapsules. Kiekviena kapsulė apskaičiuoja tikimybę, kad paveikslėlyje pavaizduotas objektas priklauso kažkuriai klasei, ir išgauna informaciją apie tokius objekto bruožus kaip pozicija, orientacija, mastelis, deformacija, spalva ir kitus panašius objekto bruožus. Pirminių kapsulių sluoksniuose nagrinėjami objektai yra paprastos geometrinės figūros. Tolimesniuose sluoksniuose objektai darosi sudėtingesni, jie ima atitikti realaus pasaulio objektus. Kapsulės tarp sluoksnių yra sujungiamos į hierarchiją. Taip kapsulinis neuroninis tinklas sukuria hierarchinę vaizdo reprezentaciją.

Pirmieji du sluoksniai kapsuliniame neuroniniame tinkle yra konvoliucijos sluoksnis ir apjungtas netiesiškumo ir ištaisymo sluoksnis su ištaisymo tiesine aktyvacijos funkcija. Šių sluoksnių tikslas yra išgauti pagrindinius požymius, kurie tolimesniame sluoksnyje yra naudojami objektų konstrukcijai.

Tolimesnio sluoksnio tipas yra pirminės kapsulės (angl. primary capsules). Šiame sluoksnyje aktyvacijos žemėlapiai yra konvertuojami į vektorius. Toliau kiekvienas vektorius atskirai yra pateikiamas squash funkcijai kaip argumentai. Squash funkcija yra formulė \ref{label}, kur $||s||$ yra visų matricos $s$ narių suma.

\begin{equation}
\label{eqn:squash}
	squash(s) = \dfrac{||s||^2}{1 + ||s||^2}\dfrac{s}{||s||}
\end{equation}

Tolimesnių sluoksnių tipai yra kapsulės. Šiuose sluoksniuose yra vykdomas dinaminis maršrutizavimas tarp kapsulių (angl. dynamic routing between capsules). Dinaminis maršrutizavimas tarp kapsulių - tai iteratyvus procesas, kurio paskirtis yra apjungti kapsules tarp dviejų sluoksnių. Prieš pradedant iteratyvią proceso dalį, kiekvienai sluoksnio $l$ kapsulei $i$ ir sluoksnio $(l + 1)$ kapsulei $j$ yra inicializuojami kintamieji $b_{ij}$ su reikšme 0. Taip pat kiekvienai kapsulių $i$ ir $j$ porai yra apskaičiuojami vektoriai $\hat{u}_{j|i}$ pagal formulę \ref{eqn:pred_vectors}, kur $W_{ij}$ yra svorio matrica tarp kapsulių $i$ ir $j$ bei $u_{i}$ - tai kapsulės $i$ išvestis.

\begin{equation}
\label{eqn:pred_vectors}
	\hat{u}_{j|i} = W_{ij} u_{ij}
\end{equation}

Tada pirmasis iteratyvaus proceso žingsnis yra apskaičiuoti apjungimo koeficientus $c_{ij}$ kiekvienai kapsulių $i$ ir $j$ porai pagal softmax funkciją atvaizduota formulėje \ref{eqn:coupling_coef}, kur $n$ yra sluoksnio $(l + 1)$ kapsulių skaičius.

\begin{equation}
\label{eqn:coupling_coef}
	c_{ij} = \dfrac{\exp^{b_{ij}}}{\sum_{k = 1}^{n} \exp^{b_{ik}}}
\end{equation}

Tolimesnis žingsnis yra apskaičiuoti svertines sumas $s_j$ kiekvienai kapsulei $j$ naudojantis formulę \ref{eqn:weighted_sum}, kur $m$ yra kapsulių skaičius sluoksnyje $l$.

\begin{equation}
\label{eqn:weighted_sum}
	s_{j} = \sum_{i = 1}^{m} c_{ij} \hat{u}_{j|i}
\end{equation}

Toliau yra apskaičiuojami vektoriai $v_j$ kiekvienai kapsulei $j$ naudojantis softmax funkcija su argumentu $s_j$. Kitaip tariant yra apskaičiuojama formulė $v_j = softmax(s_j)$. Paskutinis iteratyvios dalies žingsnis yra pakeisti kintamųjų $b_{ij}$ reikšmes naudojantis formulę $b_{ij} = b_{ij} + \hat{u}_{j|i} v_j$.

Iteratyvi dinaminio maršrutizavimo tarp kapsulių proceso dalis yra kartojama nurodyta skaičių iteracijų ir šio proceso rezultatas yra vektorius $v_j$. Šiame vektoriuje yra tikimybės, kad objektas, nagrinėjamas kapsulės $i$, yra dalis objekto, nagrinėjamo kapsulės $j$.


\section{Kapsulinių neuroninių tinklų modifikacijos ir parametrai}

\subsection{Tiriamo kapsulinio neuroninio tinklo architektūra}

Šiame darbe viena iš tiriamų kapsulinio neuroninio tinklo architektūrų yra aprašyta darbe \cite{capsNet}. Šios architektūros konfigūracija yra atvaizduota \ref{tbl:capsNet} lentelėje. Paskutinis sluoksnis $ed$ yra skirtas konvertuoti paskutinio kapsulinio sluoksnio išėjimų vektorius į skaliarines reikšmes, kurias lengviau interpretuoti kaip tikimybes, kad paveikslėlyje pavaizduotas 3D objekto modelis priklauso vienai iš klasių. Konvertavimas vyksta apskaičiuojant euklidinį atstumą nuo nulinio taško iki vektoriaus taško.

\begin{table}[]
\caption{Tiriamo kapsulinio neuroninio tinklo architektūra}
\begin{tabular}{|l|l|l|}
\hline
Sluoksnio žymėjimas & Sluoksnio tipas                     & Parametrai                                                                                                                                                 \\ \hline
                    & Įėjimo sluoksnis                    & Įėjimo matmenys = 150x150 RGB matrica                                                                                                                      \\ \hline
conv1               & Konvoliucijos sluoksnis             & \begin{tabular}[c]{@{}l@{}}Branduolių skaičius = 256\\ Branduolių dydis = 9x9\\ Lango žingsnis  = 1\end{tabular}                                           \\ \hline
relu1               & \begin{tabular}[c]{@{}l@{}}Netiesiškumo ir ištaisymo\\ sluoksnis\end{tabular} & aktyvacijos funkcija - ReLu                                                                                                                                 \\ \hline
conv2               & Konvoliucijos sluoksnis             & \begin{tabular}[c]{@{}l@{}}Branduolių skaičius = 256\\ Branduolių dydis = 9x9\\ Lango žingsnis  = 1\end{tabular}                                           \\ \hline
pc                  & Pirminių kapsulių sluoksnis         & Išėjimo vektorių dimensijų skaičius = 8                                                                                                           \\ \hline
c                   & Kapsulinis sluoksnis                & \begin{tabular}[c]{@{}l@{}}Kapsulių skaičius = 40\\ Išėjimų vektorių dimensijų skaičius = 16\\ Maršrutizavimo iteracijų skaičius = 3\end{tabular} \\ \hline
ed                  & Euklidinis atstumas                 &                                                                                                                                                            \\ \hline
\end{tabular}
\label{tbl:capsNet}
\end{table}

Šio tinklo naudojamas rekonstrukcijos tinklas yra atvaizduotas \ref{tbl:capsNetRecon} lentelėje. Šio tinklo įėjimo sluoksnis yra kapsulinio neuroninio tinklo sluoksnis $c$. Pirmasis rekonstrukcijos tinklo sluoksnis $r\_m$ apmokymo metu palieka tik kapsulės, kuri reprezentuoja klasę, kuriai priklauso paveikslėlyje pavaizduotas 3D modelis, išėjimo vektorių, visos kitos reikšmės yra padauginamos iš 0. Klasifikavimo atveju, šis sluoksnis palieka kapsulės išėjimo vektorių, kurio euklidinis atstumas nuo nulinio taško yra didžiausias, visos kitos reikšmės taip pat yra padauginamos iš 0. Šio sluoksnio išėjimas yra transformuota kapsulių išėjimų vektorių aibė į vieną vektorių. Sluoksnio $r\_t$ išėjimas yra rekonstruota 2D nuotrauka.

\begin{table}[]
\caption{Tiriamo kapsulinio neuroninio tinklo rekonstrukcijos tinklo architektūra}
\begin{tabular}{|l|l|l|}
\hline
Sluoksnio žymėjimas & Sluoksnio tipas            & Parametrai                                                                                           \\ \hline
r\_m                & Maskavimo sluoksnis        & Išėjimo matmenys = 640                                                                               \\ \hline
r\_fc               & Pilnai sujungtas sluoksnis & \begin{tabular}[c]{@{}l@{}}Neuronų skaičius = 512\\ Aktyvacijos funkcija - ReLu\end{tabular}         \\ \hline
r\_fc               & Pilnai sujungtas sluoksnis & \begin{tabular}[c]{@{}l@{}}Neuronų skaičius = 1024\\ Aktyvacijos funkcija - ReLu\end{tabular}        \\ \hline
r\_fc               & Pilnai sujungtas sluoksnis & \begin{tabular}[c]{@{}l@{}}Neuronų skaičius = 22500\\ Aktyvacijos funkcija - sigmoidinė\end{tabular} \\ \hline
r\_t                & Transformacijos sluoksnis  & Išėjimo matmenys = 150x150 RGB matrica                                                               \\ \hline
\end{tabular}
\label{tbl:capsNetRecon}
\end{table}

\subsection{Tiriamo daugiavaizdžio kapsulinio neuroninio tinklo architektūra su vaizdų sujungimo sluoksniu}

Taip pat šiame magistro baigiamajame darbe yra pritaikomas vaizdų sujungimo sluoksnio, kuris darbe \cite{cnnExp1} yra naudojamas konvoliuciniame neuroniniame tinkle, idėja kapsuliniam neuroniniui tinklui. Ši modifikacija yra vadinama daugiavaizdžiu kapsuliniu neuroniniu tinklu. Vaizdų sujungimo sluoksnis yra modifikuojamas taip, kad sluoksnio įėjimas, vietoje požymių žemėlapių rinkinių, būtų kapsulių išėjimo vektorių rinkiniai. Pirmame etape yra naudojama architektūra atvaizduota \ref{tbl:capsNet} lentelėje. Antrame etape vaizdų sujungimo sluoksnis yra įterpiamas po sluoksnio $ed$.

\subsection{Tiriamo daugiavaizdžio kapsulinio neuroninio tinklo architektūra su vaizdų kapsuliniu sluoksniu}

Vaizdų sujungimo sluoksnis yra pagrįstas maksimalaus sujungimo sluoksnio veikimo principu. Tačiau darbo \cite{capsNet} autorius teigia, kad šis sluoksnis praranda daug informacijos apie galimus požymius, nes šio sluoksnio išėjimas yra tik didžiausią tikimybę turintis požymis iš kiekvienos lango pozicijos. Tuo metu jo pasiūlytų kapsulinių sluoksnių viena iš paskirčių atitinka sujungimo sluoksnio paskirtį ir šios ydos neturi arba jos įtaka yra mažesnė.

Tad paskutinė šiame magistro baigiamajame darbe tiriama kapsulinio neuroninio tinklo modifikacija yra daugiavaizdis kapsulinis neuroninis tinklas, kuris vietoje vaizdų sujungimo sluoksnio naudoja modifikuotą kapsulinį sluoksnį. Šis sluoksnis vadinamas vaizdų kapsuliniu sluoksniu. Šiame sluoksnyje įėjimo vektorių rinkiniai, vadinami vaizdais, yra sugrupuojami į grupes su nurodytu tuo pačiu dydžiu. Tada toje pačioje grupėje esantys vaizdai yra sujungiami į vieną įėjimo vektorių rinkinį. Galiausiai tas rinkinys yra naudojamas kaip įvestis maršrutizavimo algoritmui.

Šio daugiavaizdžio kapsulinio neuroninio tinklo apmokymas taip pat gali būti padalintas į du etapus. Pirmame etape šis tinklas yra apmokomas be vaizdų kapsulinio sluoksnio. Antrame etape įterpiamas vaizdų kapsulinis sluoksnis ir apmokymas pratęsiamas.

Šiame magistro baigiamajame darbe ši modifikacija yra apmokoma įprastu ir dviejų etapų metodu. Įterpiamo vaizdų kapsulinio sluoksnio parametrai yra kapsulių skaičius - 40, išėjimų vektorių dimensijų skaičius - 32 ir maršrutizavimo iteracijų skaičius - 3. Šis sluoksnis yra įterpiamas tarp sluoksnių $c$ ir $ed$.



\section{Eksperimentiniai tyrimai}

\subsection{Tyrimams naudoti duomenys}
Šio magistro baigiamojo darbo tyrimams naudojami duomenys yra duomenys, kurie buvo naudoti darbų \cite{cnnExp1, cnnExp2} tyrimuose. Šių darbų tyrimuose naudoti duomenys yra laikomi repozitorijoje \cite{dataRepo}. Šioje repozitorijoje yra 12.79115 3D modelių iš 662 kategorijų. Tačiau darbų \cite{cnnExp1, cnnExp2} tyrimuose yra naudojamas tik šių duomenų poaibis, kuris buvo sudarytas tyrimams atliktiems darbe \cite{dbnExp}. Šiame poaibyje yra 12.311 3D modelių iš 40 kategorijų.

Kiekvienai kategorijai priklauso skirtingas skaičius modelių. Iš kiekvieno modelio darbe \cite{cnnExp2} yra sugeneruojama 12 2D nuotraukų. Visų nuotraukų kampai sudaro radialinę simetriją. Kitaip tariant, tarp dviejų kaimyninių pozicijų iš kurių buvo padaryta nuotrauka yra $30^{\circ}$ kampas iš objekto pozicijos. Visų nuotraukų pozicijos yra pakeltos $30^{\circ}$ kampu nuo horizontalės iš objekto pozicijos. Skirtingai negu darbo \cite{dbnExp} tyrimuose, darbo \cite{cnnExp2} tyrimuose nuotraukos yra generuojamos su juodu fonu ir kiekviena nuotrauka yra objektą apibrėžiantis stačiakampis (angl. bounding box). Kiekvienos nuotraukos matmenys yra $224\times224$.

Šiame magistro baigiamajame darbe duomenys yra padalinami į testavimo ir apmokymo duomenis taip pat kaip ir darbų \cite{cnnExp1, cnnExp2} atliktuose tyrimuose. Tuo metu, šių darbų tyrimuose duomenys yra padalinami taip pat kaip darbo \cite{dbnExp} tyrimuose naudojami duomenys. Tad apmokymo aibė šiame darbe susideda iš 9.843 modelių ir testavimo aibė - 2.468.


\subsection{Tyrimų aprašymai}

\subsubsection{Tyrimai su tikslumo matu}

Šiame magistro baigiamajame darbe yra palyginami 4 neuroninių tinklų architektūros: daugiavaizdis neuroninis tinklas, aprašytas poskyryje "1.4.3. Daugiavaizdis konvoliucinis neuroninis tinklas", kapsulinis neuroninis tinklas, aprašytas poskyryje "Tiriamo kapsulinio neuroninio tinklo architektūra", ir 2 daugiavaizdžiai kapsuliniai neuroniniai tinklai, aprašyti poskyriuose "Tiriamo daugiavaizdžio kapsulinio neuroninio tinklo architektūra su vaizdų sujungimo sluoksniu" ir "Tiriamo daugiavaizdžio kapsulinio neuroninio tinklo architektūra su vaizdų kapsuliniu sluoksniu". Kiekvienas tiriamas dirbtinis neuroninis tinklas yra apmokomas naudojantis visais duomenimis, aprašytais poskyryje "3.1. Tyrimams naudoti duomenys". Šie duomenys apmokymo metu yra padalinami į duomenų rinkinius, iš kurių kiekvienas yra sudarytas iš 96 2D nuotraukų. Daugiavaizdžio konvoliucinio ir kapsulinio neuroninių tinklų apmokymų antram etapui duomenų rinkiniai sudaryti iš nuotraukų grupių, kuriose yra visos konkretaus 3D objekto modelio nuotraukos. Kiekvienas dirbtinis neuroninis tinklas yra apmokomas per 10 epochų. Daugiavaizdžio konvoliucinio ir kapsulinio neuroninių tinklų abu apmokymo etapai yra apmokomi po 5 epochas.

Visų šiame magistro baigiamajame darbe tiriamų dirbtinių neuroninių tinklų apmokymai trunka po 6-7 valandas naudojantis Kaggle sistema. Šioje sistemoje vartotojui yra išskiriama viena Nvidia Tesla P100 vaizdo plokštė, kuri turi 3584 CUDA branduolių ir 16 GB RAM atminties. Taip pat šioje sistemoje vartotojui yra suteikiamas Intel(R) Xeon(R) CPU, kuris turi vieną branduolį, 39,424 MB spartinančiosios atminties ir kurio dažnis yra 2000,176 MHz. Galiausiai ši sistema vartotojui išskiria 16,4 GB RAM atminties.

Šiame magistro baigiamajame darbe bandoma optimizuoti kapsulinių neuroninių tinklų modifikacijų konfigūracijas. Pirmiausia bandoma optimizuoti šiuos tinklus naudojantis Bajeso hiperparametrų optimizavimo algoritmu, kuris yra aprašytas darbe \cite{bayes}. Toliau bandomos kitos konfigūracijos nei konfigūracijos aprašytos darbe \cite{capsNet}. Tačiau, dėl Kaggle sistemos apribojimų, nei vienas metodas neaptiko geresnių kapsulinių neuroninių tinklų modifikacijų konfigūracijų. Taip pat bandoma 
keisti mokymosi greitį. Tačiau skirtumai tarp rezultatų yra nereikšmingi.

Po kiekvienos epochos yra renkamos tikslumo metrikos: tikslumas klasifikuojant apmokymo duomenis, ši informacija pavaizduota \ref{tbl:train} lentelėje ir \ref{img:train_plot} paveikslėlyje, ir tikslumas klasifikuojant testavimo duomenis, ši informacija atvaizduota \ref{tbl:valid} lentelėje ir \ref{img:val_plot} paveikslėlyje. Tikslumas yra teisingai suklasifikuotų įrašų dalis klasifikuotų duomenų aibėje. \ref{tbl:train}, \ref{tbl:valid} lentelių stulpelio pavadinimas ir \ref{img:train_plot}, \ref{img:val_plot} paveikslėliuose pavaizduotų grafikų kreivių pavadinimas mvcnn yra daugiavaizdžio konvoliucinio neuroninio tinklo tikslumas, capsnet - kapsulinio neuroninio tinklo tikslumas, mv\_capsnet - daugiavaizdžio kapsulinio neuroninio tinklo su vaizdų sujungimo sluoksniu tikslumas, mv\_cap\_capsnet1 - daugiavaizdžio kapsulinio neuroninio tinklo su vaizdų kapsuliniu sluoksniu ir vienu mokymosi etapu tikslumas, mv\_cap\_capsnet2 - daugiavaizdžio kapsulinio neuroninio tinklo su vaizdų kapsuliniu sluoksniu ir dviem mokymosi etapais tikslumas. Brūkšninė vertikali linija \ref{img:train_plot} ir \ref{img:val_plot} paveikslėliuose pavaizduotuose grafikuose nurodo antrojo apmokymo etapo pirmąją epochą.

\begin{table}[]
\caption{
	Apmokymo duomenų klasifikavimo tikslumas, kur mvcnn yra daugiavaizdžio konvoliucinio neuroninio tinklo tikslumas, capsnet - kapsulinio neuroninio tinklo tikslumas, mv\_capsnet - daugiavaizdžio kapsulinio neuroninio tinklo su vaizdų sujungimo sluoksniu tikslumas, mv\_cap\_capsnet1 - daugiavaizdžio kapsulinio neuroninio tinklo su vaizdų kapsuliniu sluoksniu ir vienu mokymosi etapu tikslumas, mv\_cap\_capsnet2 - daugiavaizdžio kapsulinio neuroninio tinklo su vaizdų kapsuliniu sluoksniu ir dviem mokymosi etapais tikslumas. Kiekviename stulpelyje geriausi pasiekti tikslumai yra paryškinti.
}
\begin{tabular}{l|l|l|l|l|l}
	epocha &     mvcnn &   capsnet & mv\_capsnet & mv\_cap\_capsnet1 & mv\_cap\_capsnet2 \\ \hline
	1 & 0,688 &   0,119 &      0,517 &           0,281 &           0,218 \\
	2 & 0,860 &   0,529 &      0,793 &           0,807 &           0,624 \\
	3 & 0,904 &   0,660 &      0,840 &           0,880 &           0,731 \\
	4 & 0,929 &   0,720 &      0,871 &           0,911 &           0,785 \\
	5 & 0,944 &   0,758 &      0,895 &           0,933 &           0,821 \\
	6 & 0,937 &   0,783 &      0,878 &           0,948 &           0,717 \\
	7 & 0,950 &   0,805 &      0,929 &           0,958 &           0,911 \\
	8 & 0,959 &   0,821 &      0,952 &           0,965 &           0,944 \\
	9 & 0,961 &   0,835 &      0,966 &           0,970 &           0,962 \\
	10 & \textbf{0,972} &   \textbf{0,847} &      \textbf{0,972} &           \textbf{0,975} &           \textbf{0,970} \\
\end{tabular}
\label{tbl:train}
\end{table}

\begin{table}[]
\caption{
	Testavimo duomenų klasifikavimo tikslumas, kur mvcnn yra daugiavaizdžio konvoliucinio neuroninio tinklo tikslumas, capsnet - kapsulinio neuroninio tinklo tikslumas, mv\_capsnet - daugiavaizdžio kapsulinio neuroninio tinklo su vaizdų sujungimo sluoksniu tikslumas, mv\_cap\_capsnet1 - daugiavaizdžio kapsulinio neuroninio tinklo su vaizdų kapsuliniu sluoksniu ir vienu mokymosi etapu tikslumas, mv\_cap\_capsnet2 - daugiavaizdžio kapsulinio neuroninio tinklo su vaizdų kapsuliniu sluoksniu ir dviem mokymosi etapais tikslumas. Kiekviename stulpelyje geriausi pasiekti tikslumai yra paryškinti.
}
\begin{tabular}{l|l|l|l|l|l}
	epocha &     mvcnn &   capsnet & mv\_capsnet & mv\_cap\_capsnet1 & mv\_cap\_capsnet2 \\ \hline
	1 & 0,792 &   0,302 &      0,718 &           0,692 &           0,480 \\
	2 & 0,835 &   0,562 &      0,768 &           0,789 &           0,629 \\
	3 & 0,849 &   0,631 &      0,778 &           0,822 &           0,687 \\
	4 & 0,858 &   0,663 &      0,784 &           0,828 &           0,714 \\
	5 & 0,860 &   0,683 &      0,800 &           0,850 &           0,721 \\
	6 & 0,888 &   0,674 &      0,800 &           0,861 &           0,832 \\
	7 & 0,880 &   0,698 &      0,822 &           \textbf{0,864} &          \textbf{0,848} \\
	8 & 0,881 &   0,701 &      0,827 &           0,862 &           0,838 \\
	9 & 0,896 &   \textbf{0,723} &      0,808 &           0,861 &           0,841 \\
	10 & \textbf{0,901} &   0,710 &      \textbf{0,853} &           0,856 &           0,828 \\
\end{tabular}
\label{tbl:valid}
\end{table}

\begin{figure}[H]
	\centering
	\includegraphics[scale=0.5]{img/trained.png}
	\caption{
		Apmokymo duomenų klasifikavimo tikslumas, kur mvcnn yra daugiavaizdžio konvoliucinio neuroninio tinklo tikslumas, capsnet - kapsulinio neuroninio tinklo tikslumas, mv\_capsnet - daugiavaizdžio kapsulinio neuroninio tinklo su vaizdų sujungimo sluoksniu tikslumas, mv\_cap\_capsnet1 - daugiavaizdžio kapsulinio neuroninio tinklo su vaizdų kapsuliniu sluoksniu ir vienu mokymosi etapu tikslumas, mv\_cap\_capsnet2 - daugiavaizdžio kapsulinio neuroninio tinklo su vaizdų kapsuliniu sluoksniu ir dviem mokymosi etapais tikslumas. Brūkšninė vertikali linija nurodo antrojo apmokymo etapo pirmąją epochą.
	}
	\label{img:train_plot}
\end{figure}

\begin{figure}[H]
	\centering
	\includegraphics[scale=0.5]{img/validated.png}
	\caption{
		Testavimo duomenų klasifikavimo tikslumas, kur mvcnn yra daugiavaizdžio konvoliucinio neuroninio tinklo tikslumas, capsnet - kapsulinio neuroninio tinklo tikslumas, mv\_capsnet - daugiavaizdžio kapsulinio neuroninio tinklo su vaizdų sujungimo sluoksniu tikslumas, mv\_cap\_capsnet1 - daugiavaizdžio kapsulinio neuroninio tinklo su vaizdų kapsuliniu sluoksniu ir vienu mokymosi etapu tikslumas, mv\_cap\_capsnet2 - daugiavaizdžio kapsulinio neuroninio tinklo su vaizdų kapsuliniu sluoksniu ir dviem mokymosi etapais tikslumas. Brūkšninė vertikali linija nurodo antrojo apmokymo etapo pirmąją epochą.
	}
	\label{img:val_plot}
\end{figure}

\ref{img:train_plot} paveikslėlyje pavaizduotame grafike matomas visų dirbtinių neuroninių tinklų, kurie yra apmokomi dvejais etapais, apmokymo duomenų klasifikavimo tikslumo sumažėjimas prasidėjus antram etapui. Taip yra, nes apmokymo metu dirbtinis neuroninis tinklas yra keičiamas taip, kad jis pasiektų kuo tikslesnius rezultatus klasifikuojant apmokymo duomenis. Tačiau jo struktūros pakeitimas iškreipia jo rezultatus. Apmokymo tęsimas greitai pašalina šį iškreipimą.

Taip pat \ref{img:train_plot} paveikslėlyje pavaizduotame grafike matoma, kad daugiavaizdžio konvoliucinio neuroninio tinklo ir daugiavaizdžio kapsulinio neuroninio tinklo su vaizdų sujungimo sluoksniu tikslumas nukenčia žymiai mažiau nei daugiavaizdžio kapsulinio neuroninio tinklo su vaizdų kapsuliniu sluoksniu. Taip yra, nes vaizdų sujungimo sluoksnis neturi svorių. Tad jo pridėjimas po tinklo paskutinio sluoksnio tik pakoreguoja dirbtinio neuroninio tinklo rezultatą. Tačiau vaizdų kapsulinis sluoksnis turi svorius. Tad jį pridėjus jam yra parenkami pradiniai atsitiktiniai svoriai, kurie smarkiai iškreipia dirbtinio tinklo rezultatą.

Tuo metu iš \ref{img:val_plot} paveikslėlyje pavaizduoto grafiko yra matomas visų dirbtinių neuroninių tinklų, kurie yra apmokomi dvejais etapais, apmokymo duomenų klasifikavimo tikslumo padidėjimas prasidėjus antram etapui. Jei testavimo duomenys reprezentuoja populiacija, tai testavimo duomenų tikslumas parodo kaip gerai dirbtinis neuroninis tinklas klasifikuos naujus duomenis. Tikslumo padidėjimas klasifikuojant testavimo duomenis ir sumažėjimas  klasifikuojant apmokymo duomenis indikuoja, kad dirbtinio neuroninio tinklo pakeitimo nauda yra didesnė nei rezultatų iškreipimas.

Taip pat iš \ref{img:val_plot} paveikslėlyje pavaizduoto grafiko matoma, kad daugiavaizdžio kapsulinio neuroninio tinklo su vaizdų sujungimo sluoksniu tikslumas nepakinta, kai tuo metu su vaizdų kapsuliniu sluoksniu apmokant dvejais etapais gerokai padidėja. Tai indikuoja, kad vaizdų sujungimo sluoksnio pridėjimo prie daugiavaizdžio kapsulinio neuroninio tinklo nauda tik atsveria rezultatų iškreipimą. Tuo metu vaizdų kapsulinio sluoksnio pridėjimo prie daugiavaizdžio kapsulinio neuroninio tinklo nauda yra gerokai didesnė nei rezultatų iškreipimas.

\ref{img:train_plot} ir \ref{img:val_plot} paveikslėliuose pavaizduoti grafikai indikuoja, kad tiksliausius rezultatus pasiekia daugiavaizdis konvoliucinis neuroninis tinklas. Toliau -- daugiavaizdis kapsulinis neuroninis tinklas su vaizdų kapsuliniu sluoksniu ir vienu apmokymo etapu. Trečias -- daugiavaizdis kapsulinis neuroninis tinklas su vaizdų sujungimo sluoksniu. Ketvirtas -- daugiavaizdis kapsulinis neuroninis tinklas su vaizdų kapsuliniu sluoksniu ir dvejais apmokymo etapais. Mažiausiai tikslius rezultatus pasiekia kapsulinis neuroninis tinklas.

\subsubsubsection{Tyrimai su mažesne duomenų imtimi}

Darbe \cite{capsNet} yra teigiama, kad kapsuliniai neuroniniai tinklai reikalauja mažesnės apmokymo duomenų imties nei konvoliuciniai neuroniniai tinklai. Todėl šiame magistro baigiamajame darbe yra atliekami tyrimai su tyrimų duomenų poaibiais. Tiriami tik daugiavaizdis neuroninis tinklas ir daugiavaizdis kapsulinis neuroninis tinklas su vaizdų kapsuliniu sluoksniu ir vienu mokymosi etapu. Lentelėje \ref{tbl:less_datav1} ir grafike pavaizduotame paveikslėlyje \ref{img:less_datav1} pavaizduoti tyrimų rezultatai su apmokymo duomenimis, sudarytais iš 4904 3D objektų modelių, ir testavimo 1232. Lentelėje \ref{tbl:less_datav2} ir grafike pavaizduotame paveikslėlyje \ref{img:less_datav2} pavaizduoti tyrimų rezultatai su apmokymo duomenimis, sudarytais iš 3264 3D objektų modelių, ir testavimo 2464. Šių lentelių stulpelio pavadinimas ir grafikų kreivės pavadinimas mvcnn yra daugiavaizdžio neuroninio tinklo tikslumas klasifikuojant apmokymo duomenis, val\_mvcnn - daugiavaizdžio neuroninio tinklo tikslumas klasifikuojant testavimo duomenis, mv\_cap\_capsnet - daugiavaizdžio kapsulinio neuroninio tinklo su vaizdų kapsuliniu sluoksniu ir vienu mokymosi etapu tikslumas klasifikuojant apmokymo duomenis, val\_mv\_cap\_capsnet - daugiavaizdžio kapsulinio neuroninio tinklo su vaizdų kapsuliniu sluoksniu ir vienu mokymosi etapu tikslumas klasifikuojant testavimo duomenis. Brūkšninė vertikali linija grafikuose nurodo antrojo apmokymo etapo pirmąją epochą.

\begin{table}[]
	\caption{
		Tyrimų rezultatai su apmokymo duomenimis, sudarytais iš 4904 3D objektų modelių, ir testavimo 1232, kur mvcnn yra daugiavaizdžio konvoliucinio neuroninio tinklo tikslumas klasifikuojant apmokymo duomenis, val\_mvcnn - daugiavaizdžio neuroninio tinklo tikslumas klasifikuojant testavimo duomenis, mv\_cap\_capsnet - daugiavaizdžio kapsulinio neuroninio tinklo su vaizdų kapsuliniu sluoksniu ir vienu mokymosi etapu tikslumas klasifikuojant apmokymo duomenis, val\_mv\_cap\_capsnet - daugiavaizdžio kapsulinio neuroninio tinklo su vaizdų kapsuliniu sluoksniu ir vienu mokymosi etapu tikslumas klasifikuojant testavimo duomenis. Kiekviename stulpelyje geriausi pasiekti tikslumai yra paryškinti.
	}
	\begin{tabular}{l|l|l|l|l}
		epocha &     mvcnn & val\_mvcnn & mv\_cap\_capsnet & val\_mv\_cap\_capsnet \\ \hline
		 1 & 0,594 &     0,752 &          0,064 &              0,041 \\
		2 & 0,821 &     0,792 &          0,219 &              0,532 \\
		3 & 0,875 &     0,816 &          0,706 &              0,742 \\
		4 & 0,910 &     0,824 &          0,813 &              0,801 \\
		5 & 0,930 &     0,819 &          0,878 &              0,810 \\
		6 & 0,922 &     0,836 &          0,910 &              0,826 \\
		7 & 0,928 &     0,847 &          0,930 &              0,828 \\
		8 & 0,948 &     0,862 &          0,944 &              0,831 \\
		9 & 0,955 &     \textbf{0,868} &          0,956 &              0,829 \\
		10 & \textbf{0,961} &     0,847 &          \textbf{0,964} &              \textbf{0,832} \\
	\end{tabular}
	\label{tbl:less_datav1}
\end{table}


\begin{figure}[H]
	\centering
	\includegraphics[scale=0.5]{img/less_data_v1.png}
	\caption{
		Tyrimų rezultatai su apmokymo duomenimis, sudarytais iš 4904 3D objektų modelių, ir testavimo 1232, kur mvcnn yra daugiavaizdžio konvoliucinio neuroninio tinklo tikslumas klasifikuojant apmokymo duomenis, val\_mvcnn - daugiavaizdžio neuroninio tinklo tikslumas klasifikuojant testavimo duomenis, mv\_cap\_capsnet - daugiavaizdžio kapsulinio neuroninio tinklo su vaizdų kapsuliniu sluoksniu ir vienu mokymosi etapu tikslumas klasifikuojant apmokymo duomenis, val\_mv\_cap\_capsnet - daugiavaizdžio kapsulinio neuroninio tinklo su vaizdų kapsuliniu sluoksniu ir vienu mokymosi etapu tikslumas klasifikuojant testavimo duomenis. Brūkšninė vertikali linija grafikuose nurodo antrojo apmokymo etapo pirmąją epochą.
	}
	\label{img:less_datav1}
\end{figure}

\begin{table}[]
	\caption{
		Tyrimų rezultatai su apmokymo duomenimis, sudarytais iš 3264 3D objektų modelių, ir testavimo 2464, kur mvcnn yra daugiavaizdžio konvoliucinio neuroninio tinklo tikslumas klasifikuojant apmokymo duomenis, val\_mvcnn - daugiavaizdžio neuroninio tinklo tikslumas klasifikuojant testavimo duomenis, mv\_cap\_capsnet - daugiavaizdžio kapsulinio neuroninio tinklo su vaizdų kapsuliniu sluoksniu ir vienu mokymosi etapu tikslumas klasifikuojant apmokymo duomenis, val\_mv\_cap\_capsnet - daugiavaizdžio kapsulinio neuroninio tinklo su vaizdų kapsuliniu sluoksniu ir vienu mokymosi etapu tikslumas klasifikuojant testavimo duomenis. Kiekviename stulpelyje geriausi pasiekti tikslumai yra paryškinti.
	}
	\begin{tabular}{l|l|l|l|l}
		epocha &     mvcnn & val\_mvcnn & mv\_cap\_capsnet & val\_mv\_cap\_capsnet \\ \hline
		1 & 0,511 &     0,706 &          0,065 &              0,041 \\
		2 & 0,775 &     0,767 &          0,380 &              0,571 \\
		3 & 0,850 &     0,798 &          0,727 &              0,696 \\
		4 & 0,889 &     0,805 &          0,817 &              0,739 \\
		5 & 0,916 &     0,813 &          0,874 &              0,762 \\
		6 & 0,910 &     0,832 &          0,907 &              0,772 \\
		7 & 0,915 &     0,808 &          0,929 &              0,781 \\
		8 & 0,936 &     0,822 &          0,946 &              0,807 \\
		9 & 0,943 &     0,851 &          0,957 &              0,808 \\
		10 & \textbf{0,951} &     \textbf{0,853} &         \textbf{0,965} &              \textbf{0,816} \\
	\end{tabular}
	\label{tbl:less_datav2}
\end{table}


\begin{figure}[H]
	\centering
	\includegraphics[scale=0.5]{img/less_data_v2.png}
	\caption{
		Tyrimų rezultatai su apmokymo duomenimis, sudarytais iš 3264 3D objektų modelių, ir testavimo 2464, kur mvcnn yra daugiavaizdžio konvoliucinio neuroninio tinklo tikslumas klasifikuojant apmokymo duomenis, val\_mvcnn - daugiavaizdžio konvoliucinio neuroninio tinklo tikslumas klasifikuojant testavimo duomenis, mv\_cap\_capsnet - daugiavaizdžio kapsulinio neuroninio tinklo su vaizdų kapsuliniu sluoksniu ir vienu mokymosi etapu tikslumas klasifikuojant apmokymo duomenis, val\_mv\_cap\_capsnet - daugiavaizdžio kapsulinio neuroninio tinklo su vaizdų kapsuliniu sluoksniu ir vienu mokymosi etapu tikslumas klasifikuojant testavimo duomenis.	Brūkšninė vertikali linija grafikuose nurodo antrojo apmokymo etapo pirmąją epochą.
	}
	\label{img:less_datav2}
\end{figure}

\ref{img:less_datav1} ir \ref{img:less_datav2} paveikslėliuose pavaizduoti grafikai indikuoja, kad su mažesne duomenų imtimi skirtumas tarp daugiavaizdžio konvoliucinio neuroninio tinklo klasifikavimo tikslumo ir daugiavaizdžio kapsulinio neuroninio tinklo su vaizdų kapsuliniu sluoksniu ir vienu mokymosi etapu klasifikavimo tikslumo tampa nereikšmingas. Tačiau, abu grafikai indikuoja, kad  daugiavaizdis konvoliucinis neuroninis tinklas pasiekia tikslensiu klasifikavimo rezultatus ir jo apmokymas reikalauja mačiau epochų.


\subsubsection{Tyrimai su f1 įverčio matu}
Buvo nustatyta, kad testavimo duomenyse egzistuoja klasių išsibalansavimas, kuris matosi iš \ref{tbl:class_imbalance} lentelės. Tad tikslumo matas gali nekorektiškai įvertinti dirbtinių neuroninių tinklų rezultatus. Todėl vietoje tikslumo mato tolimesniuose tyrimuose yra renkamas mikro, makro ir svertinis F1 įverčio (angl. F1 score) matai, kurie geriau įvertina dirbtinius neuroninius tinklus esant klasių išsibalansavimui nei tikslumo matas. F1 įvertis yra apskaičiuojamas kiekvienai klasei atskirai naudojant formulę (\ref{eqn:f1}), kur:

\begin{itemize}
	\item $TP$ - teisingų teigiamų (angl. true positives) kiekis. Teisingi teigiami yra 3D objektai, kuriems dirbtinis neuroninis tinklas teisingai priskyrė klasę, kuriai yra skaičiuojamas F1.
	\item $FP$ - klaidingų teigiamų (angl. false positives) kiekis. Klaidingi teigiami yra 3D objektai, kuriems dirbtinis neuroninis tinklas klaidingai priskyrė klasę, kuriai yra skaičiuojamas F1.
	\item $FN$ - klaidingų neigiamų (angl. false negatives) kiekis. Klaidingi neigiami yra 3D objektai, kuriems dirbtinis neuroninis tinklas klaidingai nepriskyrė klasės, kuriai yra skaičiuojamas F1.
\end{itemize}

Jų vidurkis yra lygus makro F1 įverčiui. Toliau yra apskaičiuojamas svertinis vidurkis (angl. weighted average), kur svoriai yra 3D objektų su priskirta klase, kuriai yra apskaičiuotas F1, dalis testavimo duomenyse. Šis vidurkis yra lygus svertiniui F1 įverčiui. Galiausiai yra apskaičiuojamas mikro F1 įvertis. Mikro F1 įvertis yra apskaičiuojamas naudojant formulę \ref{eqn:f1}, kur $TP$ yra teisingų teigiamų suma, $FP$ - klaidingų teigiamų suma ir $FN$ - klaidingų neigiamų suma. F1 įverčių matai yra skaičiuojami tik testavimo duomenims.

\begin{equation}
\label{eqn:f1}
	F1 = \dfrac{TP}{TP + \dfrac{1}{2}(FP + FN)}
\end{equation}

\begin{table}[]
\caption{3D objektų skaičius kiekvienai klasei testavimo duomenyse}
\begin{tabular}{lr}
	klasė       &   3D objektų skaičius \\
	\hline
	airplane    &                    99 \\
	bathtub     &                    50 \\
	bed         &                   100 \\
	bench       &                    20 \\
	bookshelf   &                   100 \\
	bottle      &                   100 \\
	bowl        &                    20 \\
	car         &                    99 \\
	chair       &                    99 \\
	cone        &                    20 \\
	cup         &                    20 \\
	curtain     &                    20 \\
	desk        &                    86 \\
	door        &                    20 \\
	dresser     &                    86 \\
	flower\_pot  &                    20 \\
	glass\_box   &                   100 \\
	guitar      &                    99 \\
	keyboard    &                    20 \\
	lamp        &                    20 \\
	laptop      &                    20 \\
	mantel      &                   100 \\
	monitor     &                   100 \\
	night\_stand &                    86 \\
	person      &                    20 \\
	piano       &                   100 \\
	plant       &                   100 \\
	radio       &                    20 \\
	range\_hood  &                   100 \\
	sink        &                    20 \\
	sofa        &                   100 \\
	stairs      &                    20 \\
	stool       &                    20 \\
	table       &                   100 \\
	tent        &                    20 \\
	toilet      &                   100 \\
	tv\_stand    &                   100 \\
	vase        &                   100 \\
	wardrobe    &                    20 \\
	xbox        &                    20 \\
\end{tabular}
\label{tbl:class_imbalance}
\end{table}


Taip pat pastebėta iš \ref{img:val_plot} grafiko, kad kapsulinio ir daugiavaizdžių kapsulinių neuroninių tinklų tikslumas netampa stabilus po 10 epochų. Tačiau, dėl Kaggle sistemos kodo apdorojimo laiko limito, kuris yra 9 valandos, didesnis epochų skaičius nei 12 yra negalimas, nes 1 tyrimas su 12 epochų trunka apytiksliai 8,5 valandos. Tad visi tolimesni tyrimai yra  atlikti su 12 apmokymo epochų.

Tyrimų, kuriuose buvo renkamos f1 įverčio metrikos ir naudojama 12 apmokymo epochos, rezultatai yra atvaizduoti: lentelėje \ref{tbl:weighted_f1} ir grafike \ref{img:weighted_f1} - testavimo duomenų svertiniai f1 įverčiai apmokant su visais apmokymo duomenimis, lentelėje \ref{tbl:micro_f1} ir grafike \ref{img:micro_f1} - testavimo duomenų mikro f1 įverčiai apmokant su visais apmokymo duomenimis, lentelėje \ref{tbl:macro_f1} ir grafike \ref{img:macro_f1} - testavimo duomenų makro f1 įverčiai su apmokant visais apmokymo duomenimis, lentelėje \ref{tbl:half_sample_f1} ir grafike \ref{img:half_sample_f1} - testavimo duomenų f1 įverčiai apmokant su apmokymo duomenimis sudarytais iš 4904 3D objektų modelių, lentelėje \ref{tbl:3rd_sample_f1} ir grafike \ref{img:3rd_sample_f1} - testavimo duomenų f1 įverčiai apmokant su apmokymo duomenimis sudarytais iš 3264 3D objektų modelių, kur
mvcnn yra daugiavaizdžio neuroninio tinklo f1 įverčiai, capsnet - kapsulinio neuroninio tinklo f1 įverčiai, mv\_capsnet - daugiavaizdžio kapsulinio neuroninio tinklo su vaizdų sujungimo sluoksniu f1 įverčiai, mv\_cap\_capsnet1 - daugiavaizdžio kapsulinio neuroninio tinklo su vaizdų kapsuliniu sluoksniu ir vienu mokymosi etapu f1 įverčiai, mv\_cap\_capsnet2 - daugiavaizdžio kapsulinio neuroninio tinklo su vaizdų kapsuliniu sluoksniu ir dviem mokymosi etapais f1 įverčiai, 
mvcnn\_weighted -  daugiavaizdžio neuroninio tinklo svertiniai f1 įverčiai, 
mvcnn\_micro -  daugiavaizdžio neuroninio tinklo mikro f1 įverčiai, 
mvcnn\_macro -  daugiavaizdžio neuroninio tinklo makro f1 įverčiai, 
mv\_cap\_capsnet\_1\_weighted - daugiavaizdžio kapsulinio neuroninio tinklo su vaizdų kapsuliniu sluoksniu ir vienu mokymosi etapu svertiniai f1 įverčiai, 
mv\_cap\_capsnet\_1\_micro - daugiavaizdžio kapsulinio neuroninio tinklo su vaizdų kapsuliniu sluoksniu ir vienu mokymosi etapu mikro f1 įverčiai ir
mv\_cap\_capsnet\_1\_macro - daugiavaizdžio kapsulinio neuroninio tinklo su vaizdų kapsuliniu sluoksniu ir vienu mokymosi etapu makro f1 įverčiai. 
Brūkšninė vertikali linija grafikuose nurodo antrojo apmokymo etapo pirmąją epochą. Kiekviename stulpelyje geriausi pasiekti tikslumai yra paryškinti.

\begin{table}[]
	\caption{
		Testavimo duomenų klasifikavimo svertiniai f1 įverčiai, kur mvcnn yra daugiavaizdžio neuroninio tinklo svertiniai f1 įverčiai, capsnet - kapsulinio neuroninio tinklo svertiniai f1 įverčiai, mv\_capsnet - daugiavaizdžio kapsulinio neuroninio tinklo su vaizdų sujungimo sluoksniu svertiniai f1 įverčiai, mv\_cap\_capsnet1 - daugiavaizdžio kapsulinio neuroninio tinklo su vaizdų kapsuliniu sluoksniu ir vienu mokymosi etapu svertiniai f1 įverčiai, mv\_cap\_capsnet2 - daugiavaizdžio kapsulinio neuroninio tinklo su vaizdų kapsuliniu sluoksniu ir dviem mokymosi etapais svertiniai f1 įverčiai. Kiekviename stulpelyje geriausi pasiekti tikslumai yra paryškinti.
	}
	\begin{tabular}{l|l|l|l|l|l}
		epocha & mvcnn & capsnet & mv-capsnet & mv-cap-capsnet1 & mv-cap-capsnet2 \\
		\hline
		1 & 0,803 &   0,600 &      0,620 &           0,003 &           0,662 \\
		2 & 0,834 &   0,740 &      0,757 &           0,756 &           0,764 \\
		3 & 0,847 &   0,781 &      0,784 &           0,808 &           0,793 \\
		4 & 0,855 &   0,794 &      0,805 &           0,827 &           0,807 \\
		5 & 0,859 &   0,800 &      0,813 &           0,835 &           0,810 \\
		6 & 0,858 &   0,810 &      0,814 &           0,841 &           0,814 \\
		7 & 0,875 &   0,822 &      0,813 &           0,846 &           0,855 \\
		8 & 0,888 &   0,826 &      0,806 &           0,846 &           0,871 \\
		9 & 0,873 &   0,823 &      0,811 &           0,853 &           0,860 \\
		10 & 0,897 &   0,824 &      0,824 &           0,853 &           0,861 \\
		11 & 0,897 &   0,827 &      0,818 &           \textbf{0,854} &           0,880 \\
		12 & \textbf{0,903} &  \textbf{ 0,834} &      0,822 &           0,850 &           \textbf{0,881} \\
	\end{tabular}
	\label{tbl:weighted_f1}
\end{table}

\begin{table}[]
	\caption{
		Testavimo duomenų klasifikavimo mikro f1 įverčiai, kur mvcnn yra daugiavaizdžio neuroninio tinklo mikro f1 įverčiai, capsnet - kapsulinio neuroninio tinklo mikro f1 įverčiai, mv\_capsnet - daugiavaizdžio kapsulinio neuroninio tinklo su vaizdų sujungimo sluoksniu mikro f1 įverčiai, mv\_cap\_capsnet1 - daugiavaizdžio kapsulinio neuroninio tinklo su vaizdų kapsuliniu sluoksniu ir vienu mokymosi etapu mikro f1 įverčiai, mv\_cap\_capsnet2 - daugiavaizdžio kapsulinio neuroninio tinklo su vaizdų kapsuliniu sluoksniu ir dviem mokymosi etapais mikro f1 įverčiai. Kiekviename stulpelyje geriausi pasiekti tikslumai yra paryškinti.
	}
	\begin{tabular}{l|l|l|l|l|l}
		epocha & mvcnn & capsnet & mv-capsnet & mv-cap-capsnet1 & mv-cap-capsnet2 \\
		\hline
		1 & 0,806 &   0,617 &      0,640 &           0,040 &           0,677 \\
		2 & 0,833 &   0,745 &      0,763 &           0,764 &           0,772 \\
		3 & 0,845 &   0,787 &      0,787 &           0,810 &           0,796 \\
		4 & 0,852 &   0,801 &      0,809 &           0,827 &           0,809 \\
		5 & 0,859 &   0,801 &      0,814 &           0,835 &           0,811 \\
		6 & 0,856 &   0,813 &      0,815 &           0,841 &           0,817 \\
		7 & 0,874 &   0,825 &      0,814 &           0,845 &           0,857 \\
		8 & 0,888 &   0,827 &      0,808 &           0,847 &           0,874 \\
		9 & 0,874 &   0,823 &      0,815 &           0,853 &           0,862 \\
		10 & 0,898 &   0,824 &      \textbf{0,825} &           0,853 &           0,861 \\
		11 & 0,898 &   0,829 &      0,816 &           \textbf{0,855} &           \textbf{0,881} \\
		12 & \textbf{0,903} &   \textbf{0,836} &      0,819 &           0,849 &           0,880 \\
	\end{tabular}
	\label{tbl:micro_f1}
\end{table}

\begin{table}[]
	\caption{
		Testavimo duomenų klasifikavimo makro f1 įverčiai, kur mvcnn yra daugiavaizdžio neuroninio tinklo makro f1 įverčiai, capsnet - kapsulinio neuroninio tinklo makro f1 įverčiai, mv\_capsnet - daugiavaizdžio kapsulinio neuroninio tinklo su vaizdų sujungimo sluoksniu makro f1 įverčiai, mv\_cap\_capsnet1 - daugiavaizdžio kapsulinio neuroninio tinklo su vaizdų kapsuliniu sluoksniu ir vienu mokymosi etapu makro f1 įverčiai, mv\_cap\_capsnet2 - daugiavaizdžio kapsulinio neuroninio tinklo su vaizdų kapsuliniu sluoksniu ir dviem mokymosi etapais makro f1 įverčiai. Kiekviename stulpelyje geriausi pasiekti tikslumai yra paryškinti.
	}
	\begin{tabular}{l|l|l|l|l|l}
		epocha & mvcnn & capsnet & mv-capsnet & mv-cap-capsnet1 & mv-cap-capsnet2 \\
		\hline
		1 & 0,740 &   0,518 &      0,549 &           0,002 &           0,587 \\
		2 & 0,785 &   0,675 &      0,689 &           0,654 &           0,706 \\
		3 & 0,801 &   0,727 &      0,729 &           0,722 &           0,733 \\
		4 & 0,816 &   0,739 &      0,751 &           0,751 &           0,754 \\
		5 & 0,819 &   0,752 &      0,763 &           0,761 &           0,761 \\
		6 & 0,820 &   0,762 &      0,765 &           0,777 &           0,764 \\
		7 & 0,841 &   0,774 &      0,759 &           0,781 &           0,782 \\
		8 & 0,853 &   0,780 &      0,750 &           0,783 &           0,817 \\
		9 & 0,831 &   0,777 &      0,760 &           0,798 &           0,804 \\
		10 & 0,864 &   0,776 &      \textbf{0,778} &           0,791 &           0,814 \\
		11 & 0,860 &   0,783 &      0,774 &           \textbf{0,799} &           \textbf{0,827} \\
		12 & \textbf{0,865} &   \textbf{0,790} &      0,772 &           0,794 &           0,824 \\
	\end{tabular}
	\label{tbl:macro_f1}
\end{table}

% -----------------------------------------------------------------------------------------------------------------------

\begin{figure}[H]
	\centering
	\includegraphics[scale=0.4]{img/weighted.png}
	\caption{
		Testavimo duomenų klasifikavimo svertiniai f1 įverčiai, kur mvcnn yra daugiavaizdžio neuroninio tinklo svertiniai f1 įverčiai, capsnet - kapsulinio neuroninio tinklo svertiniai f1 įverčiai, mv\_capsnet - daugiavaizdžio kapsulinio neuroninio tinklo su vaizdų sujungimo sluoksniu svertiniai f1 įverčiai, mv\_cap\_capsnet1 - daugiavaizdžio kapsulinio neuroninio tinklo su vaizdų kapsuliniu sluoksniu ir vienu mokymosi etapu svertiniai f1 įverčiai, mv\_cap\_capsnet2 - daugiavaizdžio kapsulinio neuroninio tinklo su vaizdų kapsuliniu sluoksniu ir dviem mokymosi etapais svertiniai f1 įverčiai. Brūkšninė vertikali linija nurodo antrojo apmokymo etapo pirmąją epochą.
	}
	\label{img:weighted_f1}
\end{figure}

\begin{figure}[H]
	\centering
	\includegraphics[scale=0.4]{img/micro.png}
	\caption{
		Testavimo duomenų klasifikavimo mikro f1 įverčiai, kur mvcnn yra daugiavaizdžio neuroninio tinklo mikro f1 įverčiai, capsnet - kapsulinio neuroninio tinklo mikro f1 įverčiai, mv\_capsnet - daugiavaizdžio kapsulinio neuroninio tinklo su vaizdų sujungimo sluoksniu mikro f1 įverčiai, mv\_cap\_capsnet1 - daugiavaizdžio kapsulinio neuroninio tinklo su vaizdų kapsuliniu sluoksniu ir vienu mokymosi etapu mikro f1 įverčiai, mv\_cap\_capsnet2 - daugiavaizdžio kapsulinio neuroninio tinklo su vaizdų kapsuliniu sluoksniu ir dviem mokymosi etapais mikro f1 įverčiai. Brūkšninė vertikali linija nurodo antrojo apmokymo etapo pirmąją epochą.
	}
	\label{img:micro_f1}
\end{figure}

\begin{figure}[H]
	\centering
	\includegraphics[scale=0.4]{img/macro.png}
	\caption{
		Testavimo duomenų klasifikavimo makro f1 įverčiai, kur mvcnn yra daugiavaizdžio neuroninio tinklo makro f1 įverčiai, capsnet - kapsulinio neuroninio tinklo makro f1 įverčiai, mv\_capsnet - daugiavaizdžio kapsulinio neuroninio tinklo su vaizdų sujungimo sluoksniu makro f1 įverčiai, mv\_cap\_capsnet1 - daugiavaizdžio kapsulinio neuroninio tinklo su vaizdų kapsuliniu sluoksniu ir vienu mokymosi etapu makro f1 įverčiai, mv\_cap\_capsnet2 - daugiavaizdžio kapsulinio neuroninio tinklo su vaizdų kapsuliniu sluoksniu ir dviem mokymosi etapais makro f1 įverčiai. Brūkšninė vertikali linija nurodo antrojo apmokymo etapo pirmąją epochą.
	}
	\label{img:macro_f1}
\end{figure}

\begin{table}[]
	\caption{
		Tyrimų rezultatai su apmokymo duomenimis, sudarytais iš 4904 3D objektų modelių, kur mvcnn\_weighted -  daugiavaizdžio neuroninio tinklo svertiniai f1 įverčiai, 
		mvcnn\_micro -  daugiavaizdžio neuroninio tinklo mikro f1 įverčiai, 
		mvcnn\_macro -  daugiavaizdžio neuroninio tinklo makro f1 įverčiai, 
		mv\_cap\_capsnet\_1\_weighted - daugiavaizdžio kapsulinio neuroninio tinklo su vaizdų kapsuliniu sluoksniu ir vienu mokymosi etapu svertiniai f1 įverčiai, 
		mv\_cap\_capsnet\_1\_micro - daugiavaizdžio kapsulinio neuroninio tinklo su vaizdų kapsuliniu sluoksniu ir vienu mokymosi etapu mikro f1 įverčiai ir
		mv\_cap\_capsnet\_1\_macro - daugiavaizdžio kapsulinio neuroninio tinklo su vaizdų kapsuliniu sluoksniu ir vienu mokymosi etapu makro f1 įverčiai. Kiekviename stulpelyje geriausi pasiekti tikslumai yra paryškinti.
	}
	\begin{tabular}{l|l|l|l|l|l|l}
		epocha & mvcnn\_weighted & mvcnn\_micro & mvcnn\_macro & \begin{tabular}[c]{@{}l@{}}mv\_cap\_1\\capsnet\\weighted\end{tabular} & \begin{tabular}[c]{@{}l@{}}mv\_cap\_1\\capsnet\\micro\end{tabular} & \begin{tabular}[c]{@{}l@{}}mv\_cap\_1\\capsnet\\macro\end{tabular} \\
		\hline
		1 &          0,723 &       0,731 &       0,661 &                     0,003 &                  0,026 &                  0,002 \\
		2 &          0,792 &       0,794 &       0,736 &                     0,598 &                  0,616 &                  0,488 \\
		3 &          0,810 &       0,805 &       0,755 &                     0,732 &                  0,743 &                  0,644 \\
		4 &          0,824 &       0,823 &       0,775 &                     0,781 &                  0,789 &                  0,700 \\
		5 &          0,828 &       0,828 &       0,782 &                     0,798 &                  0,804 &                  0,721 \\
		6 &          0,831 &       0,831 &       0,786 &                     0,800 &                  0,806 &                  0,720 \\
		7 &          0,845 &       0,845 &       0,810 &                     0,820 &                  0,825 &                  0,752 \\
		8 &          0,854 &       0,853 &       0,815 &                     0,824 &                  \textbf{0,828} &                  0,753 \\
		9 &          0,870 &       0,868 &       0,828 &                     \textbf{0,825} &                  0,827 &                  0,755 \\
		10 &          0,860 &       0,863 &       0,829 &                     0,817 &                  0,819 &                  0,748 \\
		11 &          \textbf{0,876} &       \textbf{0,875} &       0,833 &                     0,823 &                  0,823 &                  \textbf{0,756} \\
		12 &          0,872 &       0,871 &       \textbf{0,838} &                     0,821 &                  0,822 &                  0,754 \\
	\end{tabular}
	\label{tbl:half_sample_f1}
\end{table}

\begin{figure}[H]
	\centering
	\includegraphics[scale=0.4]{img/half_sample_f1.png}
	\caption{
		Tyrimų rezultatai su apmokymo duomenimis, sudarytais iš 4904 3D objektų modelių, kur mvcnn\_weighted -  daugiavaizdžio neuroninio tinklo svertiniai f1 įverčiai, 
		mvcnn\_micro -  daugiavaizdžio neuroninio tinklo mikro f1 įverčiai, 
		mvcnn\_macro -  daugiavaizdžio neuroninio tinklo makro f1 įverčiai, 
		mv\_cap\_capsnet\_1\_weighted - daugiavaizdžio kapsulinio neuroninio tinklo su vaizdų kapsuliniu sluoksniu ir vienu mokymosi etapu svertiniai f1 įverčiai, 
		mv\_cap\_capsnet\_1\_micro - daugiavaizdžio kapsulinio neuroninio tinklo su vaizdų kapsuliniu sluoksniu ir vienu mokymosi etapu mikro f1 įverčiai ir
		mv\_cap\_capsnet\_1\_macro - daugiavaizdžio kapsulinio neuroninio tinklo su vaizdų kapsuliniu sluoksniu ir vienu mokymosi etapu makro f1 įverčiai. Brūkšninė vertikali linija grafikuose nurodo antrojo apmokymo etapo pirmąją epochą.
	}
	\label{img:half_sample_f1}
\end{figure}

% -----------------------------------------------------------------------------------------------------------------------

\begin{table}[]
	\caption{
		Tyrimų rezultatai su apmokymo duomenimis, sudarytais iš 3264 3D objektų modelių, kur mvcnn\_weighted -  daugiavaizdžio neuroninio tinklo svertiniai f1 įverčiai, 
		mvcnn\_micro -  daugiavaizdžio neuroninio tinklo mikro f1 įverčiai, 
		mvcnn\_macro -  daugiavaizdžio neuroninio tinklo makro f1 įverčiai, 
		mv\_cap\_capsnet\_1\_weighted - daugiavaizdžio kapsulinio neuroninio tinklo su vaizdų kapsuliniu sluoksniu ir vienu mokymosi etapu svertiniai f1 įverčiai, 
		mv\_cap\_capsnet\_1\_micro - daugiavaizdžio kapsulinio neuroninio tinklo su vaizdų kapsuliniu sluoksniu ir vienu mokymosi etapu mikro f1 įverčiai ir
		mv\_cap\_capsnet\_1\_macro - daugiavaizdžio kapsulinio neuroninio tinklo su vaizdų kapsuliniu sluoksniu ir vienu mokymosi etapu makro f1 įverčiai. Kiekviename stulpelyje geriausi pasiekti tikslumai yra paryškinti.
	}
	\begin{tabular}{l|l|l|l|l|l|l}
		epocha & mvcnn\_weighted & mvcnn\_micro & mvcnn\_macro & \begin{tabular}[c]{@{}l@{}}mv\_cap\_1\\capsnet\\weighted\end{tabular} & \begin{tabular}[c]{@{}l@{}}mv\_cap\_1\\capsnet\\micro\end{tabular} & \begin{tabular}[c]{@{}l@{}}mv\_cap\_1\\capsnet\\macro\end{tabular} \\
		\hline
		1 &          0,690 &       0,695 &       0,616 &                     0,003 &                  0,040 &                  0,002 \\
		2 &          0,775 &       0,778 &       0,711 &                     0,001 &                  0,013 &                  0,001 \\
		3 &          0,798 &       0,798 &       0,737 &                     0,602 &                  0,623 &                  0,496 \\
		4 &          0,797 &       0,795 &       0,745 &                     0,725 &                  0,732 &                  0,636 \\
		5 &          0,814 &       0,814 &       0,759 &                     0,767 &                  0,772 &                  0,691 \\
		6 &          0,813 &       0,814 &       0,755 &                     0,784 &                  0,787 &                  0,704 \\
		7 &          0,819 &       0,821 &       0,766 &                     0,780 &                  0,784 &                  0,702 \\
		8 &          0,815 &       0,815 &       0,754 &                     0,784 &                  0,787 &                  0,706 \\
		9 &          0,830 &       0,831 &       0,789 &                     0,788 &                  0,791 &                  0,716 \\
		10 &          0,848 &       0,846 &       0,806 &                     0,800 &                  0,803 &                  0,731 \\
		11 &          0,834 &       0,836 &       0,778 &                     0,801 &                  0,804 &                  0,734 \\
		12 &          \textbf{0,849} &       \textbf{0,849} &       \textbf{0,807} &                     \textbf{0,802} &                  \textbf{0,806} &                  \textbf{0,734} \\
	\end{tabular}
	\label{tbl:3rd_sample_f1}
\end{table}

\begin{figure}[H]
	\centering
	\includegraphics[scale=0.4]{img/3rd_sample_f1.png}
	\caption{
		Tyrimų rezultatai su apmokymo duomenimis, sudarytais iš 3264 3D objektų modelių, kur mvcnn\_weighted -  daugiavaizdžio neuroninio tinklo svertiniai f1 įverčiai, 
		mvcnn\_micro -  daugiavaizdžio neuroninio tinklo mikro f1 įverčiai, 
		mvcnn\_macro -  daugiavaizdžio neuroninio tinklo makro f1 įverčiai, 
		mv\_cap\_capsnet\_1\_weighted - daugiavaizdžio kapsulinio neuroninio tinklo su vaizdų kapsuliniu sluoksniu ir vienu mokymosi etapu svertiniai f1 įverčiai, 
		mv\_cap\_capsnet\_1\_micro - daugiavaizdžio kapsulinio neuroninio tinklo su vaizdų kapsuliniu sluoksniu ir vienu mokymosi etapu mikro f1 įverčiai ir
		mv\_cap\_capsnet\_1\_macro - daugiavaizdžio kapsulinio neuroninio tinklo su vaizdų kapsuliniu sluoksniu ir vienu mokymosi etapu makro f1 įverčiai. Brūkšninė vertikali linija grafikuose nurodo antrojo apmokymo etapo pirmąją epochą.
	}
	\label{img:3rd_sample_f1}
\end{figure}

% -----------------------------------------------------------------------------------------------------------------------


\subsubsection{Kiekybiniai tyrimai su f1 įverčio matu}
Dirbtiniai neuroniniai tinklai yra apmokomi naudojantis stochastiniu metodu. Tad kiekvieną kartą apmokant tą patį dirbtinį neuroninį tinklą naudojant tuos pačius hiperparametrus, rezultatai skiriasi. Tad kuo daugiau kartų yra pakartotas tas apts tyrimas, tuo tiksliau yra indikuojamas palyginimo rezultatas. Šiame magistro baigiamame darbe yra 10 kartų kartojami tyrimai renkant f1 įverčius su pilnomis apmokymo ir testavimo duomenų imtimis. Šiuose tyrimuose nebuvo tiriamas daugiavaizdis kapsulinis neuroninis tinklas su vaizdų sujungimo sluoksniu, nes praeiti tyrimai neindikavo reikšmingumo. Šiame kiekybiniame tyrime buvo renkami testavimo duomenų f1 įverčiai po kiekvieno pilno apmokymo.


Šio kiekybinio tyrimo rezultatai yra atvaizduoti blokinėse diagramose (angl. box plot), kurios atvaizduotos paveikslėlyje \ref{img:box_weighted_f1}, paveikslėlyje \ref{img:box_micro_f1} ir  paveikslėlyje \ref{img:box_macro_f1}, kur
mvcnn yra daugiavaizdžio neuroninio tinklo f1 įverčiai, capsnet - kapsulinio neuroninio tinklo f1 įverčiai, mv\_cap\_capsnet1 - daugiavaizdžio kapsulinio neuroninio tinklo su vaizdų kapsuliniu sluoksniu ir vienu mokymosi etapu f1 įverčiai, mv\_cap\_capsnet2 - daugiavaizdžio kapsulinio neuroninio tinklo su vaizdų kapsuliniu sluoksniu ir dviem mokymosi etapais f1 įverčiai.

\begin{figure}[H]
	\centering
	\includegraphics[scale=0.4]{img/boxplot_f1_weighted.png}
	\caption{
		Testavimo duomenų klasifikavimo svertiniai f1 įverčiai, kur mvcnn yra daugiavaizdžio neuroninio tinklo f1 įverčiai, capsnet - kapsulinio neuroninio tinklo f1 įverčiai, mv\_cap\_capsnet1 - daugiavaizdžio kapsulinio neuroninio tinklo su vaizdų kapsuliniu sluoksniu ir vienu mokymosi etapu f1 įverčiai, mv\_cap\_capsnet2 - daugiavaizdžio kapsulinio neuroninio tinklo su vaizdų kapsuliniu sluoksniu ir dviem mokymosi etapais f1 įverčiai.
	}
	\label{img:box_weighted_f1}
\end{figure}

\begin{figure}[H]
	\centering
	\includegraphics[scale=0.4]{img/boxplot_f1_micro.png}
	\caption{
		Testavimo duomenų klasifikavimo mikro f1 įverčiai, kur mvcnn yra daugiavaizdžio neuroninio tinklo f1 įverčiai, capsnet - kapsulinio neuroninio tinklo f1 įverčiai, mv\_cap\_capsnet1 - daugiavaizdžio kapsulinio neuroninio tinklo su vaizdų kapsuliniu sluoksniu ir vienu mokymosi etapu f1 įverčiai, mv\_cap\_capsnet2 - daugiavaizdžio kapsulinio neuroninio tinklo su vaizdų kapsuliniu sluoksniu ir dviem mokymosi etapais f1 įverčiai.
	}
	\label{img:box_micro_f1}
\end{figure}

\begin{figure}[H]
	\centering
	\includegraphics[scale=0.4]{img/boxplot_f1_macro.png}
	\caption{
		Testavimo duomenų klasifikavimo makro f1 įverčiai, kur mvcnn yra daugiavaizdžio neuroninio tinklo f1 įverčiai, capsnet - kapsulinio neuroninio tinklo f1 įverčiai, mv\_cap\_capsnet1 - daugiavaizdžio kapsulinio neuroninio tinklo su vaizdų kapsuliniu sluoksniu ir vienu mokymosi etapu f1 įverčiai, mv\_cap\_capsnet2 - daugiavaizdžio kapsulinio neuroninio tinklo su vaizdų kapsuliniu sluoksniu ir dviem mokymosi etapais f1 įverčiai.
	}
	\label{img:box_macro_f1}
\end{figure}


\sectionnonum{Rezultatai ir išvados}

Šiame magistro baigiamame darbe atlikta:

\begin{enumerate}
	\item Aprašytos ir realizuotos kapsulinių neuroninių tinklų dvi modifikacijos - daugiavaizdžiai kapsuliniai neuroniniai tinklai, iš kurių su vienas yra su vaizdų sujungimo sluoksniu ir kitas su vaizdų kapsuliniu sluoksniu.
	\item Atlikti tyrimai skirti nustatyti geriausią kapsulinių neuroninių tinklų modifikacijų konfigūracijas ir palyginti jas su nemodifikuotu kapsuliniu neuroniniu tinklu.
	\item Atlikti tyrimai, skirti palyginti kapsulinių neuroninių tinklų modifikacijas su dabartiniu geriausiu daugiavaizdžiu konvoliuciniu neuroniniu tinklu.
\end{enumerate}

Atlikus tyrimus šiame magistro baigiamame darbe gautos tokios išvados:

\begin{enumerate}
	\item Daugiavaizdžiai kapsuliniai neuroniniai tinklai pasiekia geresnius rezultatus per panašų apmokymo laiką nei kapsuliniai neuroniniai tinklai.
	\item Skirtumas tarp daugiavaizdžių kapsulinių neuroninių tinklų, iš kurių vienas yra su vaizdų kapsuliniu sluoksniu ir kitas su vaizdų sujungimo sluoksniu, yra nereikšmingas. Tačiau daugiavaizdžio kapsulinio neuroninio tinklo su vaizdų kapsuliniu sluoksniu, kuris apmokomas vienu etapu, tikslumas pradeda konverguoti greičiau nei daugiavaizdis kapsulinis neuroninis tinklas su vaizdų sujungimo sluoksniu. Tad daugiavaizdis kapsulinis neuroninis tinklas su vaizdų kapsuliniu sluoksniu yra pranašesnis laiko atžvilgiu.
	\item Nei viena tirta kapsulinio neuroninio tinklo modifikacija nėra pranašesnė už dabartinį geriausią daugiavaizdį konvoliucinį neuroninį tinklą.
	\item Kuo mažesnė duomenų imtis, tuo skirtumas tarp daugiavaizdžio kapsulinio neuroninio tinklo su vaizdų kapsuliniu sluoksniu, kuris yra apmokomas vienu etapu, ir dabartinio geriausio daugiavaizdžio konvoliucinio neuroninio tinklo yra nereikšmingesnis.
\end{enumerate}

Daugiavaizdžiui konvoliuciniui neuroniniui tinklui yra sudėtinga pritaikyti vaizdų kapsulinį sluoksnį nepaverčiant jo daugiavaizdžiu kapsuliniu neuroniniu tinklu. Tačiau pirmieji kapsulinių neuroninių tinklų sluoksniai sudaro konvoliucinį neuroninį tinklą. Todėl egzistuoja galimybė apjungti VGG-11 architektūrą, kuri naudojama dabartinio geriausio daugiavaizdžio konvoliucinio neuroninio tinklo pirmojo etapo apmokyme, su daugiavaizdžiu kapsuliniu neuroniniu tinklu su vaizdų kapsuliniu sluoksniu. Tad ateityje reikia tyrimais palyginti šį apjungtą dirbtinį neuroninį tinklą su dabartiniu geriausiu daugiavaizdžiu konvoliuciniu neuroniniu tinklu.


\printbibliography[heading=bibintoc]  % Literatūros šaltiniai aprašomi
% bibliografija.bib faile. Šaltinių sąraše nurodoma panaudota literatūra,
% kitokie šaltiniai. Abėcėlės tvarka išdėstoma tik darbe panaudotų (cituotų,
% perfrazuotų ar bent paminėtų) mokslo leidinių, kitokių publikacijų
% bibliografiniai aprašai (šiuo punktu pasirūpina LaTeX). Aprašai pateikiami
% netransliteruoti.

\appendix  % Priedai
% Prieduose gali būti pateikiama pagalbinė, ypač darbo autoriaus savarankiškai
% parengta, medžiaga. Savarankiški priedai gali būti pateikiami kompiuterio
% diskelyje ar kompaktiniame diske. Priedai taip pat vadinami ir numeruojami.
% Tekstas su priedais siejamas nuorodomis.

\end{document}
