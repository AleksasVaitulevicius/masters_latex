\documentclass{VUMIFInfMagistrinis}
\usepackage{algorithmicx}
\usepackage{algorithm}
\usepackage{algpseudocode}
\usepackage{amsfonts}
\usepackage{amsmath}
\usepackage{bm}
\usepackage{color}
\usepackage{listings}

% Titulinio aprašas
\university{Vilniaus universitetas}
\faculty{Matematikos ir informatikos fakultetas}
\department{Informatikos katedra}
\papertype{Magistro baigiamasis darbas}
\title{3D objektų atpažinimas iš 2D nuotraukų}
\titleineng{3D object recognition from 2D images}
\status{2 kurso 1 grupės studentas}
\author{Aleksas Vaitulevičius}
% \secondauthor{Vardonis Pavardonis}   % Pridėti antrą autorių
\supervisor{prof. habil. dr. Olga Kurasova}
\reviewer{doc. dr. Vardauskas Pavardauskas}
\date{Vilnius – \the\year}

% Nustatymai
% \setmainfont{Palemonas}   % Pakeisti teksto šriftą į Palemonas (turi būti įdiegtas sistemoje)
\bibliography{bibliografija}

\begin{document}
\maketitle

\sectionnonumnocontent{Santrauka}
Glaustai aprašomas darbo turinys, pristatoma nagrinėta problema ir padarytos
išvados. Santraukos apimtis ne didesnė nei 0,5 puslapio. Santraukų gale
nurodomi darbo raktiniai žodžiai. 
% Nurodomi iki 5 svarbiausių temos raktinių žodžių (terminų).
% Vienas terminas gali susidėti iš kelių žodžių.
\raktiniaizodziai{raktinis žodis 1, raktinis žodis 2, raktinis žodis 3, raktinis žodis 4, raktinis žodis 5}   

\sectionnonumnocontent{Summary}
Santrauka anglų kalba.
\keywords{keyword 1, keyword 2, keyword 3, keyword 4, keyword 5}

\tableofcontents

\sectionnonum{Įvadas}

Vienas iš fundamentalių kompiuterinės regos uždavinių yra informacijos apie 3 dimensijų (3D) pasaulį išgavimas naudojant 2 dimensijų (2D) nuotraukas. Šio uždavinio tikslas yra atpažinti konkrečius 3D objektus, naudojant jų, skirtingų apžvalgos taškų 2D nuotraukas. Šiam tikslui pasiekti, yra konstruojami objektų atpažinimo algoritmai, kurie klasifikuoja 2D nuotraukas į klases, kurios atstovauja vieną iš 3D objektų modelių.

3D objektų atpažinimas iš 2D nuotraukų yra naudojamas srityse, kuriose turimi 3D objektai turi būti atpažinti iš visų galimų 2D nuotraukų, turint tik poaibį šių nuotraukų. Keli šių sričių pavyzdžiai yra vogtų objektų aptikimas - turint algoritmą, atpažįstantį konkretų automobilį, galima iš viešų erdvių nuotraukų atrasti automobilio poziciją, vietos nustatymas - turint algoritmą, atpažįstantį konkrečius objektus esančius skirtingose vietovėse, ir tų vietovių koordinates galima nustatyti kurioje vietovėje buvo padaryta nuotrauka. Deja, laiko ir duomenų kaštai yra per dideli, kad pasiekti absoliutų tikslumą sprendžiant šį uždavinį. Todėl taikomi metodai yra euristiniai. Dėl to renkantis metodą, spręsti 3D objektų atpažinimo iš 2D nuotraukų uždaviniui, reikia atsižvelgti į laiko kaštus ir kaip tiksliai tuo metodu pagrįstas algoritmas klasifikuoja 2D nuotraukas, spręsdamas šį uždavinį. Šiame darbe, bus atliekami tyrimai, skirti nustatyti metodą, kuris būtų pranašesnis sprendžiant 3D objektų atpažinimo iš 2D nuotraukų uždavinį. Pasirinkti metodai yra konvoliucinio ir CapsNet neuroninio tinklo architektūros.

Gana dažnai naudojamas metodas, šiam uždaviniui spręsti, yra dirbtiniai gilieji neuroniniai tinklai. 3D objektų atpažinimo iš 2D nuotraukų uždavinyje, naudojami duomenys - 2D nuotraukos, yra nestruktūrizuoti, jiems sudėtinga vykdyti požymių išgavimą. Todėl daugelis kitų sprendimų nėra tokie patrauklūs kaip dirbtiniai gilieji neuroniniai tinklai, dėl savo sugebėjimo efektyviai vykdyti automatinį požymių išgavimą iš nestruktūrizuotų duomenų. Tačiau, norint pasiekti aukštą klasifikavimo tikslumą, naudojant šį metodą, yra reikalingas didelis kiekis duomenų. Konkrečiai šiam uždaviniui reikia didelio kiekio 3D modelių. Laimei, šiuo metu egzistuoja viešai prieinamos didelės 3D repozitorijos. Tokios kaip 3D Warehouse, TurboSquid, ir Shapeways. Dėl to, šiuo metu daugelis architektūrų jau yra išbandytos sprendžiant 3D objektų atpažinimo iš 2D nuotraukų uždavinį. Pavyzdžiui viena iš architektūrų, kuri buvo išbandyta, yra gilaus pasitikėjimo architektūra. Šiai architektūrai atlikti tyrimai yra aprašyti darbe \cite{dbnExp}.

Šiuo metu, šiam uždaviniui spręsti, optimaliausius  rezultatus, laiko ir tikslumo atžvilgiu, pasiekusi dirbtinio neuroninio tinklo architektūra yra konvoliuciniai neuroniniai tinklai. Tyrimai, kuriuose ši architektūra buvo išbandyta, yra aprašyta darbuose \cite{cnnExp1, cnnExp2}. Darbe \cite{dbnExp} atlikto tyrimo rezultatai parodo, kad pateiktas sprendimas, kuriame 3D objektų atpažinimas yra konstruojamas naudojantis tik 2D nuotraukomis, yra tikslesnis 8\% (77\% → 85\%). Todėl šiame darbe bus daromi eksperimentai su dirbtinio neuroninio tinklo architektūrų įgyvendinimais, kurie yra pagrįsti būtent šiuo metodu. Tad šiam darbui, vienas iš pasirinktų metodų yra konvoliuciniai neuroniniai tinklai, kurio įgyvendinimas ir tyrimai yra aprašyti darbe \cite{cnnExp1}. Mat šio darbo įgyvendinimas naudoja tik 2D nuotraukas, konstruojant 3D objektų atpažinimo algoritmą.

Kita tiriama dirbtinio neuroninio tinklo architektūra yra CapsNet. Lyginant su konvoliuciniu neuroniniu tinklu, tai gana nauja architektūra. Aprašyta 2017 metais darbe \cite{capsNet} CapsNet architektūros veikimo principas tiksliau imituoja žmogaus rega, remiantis faktu, kad žmogaus rega ignoruoja nereikšmingas vaizdo detales, naudodama tik seką fokusuotų taškų, taip apdorodama tik dalį vaizdinės informacijos su labai aukšta rezoliucija. \cite{capsNet} darbe atliktas tyrimas parodo, kad ši architektūra atlieka skaičių klasifikavimo užduotį tiksliau nei konvoliucinis neuroninis tinklas. Kitas tyrimas, kuris yra atliktas darbe \cite{capsCNN} su 4 duomenų rinkiniais, kuriuose yra veidai, kelio ženklai ir kasdienius objektus, parodo, kad dabartinis CapsNet neuroninis tinklas nevisąlaiką yra pranašesnis už konvoliucinį neuroninį tinklą. Parinkus geresnius parametrus arba sukūrus geresnį dizainą, konvoliucinis tinklas dar vis būna pranašesnis už CapsNet neuroninį tinklą. Tačiau taip pat šiame darbe yra teigiama, kad CapsNet neuroniniai tinklai dar nėra pasiekę pilno savo potencialo ir tolimesni tyrimai turi būti atlikti.

Tad šiame darbe yra keliama tokia \textbf{hipotezė}:

\textit{CapsNet neuroninis tinklas sprendžia 3D objektų atpažinimo iš 2D nuotraukų uždavinį efektyviau nei konvoliucinis neuroninis tinklas remiantis apmokymo laiko ir tikslumo kriterijais}.

Tikimasi, kad, sprendžiant 3D objektų atpažinimo iš 2D nuotraukų uždavinį, CapsNet neuroninio tinklo mokymas truks trumpiau nei konvoliucinio neuroninio tinklo. Taip pat, kad apmokytas CapsNet neuroninis tinklas vykdys klasifikavimą tiksliau nei konvoliucinis neuroninis tinklas.

Siekiant patikrinti iškeltą hipotezę reikia atlikti šiuos uždavinius:

\begin{enumerate}
	\item Išanalizuoti ir nustatyti dabartinį efektyviausią 3D objektų atpažinimo iš 2D nuotraukų uždavinio sprendinį, remiantis literatūra.
	\item Išanalizuoti CapsNet neuroninio tinklo veikimą, remiantis literatūra.
	\item Surasti duomenis, skirtus spręsti 3D objektų atpažinimo iš 2D nuotraukų uždaviniui.
	\item Eksperimentiškai nustatyti efektyviausius parametrus ir konfigūracijas, skirtus spręsti 3D objektų atpažinimo iš 2D nuotraukų uždaviniui, CapsNet neuroninio tinklo implementacijai, remiantis apmokymo laiko ir tikslumo kriterijais.
	\item Eksperimentiškai palyginti 1. užduoties rezultatą su 3. užduoties.
\end{enumerate}

Šiame darbe atlikta:

\begin{enumerate}
	\item Nustatyta, kad šiuo metu efektyviausiausias 3D objektų atpažinimo iš 2D nuotraukų uždavinio sprendinys yra konvoliuciniai neuroniniai tinklai, lyginant eksperimentų, aprašytų skirtinguose literatūros šaltiniuose, rezultatus. Šiuose šaltiniuose buvo surasta konvoliucinio neuroninio tinklo implementaciją ir duomenys skirti apmokymui ir testavimui.
	\item Išanalizuotas CapsNet neuroninio tinklo veikimas, surasta jo implementacija.
	\item Eksperimentiškai nustatyti efektyviausią konfigūraciją CapsNet neuroniniui tinklui sprendžiant 3D objektų atpažinimo iš 2D nuotraukų uždaviniui, naudojantis apmokymo laiko ir tikslumo kriterijais.
	\item Eksperimentiškai palyginti CapsNet neuroninio tinklo tiklsumas ir apmokymo laikas su konvoliuciniu neuroniniu tinklu, naudojantis apmokymo laiko ir tikslumo kriterijais.
\end{enumerate}

Darbas remiasi šiomis prielaidomis:

\begin{enumerate}
	\item Kiekvienam 2D paveikslėliui yra priskirta jam jį atitinkantis 3D objektas.
	\item Kiekvienas 3D objektas turi bent po vieną jį atitinkantį 2D paveikslėlį.
\end{enumerate}

Šio darbo turinys yra pateiktas taip. Pradedama nuo 3D objektų atpažinimo iš 2D nuotraukų uždavinio aprašymo ir pristatymo pirmame skyriuje. Tada yra pateikiamas konvoliucinio neuroninio tinklo veikimo aprašymas antrame skyriuje. Toliau trečiame skyriuje yra CapsNet veikimo aprašymas, išbandytos konfigūracijos ir jų rezultatai. Ketvirtame skyriuje yra aprašomi duomenys naudoti eksperimentams ir patys eksperimentai ir jų rezultatai. Paskutiniame skyriuje yra pateikiamos išvados.


\section{Medžiagos darbo tema dėstymo skyriai}
Medžiagos darbo tema dėstymo skyriuose pateikiamos nagrinėjamos temos detalės:
pradiniai duomenys, analizės ir apdorojimo metodai, sprendimų įgyvendinimas,
gautų rezultatų apibendrinimas.

\subsection{Poskyris}
Citavimo pavyzdžiai: cituojamas vienas šaltinis \cite{PvzStraipsnLt}; cituojami
keli šaltiniai \cite{PvzStraipsnEn, PvzKonfLt, PvzKonfEn, PvzKnygLt, PvzKnygEn,
PvzElPubLt, PvzElPubEn, PvzMagistrLt, PvzPhdEn}.

\subsubsection{Skirsnis}
\subsubsubsection{Straipsnis}
\subsubsection{Skirsnis}
\section{Skyrius}
\subsection{Poskyris}
\subsection{Poskyris}

\sectionnonum{Rezultatai ir išvados}
Rezultatų ir išvadų dalyje išdėstomi pagrindiniai darbo rezultatai (kažkas
išanalizuota, kažkas sukurta, kažkas įdiegta), pateikiamos išvados (daromi
nagrinėtų problemų sprendimo metodų palyginimai, siūlomos rekomendacijos,
akcentuojamos naujovės).

\printbibliography[heading=bibintoc]  % Literatūros šaltiniai aprašomi
% bibliografija.bib faile. Šaltinių sąraše nurodoma panaudota literatūra,
% kitokie šaltiniai. Abėcėlės tvarka išdėstoma tik darbe panaudotų (cituotų,
% perfrazuotų ar bent paminėtų) mokslo leidinių, kitokių publikacijų
% bibliografiniai aprašai (šiuo punktu pasirūpina LaTeX). Aprašai pateikiami
% netransliteruoti.

% \sectionnonum{Sąvokų apibrėžimai}
\sectionnonum{Santrumpos}
Sąvokų apibrėžimai ir santrumpų sąrašas sudaromas tada, kai darbo tekste
vartojami specialūs terminai, reikalaujantys paaiškinimo, ir rečiau sutinkamos
santrumpos.

\appendix  % Priedai
% Prieduose gali būti pateikiama pagalbinė, ypač darbo autoriaus savarankiškai
% parengta, medžiaga. Savarankiški priedai gali būti pateikiami kompiuterio
% diskelyje ar kompaktiniame diske. Priedai taip pat vadinami ir numeruojami.
% Tekstas su priedais siejamas nuorodomis.

\section{Niauroninio tinklo struktūra}


\section{Eksperimentinio palyginimo rezultatai}
% tablesgenerator.com - converts calculators (e.g. excel) tables to LaTeX
\begin{table}[H]\footnotesize
  \centering
  \caption{Lentelės pavyzdys}
  {\begin{tabular}{|l|c|c|} \hline
    Algoritmas & $\bar{x}$ & $\sigma^{2}$ \\
    \hline
    Algoritmas A  & 1.6335    & 0.5584       \\
    Algoritmas B  & 1.7395    & 0.5647       \\
    \hline
  \end{tabular}}
  \label{tab:table example}
\end{table}

\end{document}
