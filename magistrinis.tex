\documentclass{VUMIFInfMagistrinis}
\usepackage{algorithmicx}
\usepackage{algorithm}
\usepackage{algpseudocode}
\usepackage{amsfonts}
\usepackage{amsmath}
\usepackage{bm}
\usepackage{color}
\usepackage{listings}
\usepackage{graphicx}
\newcommand{\R}{\mathbb{R}}

% Titulinio aprašas
\university{Vilniaus universitetas}
\faculty{Matematikos ir informatikos fakultetas}
\department{Informatikos katedra}
\papertype{Magistro baigiamasis darbas}
\title{3D objektų atpažinimas iš 2D nuotraukų}
\titleineng{3D object recognition from 2D images}
\status{2 kurso 1 grupės studentas}
\author{Aleksas Vaitulevičius}
\supervisor{prof. habil. dr. Olga Kurasova}
\reviewer{doc. dr. Vardauskas Pavardauskas}
\date{Vilnius – \the\year}

% Nustatymai
% \setmainfont{Palemonas}   % Pakeisti teksto šriftą į Palemonas (turi būti įdiegtas sistemoje)
\bibliography{bibliografija}

\begin{document}
\maketitle

\sectionnonumnocontent{Santrauka}
Glaustai aprašomas darbo turinys, pristatoma nagrinėta problema ir padarytos
išvados. Santraukos apimtis ne didesnė nei 0,5 puslapio. Santraukų gale
nurodomi darbo raktiniai žodžiai. 
% Nurodomi iki 5 svarbiausių temos raktinių žodžių (terminų).
% Vienas terminas gali susidėti iš kelių žodžių.
\raktiniaizodziai{Klasifikavimo uždavinys, 3D objektai, dirbtiniai neuroniniai tinklai, kapsuliniai neuroniniai tinklai, tiesioginio sklidimo neuroniniai tinklai}   

\sectionnonumnocontent{Summary}
Santrauka anglų kalba.
\keywords{classification, 3D objects, artificial neural networks, CapsNet neural networks, convolutional neural networks}

\tableofcontents

\sectionnonum{Įvadas}

Vienas iš fundamentalių kompiuterinės regos uždavinių yra informacijos apie 3 dimensijų (3D) pasaulį išgavimas naudojant 2 dimensijų (2D) nuotraukas. Šio uždavinio tikslas yra atpažinti konkrečius 3D objektus, naudojant jų, 2D nuotraukas, padarytas iš skirtingų kampų. Šiam tikslui pasiekti, yra konstruojami objektų atpažinimo algoritmai, kurie klasifikuoja 2D nuotraukas į klases, kurios atstovauja vieną iš 3D objektų modelių.

3D objektų atpažinimas iš 2D nuotraukų yra naudojamas srityse, kuriose turimi 3D objektai turi būti atpažinti iš visų galimų 2D nuotraukų, turint tik poaibį šių nuotraukų. Keli šių sričių pavyzdžiai yra automatinė objektų inspekcija - turint algoritmą, atpažįstantį konkretų objektą, kuris turi tik jam būdinga 3D formą, galima nustatyti nuotraukas, kuriose yra tas objektas, navigacijoje - turint algoritmą, atpažįstantį konkrečius objektus, esančius skirtingose vietovėse, ir tų vietovių koordinates, galima nustatyti kurioje vietovėje buvo padaryta nuotrauka. Deja, laiko ir duomenų kaštai yra per dideli, kad pasiekti absoliutų tikslumą sprendžiant šį uždavinį. Todėl taikomi metodai yra euristiniai. Dėl to renkantis metodą, spręsti 3D objektų atpažinimo iš 2D nuotraukų uždaviniui, reikia atsižvelgti į laiko kaštus ir kaip tiksliai tuo metodu pagrįstas algoritmas klasifikuoja 2D nuotraukas, spręsdamas šį uždavinį. Šiame darbe bus atliekami tyrimai, skirti nustatyti metodą, sprendžiantį 3D objektų atpažinimo iš 2D nuotraukų uždavinį ir kuris pasiekia didžiausią tikslumą ir reikalauja mažiausiai laiko mokymui.

Gana dažnai naudojamas metodas šiam uždaviniui spręsti, yra dirbtiniai gilieji neuroniniai tinklai. 3D objektų atpažinimo iš 2D nuotraukų uždavinyje naudojami duomenys yra 2D nuotraukos, yra nestruktūrizuoti, jiems sudėtinga vykdyti požymių išgavimą. Todėl daugelis kitų sprendimų nėra tokie patrauklūs kaip dirbtiniai gilieji neuroniniai tinklai, dėl savo sugebėjimo efektyviai vykdyti automatinį požymių išgavimą iš nestruktūrizuotų duomenų. Tačiau, norint pasiekti aukštą klasifikavimo tikslumą, naudojant šį metodą, yra reikalingas didelis kiekis duomenų. Konkrečiai šiam uždaviniui reikia didelio kiekio 3D modelių. Laimei, šiuo metu egzistuoja viešai prieinamos didelės 3D repozitorijos. Tokios kaip 3D Warehouse, TurboSquid, ir Shapeways. Dėl to šiuo metu daugelis senesnių architektūrų jau yra išbandytos sprendžiant 3D objektų atpažinimo iš 2D nuotraukų uždavinį. Pavyzdžiui viena iš architektūrų, kuri buvo išbandyta, yra konvoliucinio gilaus pasitikėjimo neuroninio tinklo (angl. convolutional deep belief neural network) architektūra. Šiai architektūrai atlikti tyrimai yra aprašyti darbe \cite{dbnExp}. Tačiau 2017 metais buvo aprašyta nauja architektūra, kapsuliniai neuroniniai tinklai. Tyrimai, parodė, kad ji yra pranašesnė tikslumo atžvilgiu už senesnes architektūras, sprendžiant uždavinius panašius į 3D objektų atpažinimo iš 2D nuotraukų uždavinį.

Šiuo metu šiam uždaviniui spręsti, optimaliausius  rezultatus, laiko ir tikslumo atžvilgiu, pasiekusi dirbtinio neuroninio tinklo architektūra yra konvoliuciniai neuroniniai tinklai (angl. convolutional neural networks). Tyrimai, kuriuose ši architektūra buvo išbandyta, yra aprašyta darbuose \cite{cnnExp1, cnnExp2}. Darbe \cite{dbnExp} atlikto tyrimo rezultatai parodo, kad pateiktas sprendimas, kuriame 3D objektų atpažinimas yra konstruojamas naudojantis tik 2D nuotraukomis, yra tikslesnis 8 \%. Algoritmas, naudojantis 3D modelius, pasiekė 77 \% tikslumą, o algoritmas, naudojantis tik 2D nuotraukas, pasiekė 85 \% tikslumą. Todėl šiame darbe bus daromi eksperimentai su dirbtinio neuroninio tinklo architektūrų įgyvendinimais, kurie yra pagrįsti būtent šiuo metodu. Tad šiam darbui, vienas iš pasirinktų metodų yra konvoliucinis neuroninis tinklas, kurio įgyvendinimas ir tyrimai yra aprašyti darbe \cite{cnnExp1}. Mat šio darbo įgyvendinimas naudoja tik 2D nuotraukas, konstruojant 3D objektų atpažinimo algoritmą.

Kita tiriama dirbtinio neuroninio tinklo architektūra yra kapsuliniai neuroniniai tinklai. Lyginant su konvoliuciniais neuroniniais tinklais, tai gana nauja architektūra. Aprašytos 2017 metais \cite{capsNet} darbe Kapsulinių neuroninių tinklų architektūros veikimo principas tiksliau imituoja žmogaus rega, remiantis faktu, kad žmogaus rega ignoruoja nereikšmingas vaizdo detales, naudodama tik seką fokusuotų taškų, taip apdorodama tik dalį vaizdinės informacijos su labai aukšta rezoliucija. \cite{capsNet} darbe atliktas tyrimas parodo, kad ši architektūra atlieka ranka rašytų skaičių klasifikavimo užduotį tiksliau nei konvoliuciniai neuroniniai tinklai. Kitas tyrimas, kuris yra atliktas darbe \cite{capsCNN} su 4 duomenų rinkiniais, kuriuose yra veidai, kelio ženklai ir kasdieniai objektai, parodo, kad dabartiniai kapsuliniai neuroniniai tinklai ne visada yra pranašesni už konvoliucinius neuroninius tinklus. Parinkus geresnius parametrus ir modifikacijas (sluoksnių skaičių, neuronų skaičių kiekviename sluoksnyje, aktyvacijos funkcijos), konvoliucinis neuroninis tinklas dar vis būna pranašesnis už kapsulinį neuroninį tinklą. Tačiau taip pat darbe \cite{capsCNN} yra teigiama, kad kapsuliniai neuroniniai tinklai dar nėra pasiekę pilno savo potencialo ir tolimesni tyrimai turi būti atlikti.

Tad šio darbo \textbf{tikslas} yra įrodyti arba paneigti keliamą \textbf{hipotezę}:

\textit{Kapsuliniai neuroniniai tinklai sprendžia 3D objektų atpažinimo iš 2D nuotraukų uždavinį efektyviau nei konvoliuciniai neuroniniai tinklai remiantis apmokymo laiko ir tikslumo kriterijais}.

Tikimasi, kad, sprendžiant 3D objektų atpažinimo iš 2D nuotraukų uždavinį, kapsulinio neuroninio tinklo mokymas truks trumpiau nei konvoliucinio neuroninio tinklo. Taip pat, kad apmokytas kapsulinis neuroninis tinklas vykdys klasifikavimą tiksliau nei konvoliucinis neuroninis tinklas.

Siekiant patikrinti iškeltą hipotezę reikia atlikti šiuos uždavinius:

\begin{enumerate}
	\item Išanalizuoti ir nustatyti dabartinį efektyviausią 3D objektų atpažinimo iš 2D nuotraukų uždavinio sprendinį.
	\item Išanalizuoti kapsulinių neuroninių tinklų veikimą.
	\item Surasti duomenis, skirtus spręsti 3D objektų atpažinimo iš 2D nuotraukų uždaviniui.
	\item Eksperimentiškai nustatyti efektyviausius parametrus ir modifikacijas, skirtus spręsti 3D objektų atpažinimo iš 2D nuotraukų uždaviniui, kapsulinio neuroninio tinklo implementacijai, remiantis apmokymo laiko ir tikslumo kriterijais.
	\item Atlikti eksperimentus, skirtus palyginti kapsulino neuronino tinklo ir konvoliucinio neuronino tinklo tikslumą ir apmokymo laiką, sprendžiant 3D objektų atpažinimo iš 2D nuotraukų uždavinį.
\end{enumerate}

Šiame darbe planuojami rezultatai:

\begin{enumerate}
	\item Nustatyta, kad šiuo metu efektyviausias 3D objektų atpažinimo iš 2D nuotraukų uždavinio sprendinys yra konvoliuciniai neuroniniai tinklai, lyginant eksperimentų, aprašytų skirtinguose literatūros šaltiniuose, rezultatus. Šiuose šaltiniuose buvo surasta konvoliucinio neuroninio tinklo implementaciją ir duomenys skirti apmokymui ir testavimui.
	\item Išanalizuotas kapsulinių neuroninių tinklų veikimas, surasta jo implementacija.
	\item Eksperimentiškai nustatyta efektyviausia konfigūracija kapsuliniui neuroniniui tinklui sprendžiant 3D objektų atpažinimo iš 2D nuotraukų uždaviniui, naudojantis apmokymo laiko ir tikslumo kriterijais.
	\item Eksperimentiškai palygintas kapsulinio neuroninio tinklo tikslumas ir apmokymo laikas su konvoliuciniu neuroniniu tinklu, naudojantis apmokymo laiko ir tikslumo kriterijais.
\end{enumerate}

Darbas remiasi šiomis prielaidomis:

\begin{enumerate}
	\item Kiekvienam 2D paveikslėliui yra priskirta jam jį atitinkantis 3D objektas.
	\item Kiekvienas 3D objektas turi bent po vieną jį atitinkantį 2D paveikslėlį.
\end{enumerate}

Šio darbo turinys yra sudarytas iš 4 skyrių. Pirmame skyriuje yra pateikiama literatūros analizė. Jame yra pateiktas 3D objektų atpažinimo iš 2D nuotraukų uždavinio aprašymas, egzistuojančių sprendimų apžvalga, bendrieji neuroninių tinklų principai, konvoliucinio neuroninio tinklo veikimo aprašymas ir kapsulinio neuroninio tinklo aprašymas.
Tada antrame skyriuje yra pateikiami šiame darbe bandomų kapsulinių neuroninių tinklų modifikacijos ir parinkti parametrai.
Trečiame skyriuje yra aprašomi tyrimams naudoti duomenys.
Taip pat šiame skyriuje yra aprašomi tyrimai, skirti nustatyti kapsulinių neuroninių tinklų modifikaciją ir parametrus, kurie pasiekia didžiausią tikslumą ir reikalauja mažiausiai laiko apmokymui, sprendžiant 3D objektų atpažinimo iš 2D nuotraukų uždavinį.
Galiausiai šiame skyriuje yra aprašomi tyrimai, skirti palyginti kapsulinių neuroninių tinklų ir konvoliucinių neuroninių tinklų tikslumą ir apmokymo laiką, sprendžiant 3D objektų atpažinimo iš 2D nuotraukų uždavinį.
Paskutiniame skyriuje yra pateikiami rezultatai ir išvados.


\section{Literatūros analizė}

\subsection{3D objektų atpažinimo iš 2D nuotraukų uždavinys}

3D objektų atpažinimo iš 2D nuotraukų uždavinys - tai klasifikavimo uždavinys, kuriame pateiktos 2D nuotraukos, kuriose yra atvaizduotas 3D objektas iš atsitiktinio apžvalgos taško, turi būti priskirtas 3D modeliui, kuris yra atvaizduotas toje 2D nuotraukoje.

Klasifikavimo uždavinys - tai uždavinys, kuriame kuriamas metodas, kaip nustatyti pavyzdžio, iš tiriamos srities populiacijos, klasę. 3D objektų atpažinimo iš 2D nuotraukų uždavinio atveju, tiriama sritis yra 2D nuotraukos, kuriose yra atvaizduotas 3D objektas iš bet kurio apžvalgos taško ir klasė - 3D objektas. Taip pat, šio darbo atveju, metodas yra dirbtinio neuroninio tinklo (kapsulinio arba konvoliucinio) apmokytas modelis.

Kaip jau minėta įvade, šiam uždaviniui spręsti efektyviausia yra naudoti mašininio mokymo metodą, kurio mokymo duomenys yra tik 2D nuotraukos, o 3D objektai bus tik duomenų klasės. Darbe \cite{dbnExp} atlikto tyrimo rezultatai parodo, kad pateiktas sprendimas, kuriame 3D objektų atpažinimas yra konstruojamas naudojantis tik 2D nuotraukomis, yra tikslesnis 8 \%. Algoritmas, naudojantis 3D modelius, pasiekė 77 \% tikslumą, o algoritmas, naudojantis tik 2D nuotraukas, pasiekė 85 \% tikslumą. Šaltinyje \cite{cnnExp1} yra teigiama, kad to priežastis yra reliatyviai efektyvesnis 2D nuotraukų informacijos saugojimas negu 3D modelių. Todėl, kad, nors 3D modelis turi visą informaciją apie atvaizduotą 3D objektą, tačiau tam, kad panaudoti vokselinę 3D objekto reprezentaciją mašininiame mokyme, kurio mokymas su pakankamai didele duomenų imtimi užtruktų racionalų laiko tarpą, tenka ženkliai sumažinti 3D modelio rezoliuciją. Pavyzdžiui, 3D modelio, kurio rezoliucija yra $30\times30\times30$ vokseliai, įvesties dydis yra apytiksliai lygus 2D paveikslėlio, kurio rezoliucija yra $164\times164$ pikseliai. Tad šiuo atveju, 3D modelis yra apdorojamas per tiek pat laiko kaip ir 2D paveikslėlis, bet modelio rezoliucija yra apytiksliai 5.5 karto mažesnė. Todėl mašininio mokymo metodas, kurio mokymo duomenys yra 3D modelis, gauna mažesnės raiškos įvestį, negu metodas, kurio mokymo duomenys yra 2D paveikslėliai.

\subsection{3D objektų atpažinimo iš 2D nuotraukų uždavinio sprendinių pavyzdžiai}

Vienas seniausių šio uždavinio sprendinių, taikantis tokią metodologiją, yra aprašytas darbe \cite{prevWparEig}. Šis sprendinys atpažįsta 3D objektus lygindamas jų vaizdus, kurie buvo suformuoti iš didelės imties 2D nuotraukų, parametrizuotoje 
eigenerdvėje (angl. eigenspace) % Kaip lietuviškai?
. Šios nuotraukos buvo sugeneruotos iš 3D modelių naudojant skirtingus apžvalgos taškus ir apšvietimus. 
Kitas pavyzdys, kuris yra gana populiarus kompiuterinėje grafikoje, yra 
šviesos lauko deskriptorius (angl. light field descriptor)% light field descriptor
, kuris yra aprašytas darbe \cite{prevWLightFld}. Šis sprendinys išgauna geometrinius ir 
Fourier'io % kaip daryti su angliškais vardais?
deskriptorius iš 3D objektų siluetų, kurie buvo sugeneruoti iš 3D modelių, naudojant skirtingus apžvalgos taškus. 
Darbe \cite{prevWShockGraph} aprašytas šio uždavinio sprendimas, kuris 3D objekto siluetus išskaido į dalis ir išsaugo juos į 
orientuotą beciklį grafą (angl. directed acyclic graph) % directed acyclic graph 
, šoko grafą. % shock graph
Kitas pavyzdys aprašytas darbe \cite{prevWSimMet}, naudoja panašumo metrikas (angl. similarity metrics) % similarity metrics
, kurios yra pagrįstos kreivių palyginimu (angl. curve matching)% curve matching
ir sugrupuotomis panašiomis 2D nuotraukomis.

Šiuo metu, 3D objektų atpažinimo iš 2D nuotraukų uždaviniui spręsti, optimaliausius  rezultatus, laiko ir tikslumo atžvilgiu, pasiekęs mašininio mokymo metodu pagrįstas sprendimas yra konvoliuciniai dirbtiniai neuroniniai tinklai. Tai eksperimentu buvo įrodyta darbe \cite{cnnExp1}. Šiame eksperimente buvo palyginti įvairūs konvoliucinių neuroninių tinklų tipai sprendžiant šį uždavinį ir geriausią rezultatą pasiekęs tipas buvo
daugiavaizdis (angl. multi-view convolutional network)% multi-view
konvoliucinis neuroninis tinklas, kurio tikslumas buvo 90.1\%.

\subsection{Dirbtinių neuroninių tinklų bendrieji principai}
\subsubsection{Dirbtinis neuronas, perceptronas}

Šiame magistro baigiamajame darbe nagrinėjami dirbtiniai neuroniniai tinklai yra sudaryti iš Rosenblato darbe \cite{rosenPerc} aprašytų dirbtinių neuronų, perceptronų. Perceptronas --  tai iteratyviai apmokomas tiesinis klasifikatorius, kuris susideda iš $\boldsymbol{x} = \{x_{0}, x_{1}, x_{2}, ..., x_{p}\}$ mokymo aibės vektorių, vadinamais įėjimais, $\{w_{0}, w_{1}, w_{2}, ..., w_{p}\} \in \R$ perdavimo koeficientų, vadinamų svoriais, aktyvacijos (perdavimo) funkcijos $f(a)$ ir $\{y_{0}, y_{1}, y_{2}, ..., y_{n}\}$ reikšmių, vadinamų išėjimais. Įėjimas $x_{0}$ yra vadinamas nuliniu įėjimu ir jo reikšmė yra pastovi $x_{0} = 1$, o $w_{0}$ - nuliniu svoriu arba slenksčiu (angl. bias). Perceptronas yra atvaizduotas \ref{img:perceptron} paveikslėlyje.

\begin{figure}[H]
	\centering
	\includegraphics[scale=0.5]{img/perceptron.png}
	\caption{Perceptronas}
	\label{img:perceptron}
\end{figure}

Formulė (\ref{eqn:activ_arg}) yra aktyvacijos funkcijos argumentas.

\begin{equation}
	\label{eqn:activ_arg}
	a = \sum_{k = 0}^{p} w_{k}x_k
\end{equation}

Dažniausiai perceptronui yra naudojamos šios aktyvacijos funkcijos: slenkstinė (angl. unit step) (\ref{eqn:unitStep}), sigmoidinė (angl. sigmoid) (\ref{eqn:sigmoid}), gabalais tiesinė (angl. piecewise linear) (\ref{eqn:pieceLinear}), Gauso (angl. Gaussian) (\ref{eqn:gaussian}) ir tiesinė (angl. linear) (\ref{eqn:linear}), kur $\beta$, $\mu$, $\sigma$, $m$, $a_{min}$, $a_{max}$ yra konstantos priklausančios realiųjų skaičių aibei bei $a_{min} < a_{max}$.

\begin{equation}
	\label{eqn:unitStep}
	f(a) =
	\begin{cases}
		0, & \mbox{jei } 0 > a \\
		1, & \mbox{jei } 0 \leq a
	\end{cases}
\end{equation}

\begin{equation}
	\label{eqn:sigmoid}
	f(a) = \dfrac{1}{1 + \exp(-a)}
\end{equation}

\begin{equation}
	\label{eqn:pieceLinear}
	f(a) =
	\begin{cases}
		0, & \mbox{jei } a_{min} \geq a \\
		ma + b, & \mbox{jei } a_{min} < a < a_{max} \\
		1, & \mbox{jei } a_{max} \leq a
	\end{cases}
\end{equation}

\begin{equation}
	\label{eqn:gaussian}
	f(a) = \dfrac{1}{\sqrt{2\pi\sigma}} \exp(\dfrac{-(a - \mu)^2}{2\sigma^2})
\end{equation}

\begin{equation}
	\label{eqn:linear}
	f(a) = ma + b
\end{equation}

Perceptronas yra skirtas spręsti klasifikavimo uždavinius. Tam kad perceptronas spręstų konkretų klasifikavimo uždavinį, jis turi būti apmokytas. Perceptrono apmokymas yra iteratyvus procesas, kuriame randami svoriai $W = \{w_{0}, w_{1}, w_{2}, ..., w_{p}\}$, su kuriais funkcijos (\ref{eqn:mse}) rezultatas įgyja kiek galima mažiausią reikšmę. Funkcijoje (\ref{eqn:mse}) $y_i$ yra perceptrono \textit{i}-tasis išėjimas, $t_i$ - \textit{i}-tojo įėjimo norima klasė ir $n$ - apmokymo duomenų vektorių skaičius.

\begin{equation}
	\label{eqn:mse}
	e(w) = \dfrac{1}{n}\sum_{i=1}^{n}(y_i - t_i)^2
\end{equation}

Apmokymo pradžioje pradiniai svoriai yra parenkami atsitiktinai. Toliau gradientinio nusileidimo algoritmu judant antigradiento kryptimi, svorių reikšmės perskaičiuojamos naudojantis formule (\ref{eqn:w_recalc}), kur \textit{k}-tojo svorio gradientas $\Delta w_k(t)$ apskaičiuojamas pagal formulę (\ref{eqn:w_change}), $t$ - iteracijos numeris, $\eta \in [0, +\infty]$ - parinktas mokymo greitis (angl. learning rate). Vienoje iteracijoje yra naudojamas tik vienas įėjimo vektorius iš duomenų aibės. Svoriai yra perskaičiuojami norima skaičių kartų.

\begin{equation}
	\label{eqn:w_recalc}
	w_k(t + 1) = w_k(t) + \Delta w_k(t)
\end{equation}

\begin{equation}
	\label{eqn:w_change}
	\Delta w_k(t) = - \eta \dfrac{\partial e(w)}{\partial w_k}
\end{equation}
% išsivedamas bendras atvejis ------------------------------------------------------------------------------------------------------------------------
Pritaikius formulę (\ref{eqn:activ_arg}) \textit{i}-tojo įėjimo vektoriaus aktyvacijos funkcijos argumento apskaičiavimui gaunama formulė (\ref{eqn:activ_arg_per_i}), kur $a_i$ yra i-tojo įėjimo vektoriaus aktyvacijos funkcijos argumentas, $x_{ik}$ yra \textit{i}-tojo įėjimo vektoriaus \textit{k}-atoji komponentė.

\begin{equation}
	\label{eqn:activ_arg_per_i}
	a_i = \sum_{k = 0}^{p} w_{k}x_{ik}
\end{equation}

Tad \textit{i}-tasis perceptrono išėjimas $y_i$ yra $y_i = f(a_i)$. Tada funkcijos (\ref{eqn:mse}) išvestinė yra paskaičiuojama pagal formulę (\ref{eqn:expanded}).

\begin{equation}
	\label{eqn:expanded}
	\dfrac{\partial e(w)}{\partial w_k} = (\dfrac{1}{n}\sum_{i=1}^{n} (y_i - t_i)^2)'
		= \dfrac{2}{n}\sum_{i=1}^{n}
			((y_i - t_i)(f'(a_i))(\sum_{k = 1}^{p} x_{ik}))
\end{equation}

Tada bendru atveju perceptrono mokymo taisyklė (\ref{eqn:w_recalc}) yra funkcija (\ref{eqn:general}).

\begin{equation}
	\label{eqn:general}
	w_k(t + 1) = w_k(t) - \eta \dfrac{2}{n}\sum_{i=1}^{n} ((y_i - t_i)(f'(a_i))(\sum_{k = 0}^{p} x_{ik}))
\end{equation}

% išsivedamas bendras atvejis ------------------------------------------------------------------------------------------------------------------------
Naudojantis apmokytu perceptronu galima nustatyti ar duotas duomenų vektorius $\boldsymbol{x}'$ priklauso klasei $c$. Pirmiausiai randamas skiriamasis paviršius (angl. decision boundary). Skiriamasis paviršius - tai kreivė, gaunama iš formulės (\ref{eqn:activ_arg}) su apmokyto perceptrono svoriais ir kai $a = d$, kur $d$ yra konstanta, su kuria vektoriai, kurie patenkina sąlygą $d \ge f(a)$, yra interpretuojami kaip nepriklausantys klasei $c$. Pavyzdžiui, aktyvacijos funkcijos (\ref{eqn:sigmoid}) konstanta $d = 0,5$. Skiriamasis paviršius padalina duomenų vektorių erdvę į du regionus. Jei $\boldsymbol{x}'$ priklauso regionui, kuriame vektoriai patenkina sąlygą $a > d$, kur $a$ randamas naudojant formulę (\ref{eqn:activ_arg}) su apmokyto perceptrono svoriais ir $\{x_{0}, x_{1}, x_{2}, ..., x_{p}\} = \boldsymbol{x}'$, tai $\boldsymbol{x}'$ priklauso klasei $c$, kitu atveju - ne.
\subsubsection{Dirbtinis neuronas, perceptronas}

Perceptronas -  tai iteratyviai apmokomas tiesinis klasifikatorius, kuris susideda iš $\{x_{0}, x_{1}, x_{2}, ..., x_{n}\}$ mokymo aibės vektorių, vadinamais įėjimais, $\{w_{0}, w_{1}, w_{2}, ..., w_{n}\} \in \R$ perdavimo koeficientų, vadinamų svoriais, aktyvacijos (perdavimo) funkcijos $f(a)$ ir $\{y_{0}, y_{1}, y_{2}, ..., y_{n}\}$ reikšmių, vadinamų išėjimais. Įėjimas $x_{0}$ yra vadinamas nuliniu įėjimu ir jo reikšmė yra pastovi $x_{0} = 1$, o $w_{0}$ - nuliniu svoriu arba slenksčiu (angl. bias). Aktyvacijos funkcijos argumentas yra įėjimo reikšmių ir svorių sandaugų suma:

\begin{equation}
	a = \sum_{k = 1}^{n} w_{k}x_{k}
\end{equation}

Dažniausiai yra naudojamos šios aktyvacijos funkcijos: slenkstinė (angl. unit step) \ref{eqn:unitStep}, sigmoidinė (angl. sigmoid) \ref{eqn:sigmoid}, gabalais tiesinė (angl. piecewise linear) \ref{eqn:pieceLinear}, Gauso (angl. Gaussian) \ref{eqn:gaussian} ir tiesinė (angl. linear) \ref{eqn:linear}

\begin{equation}
\label{eqn:unitStep}
	f(a) =
	\begin{cases}
		0, & \mbox{if } \beta > a \\
		1, & \mbox{if } \beta \leq a
	\end{cases}
\end{equation}

\begin{equation}
	\label{eqn:sigmoid}
	f(a) = \dfrac{1}{1 + \exp^{-\beta a}}
\end{equation}

\begin{equation}
	\label{eqn:pieceLinear}
	f(a) =
	\begin{cases}
		0, & \mbox{if } a_{min} \geq a \\
		ma + b, & \mbox{if } a_{min} < a < a_{max} \\
		1, & \mbox{if } a_{max} \leq a
	\end{cases}
\end{equation}

\begin{equation}
	\label{eqn:gaussian}
	f(a) = \dfrac{1}{\sqrt{2\pi\sigma}} \exp^{\dfrac{-(x - \mu)^2}{2\sigma^2}}
\end{equation}

\begin{equation}
	\label{eqn:linear}
	f(a) = ma + b
\end{equation}

Perceptrono mokymas yra iteratyvus procesas, kuriame randami svoriai $W = \{w_{0}, w_{1}, w_{2}, ..., w_{n}\}$, su kuriais funkcijos \ref{eqn:mse} rezultatas įgyja mažiausią reikšmę. Funkcijoje \ref{eqn:mse} $y_i$ yra perceptrono i-tasis išėjimas ir $t_i$ - i-tojo įėjimo norima klasė.

% coming up next: rasyk kad generuoja random pirminius svorius, tada minimizavimo funkcija

\begin{equation}
\label{eqn:mse}
e(w) = \dfrac{1}{n}\sum_{i=1}^{n}(y_i - t_i)^2
\end{equation}

\subsubsection{Dirbtiniai neuroniniai tinklai}
\subsubsection{Gilieji neuroniniai tinklai}

% Apmokymas vyksta epochomis, batcho dydis

\subsection{Konvoliucinių dirbtinių neuroninių tinklų apžvalga}
\subsubsection{Konvoliucija}

Konvoliuciniai neuroniniai tinklai yra vieni populiariausių giliųjų neuroninių tinklų tipas. Pirma karta sėkmingai įgyvendintas konvoliucinis neuroninis tinklas yra aprašytas darbe \cite{cnn}. Šis tinklas yra skirtas ranka rašytiems pašto kodams atpažinti. Konvoliucinis neuroninis tinklas - tai gilusis neuroninis tinklas, kurio bent viename sluoksnyje yra naudojama konvoliucijos operacija, dar vadinama sąsuka. Konvoliucija - tai matematinė operacija, kurios operandai yra dvi funkcijos $f$ ir $g$, ir kurios rezultatas yra funkcija, kuri apibūdina kaip viena funkcija keičia kitą. Ši operacija yra žymima $f * g$ ir ji yra apibrėžiama kaip integralinės transformacijos rūšis pavaizduota formulėje \ref{eqn:convolution}, kur a ir b nurodo funkcijų  $f$ ir $g$ apibrėžimo sritį.

\begin{equation}
\label{eqn:convolution}
	(f * g)(t) = \int_{a}^{b} f(\tau)g(t - \tau) d\tau
\end{equation}

Konvoliucijos algebros savybės yra komutatyvumas ($f * g = g * f$), asociatyvumas ($f * (g * h) = (f * g) * h$), distributyvumas ($f * (g + h) = (f * g) + (f * h)$), vienetinis elementas $f * \delta = \delta * f = f$ ir daugybos su skaliaru asociatyvumas ($a(f * g) = (af) * g = f * (ag)$, kur $a \in \R$).

Dažniausiai konvoliuciniuose neuroniniuose tinkluose yra vykdoma konvoliucija diskrečioms funkcijoms. Konvoliucija, kurios operandai yra diskrečios funkcijos yra vadinama diskreti konvoliucija ir ji yra apibrėžiama kaip formulė \ref{eqn:discrete_convolution}, kur a ir b nurodo funkcijų  $f$ ir $g$ apibrėžimo sritį.

\begin{equation}
\label{eqn:discrete_convolution}
	(f * g)(t) = \sum_{\tau = a}^{b} f(\tau)g(t - \tau)
\end{equation}

Šio darbo tyrimuose yra naudojamos 2D nuotraukos, kurios yra saugomos kaip dviejų dimensijų vaizdai, vadinamos matricomis. Diskreti konvoliucija matricoms yra atliekama naudojantis formulę \ref{eqn:matrix_convolution}.

\begin{equation}
\label{eqn:matrix_convolution}
	(I * K)(i, j) = \sum_{m} \sum_{n} I(m, n) K(i - m, j - n)
\end{equation}

Konvoliuciniuose neuroniniuose tinkluose matrica $I$ yra vadinama įvestimi, o matrica $K$ - branduoliu arba filtru. Konvoliucija yra komutatyvi, todėl formulė \ref{eqn:matrix_convolution} gali būti išreikšta kaip \ref{eqn:cnn_convolution}.

\begin{equation}
\label{eqn:cnn_convolution}
	(K * I)(i, j) = \sum_{m} \sum_{n} I(i - m, j - n) K(m, n)
\end{equation}

Dažniausiai ši išraiška yra naudojama konvoliuciniuose neuroniniuose tinkluose. Branduolys $K$ dažniausiai yra žymiai mažesnio dydžio nei įvesties matrica $I$ ir $K$ dažniausiai yra išretinta matrica (angl. sparse matrix). Išretinta matrica yra matrica, kurios didžioji dalis elementų yra lygūs 0. Konvoliucijos naudojimas giliuosiuose neuroniniuose tinkluose pagreitina mašininį mokymąsi dėl konvoliucijos principų - išretintos sąveikos (angl. sparsity), parametrų pasidalinimo ir ekvivalentiško atvaizdavimo.

Paprastame daugiasluoksniame perceptrone kiekvieno sluoksnio visi perceptronai turi po vieną jungtį su tolimesnio sluoksnio kiekvienu perceptronu. Tuo metu konvoliuciniuose tinkluose yra taikoma išretinta sąveika. Išretinta sąveika - tai konvoliucijos padarinys dirbtiniam neuroniniam tinklui, dėl kurios sluoksniuose, kuriuose taikoma konvoliucija, sumažėja jungčių skaičius su tolimesnio sluoksnio perceptronais. Išretintos sąveikos tarp dviejų sluoksnių pavyzdys yra pateiktas paveikslėlyje \ref{img:sparsity}

\begin{figure}[H]
	\centering
	\includegraphics[scale=0.5]{img/sparsity.png}
	\caption{Išretinta sąveika}
	\label{img:sparsity}
\end{figure}

Dėl išretintos sąveikos konvoliuciniai neuroniniai tinklai atsižvelgia tik į reikšmingus požymius. Todėl konvoliucinių neuroninių tinklų mokymas trunka trumpiau ir triukšmas turi mažesnę įtaką rezultatui.

Kitas principas dėl kurio konvoliuciniai neuroniniai tinklai yra spartesni nei daugiasluoksniai perceptronai yra parametrų pasidalinimas. Parametrų pasidalinimas yra 


\subsection{Kapsulinių dirbtinių neuroninių tinklų apžvalga}
Kapsuliniai neuroniniai tinklai yra aprašyti darbe \cite{capsNet}. Kapsuliniai neuroniniai tinklai yra giliųjų neuroninių tinklų tipas, kurio sluoksnio perceptronai yra grupuojami į kapsules. Kiekviena kapsulė apskaičiuoja tikimybę, kad paveikslėlyje pavaizduotas objektas priklauso kažkuriai klasei, ir išgauna informaciją apie tokius objekto bruožus kaip pozicija, orientacija, mastelis, deformacija, spalva ir kitus panašius objekto bruožus. Pirminių kapsulių sluoksniuose nagrinėjami objektai yra paprastos geometrinės figūros. Tolimesniuose sluoksniuose objektai darosi sudėtingesni, jie ima atitikti realaus pasaulio objektus. Kapsulės tarp sluoksnių yra sujungiamos į hierarchiją. Taip kapsulinis neuroninis tinklas sukuria hierarchinę vaizdo reprezentaciją.

Pirmieji du sluoksniai kapsuliniame neuroniniame tinkle yra konvoliucijos sluoksnis ir apjungtas netiesiškumo ir ištaisymo sluoksnis su ištaisymo tiesine aktyvacijos funkcija. Šių sluoksnių tikslas yra išgauti pagrindinius požymius, kurie tolimesniame sluoksnyje yra naudojami objektų konstrukcijai.

Tolimesnio sluoksnio tipas yra pirminės kapsulės (angl. primary capsules). Šiame sluoksnyje aktyvacijos žemėlapiai yra konvertuojami į vektorius. Toliau kiekvienas vektorius atskirai yra pateikiamas squash funkcijai kaip argumentai. Squash funkcija yra formulė \ref{label}, kur $||s||$ yra visų matricos $s$ narių suma.

\begin{equation}
\label{eqn:squash}
	squash(s) = \dfrac{||s||^2}{1 + ||s||^2}\dfrac{s}{||s||}
\end{equation}

Tolimesnių sluoksnių tipai yra kapsulės. Šiuose sluoksniuose yra vykdomas dinaminis maršrutizavimas tarp kapsulių (angl. dynamic routing between capsules). Dinaminis maršrutizavimas tarp kapsulių - tai iteratyvus procesas, kurio paskirtis yra apjungti kapsules tarp dviejų sluoksnių. Prieš pradedant iteratyvią proceso dalį, kiekvienai sluoksnio $l$ kapsulei $i$ ir sluoksnio $(l + 1)$ kapsulei $j$ yra inicializuojami kintamieji $b_{ij}$ su reikšme 0. Taip pat kiekvienai kapsulių $i$ ir $j$ porai yra apskaičiuojami vektoriai $\hat{u}_{j|i}$ pagal formulę \ref{eqn:pred_vectors}, kur $W_{ij}$ yra svorio matrica tarp kapsulių $i$ ir $j$ bei $u_{i}$ - tai kapsulės $i$ išvestis.

\begin{equation}
\label{eqn:pred_vectors}
	\hat{u}_{j|i} = W_{ij} u_{ij}
\end{equation}

Tada pirmasis iteratyvaus proceso žingsnis yra apskaičiuoti apjungimo koeficientus $c_{ij}$ kiekvienai kapsulių $i$ ir $j$ porai pagal softmax funkciją atvaizduota formulėje \ref{eqn:coupling_coef}, kur $n$ yra sluoksnio $(l + 1)$ kapsulių skaičius.

\begin{equation}
\label{eqn:coupling_coef}
	c_{ij} = \dfrac{\exp^{b_{ij}}}{\sum_{k = 1}^{n} \exp^{b_{ik}}}
\end{equation}

Tolimesnis žingsnis yra apskaičiuoti svertines sumas $s_j$ kiekvienai kapsulei $j$ naudojantis formulę \ref{eqn:weighted_sum}, kur $m$ yra kapsulių skaičius sluoksnyje $l$.

\begin{equation}
\label{eqn:weighted_sum}
	s_{j} = \sum_{i = 1}^{m} c_{ij} \hat{u}_{j|i}
\end{equation}

Toliau yra apskaičiuojami vektoriai $v_j$ kiekvienai kapsulei $j$ naudojantis softmax funkcija su argumentu $s_j$. Kitaip tariant yra apskaičiuojama formulė $v_j = softmax(s_j)$. Paskutinis iteratyvios dalies žingsnis yra pakeisti kintamųjų $b_{ij}$ reikšmes naudojantis formulę $b_{ij} = b_{ij} + \hat{u}_{j|i} v_j$.

Iteratyvi dinaminio maršrutizavimo tarp kapsulių proceso dalis yra kartojama nurodyta skaičių iteracijų ir šio proceso rezultatas yra vektorius $v_j$. Šiame vektoriuje yra tikimybės, kad objektas, nagrinėjamas kapsulės $i$, yra dalis objekto, nagrinėjamo kapsulės $j$.


\section{Kapsulinių neuroninių tinklų modifikacijos ir parametrai}

\section{Eksperimentiniai tyrimai}

\subsection{Tyrimams naudoti duomenys}

\subsection{Kapsulinių neuroninių tinklų modifikacijų ir parametrų eksperimentiniai tyrimai}

\subsection{Kapsulinių ir konvoliucinių neuroninių tinklų eksperimentiniai tyrimai}

Šiame darbe yra palyginami 3 neuroninių tinklų architektūros: daugiavaizdis neuroninis tinklas, aprašytas poskyryje Daugiavaizdis konvoliucinis neuroninis tinklas, kapsulinis neuroninis tinklas, aprašytas poskyryje Tiriamo kapsulinio neuroninio tinklo architektūra, ir daugiavaizdis kapsulinis neuroninis tinklas, aprašytas poskyryje Tiriamo daugiavaizdžio kapsulinio neuroninio tinklo architektūra. Kiekvienas tiriamas dirbtinis neuroninis tinklas yra apmokomas naudojantis visais duomenimis, aprašytais poskyryje Tyrimams naudoti duomenys. Šie duomenys apmokymo metu yra padalinami į duomenų rinkinius, kurių dydžiai yra 96. Daugiavaizdžio konvoliucinio ir kapsulinio neuroninių tinklų apmokymų antram etapui duomenų rinkiniai buvo sudaryti iš nuotraukų grupių, kuriose yra visos konkretaus 3D objekto modelio nuotraukos. Kiekvienas dirbtinis neuroninis tinklas yra apmokomas per 10 epochų. Daugiavaizdžio konvoliucinio ir kapsulinio neuroninių tinklų abu apmokymo etapai yra apmokomi po 5 epochas.

Po kiekvienos epochos yra renkamos tikslumo metrikos: tikslumas klasifikuojant apmokymo duomenis, tikslumas klasifikuojant testavimo duomenis, nuostolių funkcijos rezultatas klasifikuojant apmokymo duomenis ir nuostolių funkcijos rezultatas klasifikuojant testavimo duomenis. Tikslumas yra teisingai suklasifikuotų įrašų dalis klasifikuotų duomenų aibėje. Dauigavaizdžiame kapsuliniame neuroniniame tinkle ir kapsuliniame neuroniniame tinkle naudojama nuostolių funkcija yra margin nuostolių funkcija, o daugiavaizdžiame konvoliuciniame neuroniniame tinkle - kryžminės entropijos nuostolių funkcija.


\sectionnonum{Rezultatai ir išvados}

Šiame magistro baigiamame darbe atlikta:

\begin{enumerate}
	\item Aprašytos ir realizuotos kapsulinių neuroninių tinklų dvi modifikacijos - daugiavaizdžiai kapsuliniai neuroniniai tinklai, iš kurių su vienas yra su vaizdų sujungimo sluoksniu ir kitas su vaizdų kapsuliniu sluoksniu.
	\item Atlikti tyrimai skirti nustatyti geriausią kapsulinių neuroninių tinklų modifikacijų konfigūracijas ir palyginti jas su nemodifikuotu kapsuliniu neuroniniu tinklu.
	\item Atlikti tyrimai, skirti palyginti kapsulinių neuroninių tinklų modifikacijas su dabartiniu geriausiu daugiavaizdžiu konvoliuciniu neuroniniu tinklu.
\end{enumerate}

Atlikus tyrimus šiame magistro baigiamame darbe gautos tokios išvados:

\begin{enumerate}
	\item Daugiavaizdžiai kapsuliniai neuroniniai tinklai pasiekia geresnius rezultatus per panašų apmokymo laiką nei kapsuliniai neuroniniai tinklai.
	\item Skirtumas tarp daugiavaizdžių kapsulinių neuroninių tinklų, iš kurių vienas yra su vaizdų kapsuliniu sluoksniu ir kitas su vaizdų sujungimo sluoksniu, yra nereikšmingas. Tačiau daugiavaizdžio kapsulinio neuroninio tinklo su vaizdų kapsuliniu sluoksniu, kuris apmokomas vienu etapu, tikslumas pradeda konverguoti greičiau nei daugiavaizdis kapsulinis neuroninis tinklas su vaizdų sujungimo sluoksniu. Tad daugiavaizdis kapsulinis neuroninis tinklas su vaizdų kapsuliniu sluoksniu yra pranašesnis laiko atžvilgiu.
	\item Nei viena tirta kapsulinio neuroninio tinklo modifikacija nėra pranašesnė už dabartinį geriausią daugiavaizdį konvoliucinį neuroninį tinklą.
	\item Kuo mažesnė duomenų imtis, tuo skirtumas tarp daugiavaizdžio kapsulinio neuroninio tinklo su vaizdų kapsuliniu sluoksniu, kuris yra apmokomas vienu etapu, ir dabartinio geriausio daugiavaizdžio konvoliucinio neuroninio tinklo yra nereikšmingesnis.
\end{enumerate}

Daugiavaizdžiui konvoliuciniui neuroniniui tinklui yra sudėtinga pritaikyti vaizdų kapsulinį sluoksnį nepaverčiant jo daugiavaizdžiu kapsuliniu neuroniniu tinklu. Tačiau pirmieji kapsulinių neuroninių tinklų sluoksniai sudaro konvoliucinį neuroninį tinklą. Todėl egzistuoja galimybė apjungti VGG-11 architektūrą, kuri naudojama dabartinio geriausio daugiavaizdžio konvoliucinio neuroninio tinklo pirmojo etapo apmokyme, su daugiavaizdžiu kapsuliniu neuroniniu tinklu su vaizdų kapsuliniu sluoksniu. Tad ateityje reikia tyrimais palyginti šį apjungtą dirbtinį neuroninį tinklą su dabartiniu geriausiu daugiavaizdžiu konvoliuciniu neuroniniu tinklu.

Rezultatų ir išvadų dalyje išdėstomi pagrindiniai darbo rezultatai (kažkas
išanalizuota, kažkas sukurta, kažkas įdiegta), pateikiamos išvados (daromi
nagrinėtų problemų sprendimo metodų palyginimai, siūlomos rekomendacijos,
akcentuojamos naujovės).

\printbibliography[heading=bibintoc]  % Literatūros šaltiniai aprašomi
% bibliografija.bib faile. Šaltinių sąraše nurodoma panaudota literatūra,
% kitokie šaltiniai. Abėcėlės tvarka išdėstoma tik darbe panaudotų (cituotų,
% perfrazuotų ar bent paminėtų) mokslo leidinių, kitokių publikacijų
% bibliografiniai aprašai (šiuo punktu pasirūpina LaTeX). Aprašai pateikiami
% netransliteruoti.

% \sectionnonum{Sąvokų apibrėžimai}
\sectionnonum{Santrumpos}
Sąvokų apibrėžimai ir santrumpų sąrašas sudaromas tada, kai darbo tekste
vartojami specialūs terminai, reikalaujantys paaiškinimo, ir rečiau sutinkamos
santrumpos.

\appendix  % Priedai
% Prieduose gali būti pateikiama pagalbinė, ypač darbo autoriaus savarankiškai
% parengta, medžiaga. Savarankiški priedai gali būti pateikiami kompiuterio
% diskelyje ar kompaktiniame diske. Priedai taip pat vadinami ir numeruojami.
% Tekstas su priedais siejamas nuorodomis.

\section{Niauroninio tinklo struktūra}

\section{Eksperimentinio palyginimo rezultatai}
% tablesgenerator.com - converts calculators (e.g. excel) tables to LaTeX
\begin{table}[H]\footnotesize
  \centering
  \caption{Lentelės pavyzdys}
  {\begin{tabular}{|l|c|c|} \hline
    Algoritmas & $\bar{x}$ & $\sigma^{2}$ \\
    \hline
    Algoritmas A  & 1.6335    & 0.5584       \\
    Algoritmas B  & 1.7395    & 0.5647       \\
    \hline
  \end{tabular}}
  \label{tab:table example}
\end{table}

\end{document}
